% Generated by Sphinx.
\def\sphinxdocclass{report}
\documentclass[letterpaper,10pt,english]{sphinxmanual}
\usepackage[utf8]{inputenc}
\DeclareUnicodeCharacter{00A0}{\nobreakspace}
\usepackage[T1]{fontenc}
\usepackage{babel}
\usepackage{times}
\usepackage[Bjarne]{fncychap}
\usepackage{longtable}
\usepackage{sphinx}


\title{DataMYNE Documentation}
\date{May 16, 2011}
\release{1.0.0}
\author{The New School}
\newcommand{\sphinxlogo}{}
\renewcommand{\releasename}{Release}
\makeindex

\makeatletter
\def\PYG@reset{\let\PYG@it=\relax \let\PYG@bf=\relax%
    \let\PYG@ul=\relax \let\PYG@tc=\relax%
    \let\PYG@bc=\relax \let\PYG@ff=\relax}
\def\PYG@tok#1{\csname PYG@tok@#1\endcsname}
\def\PYG@toks#1+{\ifx\relax#1\empty\else%
    \PYG@tok{#1}\expandafter\PYG@toks\fi}
\def\PYG@do#1{\PYG@bc{\PYG@tc{\PYG@ul{%
    \PYG@it{\PYG@bf{\PYG@ff{#1}}}}}}}
\def\PYG#1#2{\PYG@reset\PYG@toks#1+\relax+\PYG@do{#2}}

\def\PYG@tok@gd{\def\PYG@tc##1{\textcolor[rgb]{0.63,0.00,0.00}{##1}}}
\def\PYG@tok@gu{\let\PYG@bf=\textbf\def\PYG@tc##1{\textcolor[rgb]{0.50,0.00,0.50}{##1}}}
\def\PYG@tok@gt{\def\PYG@tc##1{\textcolor[rgb]{0.00,0.25,0.82}{##1}}}
\def\PYG@tok@gs{\let\PYG@bf=\textbf}
\def\PYG@tok@gr{\def\PYG@tc##1{\textcolor[rgb]{1.00,0.00,0.00}{##1}}}
\def\PYG@tok@cm{\let\PYG@it=\textit\def\PYG@tc##1{\textcolor[rgb]{0.25,0.50,0.56}{##1}}}
\def\PYG@tok@vg{\def\PYG@tc##1{\textcolor[rgb]{0.73,0.38,0.84}{##1}}}
\def\PYG@tok@m{\def\PYG@tc##1{\textcolor[rgb]{0.13,0.50,0.31}{##1}}}
\def\PYG@tok@mh{\def\PYG@tc##1{\textcolor[rgb]{0.13,0.50,0.31}{##1}}}
\def\PYG@tok@cs{\def\PYG@tc##1{\textcolor[rgb]{0.25,0.50,0.56}{##1}}\def\PYG@bc##1{\colorbox[rgb]{1.00,0.94,0.94}{##1}}}
\def\PYG@tok@ge{\let\PYG@it=\textit}
\def\PYG@tok@vc{\def\PYG@tc##1{\textcolor[rgb]{0.73,0.38,0.84}{##1}}}
\def\PYG@tok@il{\def\PYG@tc##1{\textcolor[rgb]{0.13,0.50,0.31}{##1}}}
\def\PYG@tok@go{\def\PYG@tc##1{\textcolor[rgb]{0.19,0.19,0.19}{##1}}}
\def\PYG@tok@cp{\def\PYG@tc##1{\textcolor[rgb]{0.00,0.44,0.13}{##1}}}
\def\PYG@tok@gi{\def\PYG@tc##1{\textcolor[rgb]{0.00,0.63,0.00}{##1}}}
\def\PYG@tok@gh{\let\PYG@bf=\textbf\def\PYG@tc##1{\textcolor[rgb]{0.00,0.00,0.50}{##1}}}
\def\PYG@tok@ni{\let\PYG@bf=\textbf\def\PYG@tc##1{\textcolor[rgb]{0.84,0.33,0.22}{##1}}}
\def\PYG@tok@nl{\let\PYG@bf=\textbf\def\PYG@tc##1{\textcolor[rgb]{0.00,0.13,0.44}{##1}}}
\def\PYG@tok@nn{\let\PYG@bf=\textbf\def\PYG@tc##1{\textcolor[rgb]{0.05,0.52,0.71}{##1}}}
\def\PYG@tok@no{\def\PYG@tc##1{\textcolor[rgb]{0.38,0.68,0.84}{##1}}}
\def\PYG@tok@na{\def\PYG@tc##1{\textcolor[rgb]{0.25,0.44,0.63}{##1}}}
\def\PYG@tok@nb{\def\PYG@tc##1{\textcolor[rgb]{0.00,0.44,0.13}{##1}}}
\def\PYG@tok@nc{\let\PYG@bf=\textbf\def\PYG@tc##1{\textcolor[rgb]{0.05,0.52,0.71}{##1}}}
\def\PYG@tok@nd{\let\PYG@bf=\textbf\def\PYG@tc##1{\textcolor[rgb]{0.33,0.33,0.33}{##1}}}
\def\PYG@tok@ne{\def\PYG@tc##1{\textcolor[rgb]{0.00,0.44,0.13}{##1}}}
\def\PYG@tok@nf{\def\PYG@tc##1{\textcolor[rgb]{0.02,0.16,0.49}{##1}}}
\def\PYG@tok@si{\let\PYG@it=\textit\def\PYG@tc##1{\textcolor[rgb]{0.44,0.63,0.82}{##1}}}
\def\PYG@tok@s2{\def\PYG@tc##1{\textcolor[rgb]{0.25,0.44,0.63}{##1}}}
\def\PYG@tok@vi{\def\PYG@tc##1{\textcolor[rgb]{0.73,0.38,0.84}{##1}}}
\def\PYG@tok@nt{\let\PYG@bf=\textbf\def\PYG@tc##1{\textcolor[rgb]{0.02,0.16,0.45}{##1}}}
\def\PYG@tok@nv{\def\PYG@tc##1{\textcolor[rgb]{0.73,0.38,0.84}{##1}}}
\def\PYG@tok@s1{\def\PYG@tc##1{\textcolor[rgb]{0.25,0.44,0.63}{##1}}}
\def\PYG@tok@gp{\let\PYG@bf=\textbf\def\PYG@tc##1{\textcolor[rgb]{0.78,0.36,0.04}{##1}}}
\def\PYG@tok@sh{\def\PYG@tc##1{\textcolor[rgb]{0.25,0.44,0.63}{##1}}}
\def\PYG@tok@ow{\let\PYG@bf=\textbf\def\PYG@tc##1{\textcolor[rgb]{0.00,0.44,0.13}{##1}}}
\def\PYG@tok@sx{\def\PYG@tc##1{\textcolor[rgb]{0.78,0.36,0.04}{##1}}}
\def\PYG@tok@bp{\def\PYG@tc##1{\textcolor[rgb]{0.00,0.44,0.13}{##1}}}
\def\PYG@tok@c1{\let\PYG@it=\textit\def\PYG@tc##1{\textcolor[rgb]{0.25,0.50,0.56}{##1}}}
\def\PYG@tok@kc{\let\PYG@bf=\textbf\def\PYG@tc##1{\textcolor[rgb]{0.00,0.44,0.13}{##1}}}
\def\PYG@tok@c{\let\PYG@it=\textit\def\PYG@tc##1{\textcolor[rgb]{0.25,0.50,0.56}{##1}}}
\def\PYG@tok@mf{\def\PYG@tc##1{\textcolor[rgb]{0.13,0.50,0.31}{##1}}}
\def\PYG@tok@err{\def\PYG@bc##1{\fcolorbox[rgb]{1.00,0.00,0.00}{1,1,1}{##1}}}
\def\PYG@tok@kd{\let\PYG@bf=\textbf\def\PYG@tc##1{\textcolor[rgb]{0.00,0.44,0.13}{##1}}}
\def\PYG@tok@ss{\def\PYG@tc##1{\textcolor[rgb]{0.32,0.47,0.09}{##1}}}
\def\PYG@tok@sr{\def\PYG@tc##1{\textcolor[rgb]{0.14,0.33,0.53}{##1}}}
\def\PYG@tok@mo{\def\PYG@tc##1{\textcolor[rgb]{0.13,0.50,0.31}{##1}}}
\def\PYG@tok@mi{\def\PYG@tc##1{\textcolor[rgb]{0.13,0.50,0.31}{##1}}}
\def\PYG@tok@kn{\let\PYG@bf=\textbf\def\PYG@tc##1{\textcolor[rgb]{0.00,0.44,0.13}{##1}}}
\def\PYG@tok@o{\def\PYG@tc##1{\textcolor[rgb]{0.40,0.40,0.40}{##1}}}
\def\PYG@tok@kr{\let\PYG@bf=\textbf\def\PYG@tc##1{\textcolor[rgb]{0.00,0.44,0.13}{##1}}}
\def\PYG@tok@s{\def\PYG@tc##1{\textcolor[rgb]{0.25,0.44,0.63}{##1}}}
\def\PYG@tok@kp{\def\PYG@tc##1{\textcolor[rgb]{0.00,0.44,0.13}{##1}}}
\def\PYG@tok@w{\def\PYG@tc##1{\textcolor[rgb]{0.73,0.73,0.73}{##1}}}
\def\PYG@tok@kt{\def\PYG@tc##1{\textcolor[rgb]{0.56,0.13,0.00}{##1}}}
\def\PYG@tok@sc{\def\PYG@tc##1{\textcolor[rgb]{0.25,0.44,0.63}{##1}}}
\def\PYG@tok@sb{\def\PYG@tc##1{\textcolor[rgb]{0.25,0.44,0.63}{##1}}}
\def\PYG@tok@k{\let\PYG@bf=\textbf\def\PYG@tc##1{\textcolor[rgb]{0.00,0.44,0.13}{##1}}}
\def\PYG@tok@se{\let\PYG@bf=\textbf\def\PYG@tc##1{\textcolor[rgb]{0.25,0.44,0.63}{##1}}}
\def\PYG@tok@sd{\let\PYG@it=\textit\def\PYG@tc##1{\textcolor[rgb]{0.25,0.44,0.63}{##1}}}

\def\PYGZbs{\char`\\}
\def\PYGZus{\char`\_}
\def\PYGZob{\char`\{}
\def\PYGZcb{\char`\}}
\def\PYGZca{\char`\^}
\def\PYGZsh{\char`\#}
\def\PYGZpc{\char`\%}
\def\PYGZdl{\char`\$}
\def\PYGZti{\char`\~}
% for compatibility with earlier versions
\def\PYGZat{@}
\def\PYGZlb{[}
\def\PYGZrb{]}
\makeatother

\begin{document}

\maketitle
\tableofcontents
\phantomsection\label{index::doc}


Contents:

\begin{tabulary}{\linewidth}{LL}
\hline

{\hyperref[generated/apps.profiles.models:module-apps.profiles.models]{\code{profiles.models}}}
 & 

\\

{\hyperref[generated/apps.profiles.views:module-apps.profiles.views]{\code{profiles.views}}}
 & 

\\

{\hyperref[generated/apps.profiles.fields:module-apps.profiles.fields]{\code{profiles.fields}}}
 & 
Created on Apr 27, 2011
\\

{\hyperref[generated/apps.profiles.forms:module-apps.profiles.forms]{\code{profiles.forms}}}
 & 

\\

{\hyperref[generated/apps.profiles.handlers:module-apps.profiles.handlers]{\code{profiles.handlers}}}
 & 
Created on Aug 18, 2010
\\

{\hyperref[generated/apps.profiles.lookups:module-apps.profiles.lookups]{\code{profiles.lookups}}}
 & 

\\

{\hyperref[generated/apps.profiles.backends:module-apps.profiles.backends]{\code{profiles.backends}}}
 & 
Created on Mar 2, 2011
\\

{\hyperref[generated/apps.reporting.models:module-apps.reporting.models]{\code{reporting.models}}}
 & 

\\

{\hyperref[generated/apps.reporting.views:module-apps.reporting.views]{\code{reporting.views}}}
 & 

\\

{\hyperref[generated/apps.reporting.forms:module-apps.reporting.forms]{\code{reporting.forms}}}
 & 
Created on Apr 7, 2011
\\

{\hyperref[generated/apps.reporting.handlers:module-apps.reporting.handlers]{\code{reporting.handlers}}}
 & 
Created on Aug 18, 2010
\\

{\hyperref[generated/apps.mobile.views:module-apps.mobile.views]{\code{mobile.views}}}
 & 

\\
\hline
\end{tabulary}



\chapter{Profiles: Models}
\label{generated/apps.profiles.models:welcome-to-datamyne-s-documentation}\label{generated/apps.profiles.models:module-apps.profiles.models}\label{generated/apps.profiles.models::doc}\label{generated/apps.profiles.models:profiles-models}
\index{apps.profiles.models (module)}
\index{ActivePersonManager (class in apps.profiles.models)}

\begin{fulllineitems}
\phantomsection\label{generated/apps.profiles.models:apps.profiles.models.ActivePersonManager}\pysigline{\strong{class }\code{apps.profiles.models.}\bfcode{ActivePersonManager}}{}
Bases: \code{django.db.models.manager.Manager}

The \code{ActivePersonManager} supports the \code{Person} class by restricting
queries to only those people who have active user accounts.

\end{fulllineitems}


\index{AreaOfStudy (class in apps.profiles.models)}

\begin{fulllineitems}
\phantomsection\label{generated/apps.profiles.models:apps.profiles.models.AreaOfStudy}\pysiglinewithargsret{\strong{class }\code{apps.profiles.models.}\bfcode{AreaOfStudy}}{\emph{*args}, \emph{**kwargs}}{}
Bases: {\hyperref[generated/apps.profiles.models:apps.profiles.models.BaseModel]{\code{apps.profiles.models.BaseModel}}}

An \code{AreaOfStudy} is a more amorphous grouping of courses that are not
necessarily organized under the traditional hierarchy.

\index{unit\_permissions (apps.profiles.models.AreaOfStudy attribute)}

\begin{fulllineitems}
\phantomsection\label{generated/apps.profiles.models:apps.profiles.models.AreaOfStudy.unit_permissions}\pysigline{\bfcode{unit\_permissions}}{}
This class provides the functionality that makes the related-object
managers available as attributes on a model class, for fields that have
multiple ``remote'' values and have a GenericRelation defined in their model
(rather than having another model pointed \emph{at} them). In the example
``article.publications'', the publications attribute is a
ReverseGenericRelatedObjectsDescriptor instance.

\end{fulllineitems}


\end{fulllineitems}


\index{BaseModel (class in apps.profiles.models)}

\begin{fulllineitems}
\phantomsection\label{generated/apps.profiles.models:apps.profiles.models.BaseModel}\pysiglinewithargsret{\strong{class }\code{apps.profiles.models.}\bfcode{BaseModel}}{\emph{*args}, \emph{**kwargs}}{}
Bases: \code{django.db.models.base.Model}

BaseModel is the root of almost all other objects within the DataMYNE
system. It establishes the creation and modification date fields, as well 
as indicating which user last changed the given model.  It also stubs out
the permissions for a \code{auth.User} and other objects based upon their 
associations within the university's hierarchy of ``units'' 
(e.g. \code{profiles.Division}, \code{profiles.School}, etc.)

\index{get\_unit() (apps.profiles.models.BaseModel method)}

\begin{fulllineitems}
\phantomsection\label{generated/apps.profiles.models:apps.profiles.models.BaseModel.get_unit}\pysiglinewithargsret{\bfcode{get\_unit}}{}{}
get\_unit returns the organizational unit that hold responsibility
for this object.  For example, a committee executing get\_unit could
return the division that has authority over it.

\end{fulllineitems}


\index{has\_unit\_permission() (apps.profiles.models.BaseModel method)}

\begin{fulllineitems}
\phantomsection\label{generated/apps.profiles.models:apps.profiles.models.BaseModel.has_unit_permission}\pysiglinewithargsret{\bfcode{has\_unit\_permission}}{\emph{user}}{}
Given a user, this method will check to see if that user is
attached to the appropriate organizational unit to have access
to edit this object.  This allows people in higher levels of
the university's hierarchy the ability to edit a greater number
of objects, while restricting those lower down to objects that
only affect their unit.

\end{fulllineitems}


\index{save() (apps.profiles.models.BaseModel method)}

\begin{fulllineitems}
\phantomsection\label{generated/apps.profiles.models:apps.profiles.models.BaseModel.save}\pysiglinewithargsret{\bfcode{save}}{\emph{*args}, \emph{**kwargs}}{}
Every DataMyne object has a \code{created\_by} field that contains the \code{User}
object of the user who was logged in when the system created the
object.  The method uses the datamining.middleware module to access
another thread and check on the current user.

\end{fulllineitems}


\index{unit\_permissions (apps.profiles.models.BaseModel attribute)}

\begin{fulllineitems}
\phantomsection\label{generated/apps.profiles.models:apps.profiles.models.BaseModel.unit_permissions}\pysigline{\bfcode{unit\_permissions}}{}
This class provides the functionality that makes the related-object
managers available as attributes on a model class, for fields that have
multiple ``remote'' values and have a GenericRelation defined in their model
(rather than having another model pointed \emph{at} them). In the example
``article.publications'', the publications attribute is a
ReverseGenericRelatedObjectsDescriptor instance.

\end{fulllineitems}


\end{fulllineitems}


\index{ContactEmail (class in apps.profiles.models)}

\begin{fulllineitems}
\phantomsection\label{generated/apps.profiles.models:apps.profiles.models.ContactEmail}\pysiglinewithargsret{\strong{class }\code{apps.profiles.models.}\bfcode{ContactEmail}}{\emph{*args}, \emph{**kwargs}}{}
Bases: {\hyperref[generated/apps.profiles.models:apps.profiles.models.BaseModel]{\code{apps.profiles.models.BaseModel}}}

\code{ContactEmail} is legacy code that needs to be refactored out{}`{}`

\index{unit\_permissions (apps.profiles.models.ContactEmail attribute)}

\begin{fulllineitems}
\phantomsection\label{generated/apps.profiles.models:apps.profiles.models.ContactEmail.unit_permissions}\pysigline{\bfcode{unit\_permissions}}{}
This class provides the functionality that makes the related-object
managers available as attributes on a model class, for fields that have
multiple ``remote'' values and have a GenericRelation defined in their model
(rather than having another model pointed \emph{at} them). In the example
``article.publications'', the publications attribute is a
ReverseGenericRelatedObjectsDescriptor instance.

\end{fulllineitems}


\end{fulllineitems}


\index{Course (class in apps.profiles.models)}

\begin{fulllineitems}
\phantomsection\label{generated/apps.profiles.models:apps.profiles.models.Course}\pysiglinewithargsret{\strong{class }\code{apps.profiles.models.}\bfcode{Course}}{\emph{*args}, \emph{**kwargs}}{}
Bases: {\hyperref[generated/apps.profiles.models:apps.profiles.models.BaseModel]{\code{apps.profiles.models.BaseModel}}}

A \code{Course} is an object that contains course information across time. A
\code{Section} object is connected to a course and shows only the information
that is specific to a particular semester and set of faculty.

The distinction between \code{Course} and \code{Section} is subtle but crucial.
A course effectively lives outside of time.  Faculty are never associated
with a course.  Nor are semesters.  A section, by contrast, has associated
with it a particular \code{Semester} as well as zero or many \code{FacultyMember}
objects (it is possible to have zero faculty for a \code{Section} in cases 
where the faculty assignments are TBD.)

It is Mike Edwards's \textbf{strong} recommendation that the \code{taken} field
by replaced with code that defines an \code{Affiliation}.  See the documentation
of the \code{reporting} app for a complete explanation of the design decision
to favor \code{Affiliation} objects over simple ManyToManyField fields.

Also, the \code{projects} field should be refactored and removed, since the 
\code{Project} model is deprecated.  It should be replaced with code that
relates to \code{Work} objects instead, either as a ManyToMany or something
similar to what the \code{Affiliation} object seeks to achieve between
\code{Person} objects and every other DataMYNE class.

\index{get\_unit() (apps.profiles.models.Course method)}

\begin{fulllineitems}
\phantomsection\label{generated/apps.profiles.models:apps.profiles.models.Course.get_unit}\pysiglinewithargsret{\bfcode{get\_unit}}{}{}
A \code{Course} is assumed to be under the authority of whatever
organizational unit manages the course's \code{Subject} object.

\end{fulllineitems}


\index{unit\_permissions (apps.profiles.models.Course attribute)}

\begin{fulllineitems}
\phantomsection\label{generated/apps.profiles.models:apps.profiles.models.Course.unit_permissions}\pysigline{\bfcode{unit\_permissions}}{}
This class provides the functionality that makes the related-object
managers available as attributes on a model class, for fields that have
multiple ``remote'' values and have a GenericRelation defined in their model
(rather than having another model pointed \emph{at} them). In the example
``article.publications'', the publications attribute is a
ReverseGenericRelatedObjectsDescriptor instance.

\end{fulllineitems}


\end{fulllineitems}


\index{CourseImage (class in apps.profiles.models)}

\begin{fulllineitems}
\phantomsection\label{generated/apps.profiles.models:apps.profiles.models.CourseImage}\pysiglinewithargsret{\strong{class }\code{apps.profiles.models.}\bfcode{CourseImage}}{\emph{*args}, \emph{**kwargs}}{}
Bases: {\hyperref[generated/apps.profiles.models:apps.profiles.models.BaseModel]{\code{apps.profiles.models.BaseModel}}}

A \code{CourseImage} connects a \code{Course} to an image file and associated metadata.

It is Mike Edwards's \textbf{strong} recommendation that this be refactored and
migrated into a generic \code{Image} class that can relate the same data to a
GenericForeignKey.  This would allow images to decorate any other object within
the DataMYNE system, allowing for a common set of code to handle image assets,
rather than duplicating the effort.  In addition, it may also be worth considering
how a generic \code{Image} object could inherit from a generic \code{Media} object, 
leaving open the possibility of achieving similar efficiency for video, audio, 
and other unforeseen media.

Of course, some of this addresses issues at the heart of what DataMYNE is and
what it ought to be.  The degree to which this system stores (or references)
other data depends largely on how much data needs to be internally understood
by the system (for the purposes of searching, cross-referencing, etc.) and how
much ought to be offloaded to the rest of the Web (e.g. Flickr, YouTube, etc.)
At the time of writing, this issue is still very much in flux.  As such,
designing for flexibility (instead of performance or simplicity) is paramount.

\index{unit\_permissions (apps.profiles.models.CourseImage attribute)}

\begin{fulllineitems}
\phantomsection\label{generated/apps.profiles.models:apps.profiles.models.CourseImage.unit_permissions}\pysigline{\bfcode{unit\_permissions}}{}
This class provides the functionality that makes the related-object
managers available as attributes on a model class, for fields that have
multiple ``remote'' values and have a GenericRelation defined in their model
(rather than having another model pointed \emph{at} them). In the example
``article.publications'', the publications attribute is a
ReverseGenericRelatedObjectsDescriptor instance.

\end{fulllineitems}


\end{fulllineitems}


\index{Department (class in apps.profiles.models)}

\begin{fulllineitems}
\phantomsection\label{generated/apps.profiles.models:apps.profiles.models.Department}\pysiglinewithargsret{\strong{class }\code{apps.profiles.models.}\bfcode{Department}}{\emph{*args}, \emph{**kwargs}}{}
Bases: {\hyperref[generated/apps.profiles.models:apps.profiles.models.BaseModel]{\code{apps.profiles.models.BaseModel}}}

A \code{Department} is one of the the highest ``units'' within the university.  
Below it are programs.  It sits at the same level as a \code{Department}

Historically, all divisions except Parsons the New School for Design 
have Departments.

\index{authorities (apps.profiles.models.Department attribute)}

\begin{fulllineitems}
\phantomsection\label{generated/apps.profiles.models:apps.profiles.models.Department.authorities}\pysigline{\bfcode{authorities}}{}
This class provides the functionality that makes the related-object
managers available as attributes on a model class, for fields that have
multiple ``remote'' values and have a GenericRelation defined in their model
(rather than having another model pointed \emph{at} them). In the example
``article.publications'', the publications attribute is a
ReverseGenericRelatedObjectsDescriptor instance.

\end{fulllineitems}


\index{unit\_permissions (apps.profiles.models.Department attribute)}

\begin{fulllineitems}
\phantomsection\label{generated/apps.profiles.models:apps.profiles.models.Department.unit_permissions}\pysigline{\bfcode{unit\_permissions}}{}
This class provides the functionality that makes the related-object
managers available as attributes on a model class, for fields that have
multiple ``remote'' values and have a GenericRelation defined in their model
(rather than having another model pointed \emph{at} them). In the example
``article.publications'', the publications attribute is a
ReverseGenericRelatedObjectsDescriptor instance.

\end{fulllineitems}


\end{fulllineitems}


\index{Division (class in apps.profiles.models)}

\begin{fulllineitems}
\phantomsection\label{generated/apps.profiles.models:apps.profiles.models.Division}\pysiglinewithargsret{\strong{class }\code{apps.profiles.models.}\bfcode{Division}}{\emph{*args}, \emph{**kwargs}}{}
Bases: {\hyperref[generated/apps.profiles.models:apps.profiles.models.BaseModel]{\code{apps.profiles.models.BaseModel}}}

A \code{Division} is the highest ``unit'' within the university.  Below it
are \code{Department} , \code{School}, etc.

\index{authorities (apps.profiles.models.Division attribute)}

\begin{fulllineitems}
\phantomsection\label{generated/apps.profiles.models:apps.profiles.models.Division.authorities}\pysigline{\bfcode{authorities}}{}
This class provides the functionality that makes the related-object
managers available as attributes on a model class, for fields that have
multiple ``remote'' values and have a GenericRelation defined in their model
(rather than having another model pointed \emph{at} them). In the example
``article.publications'', the publications attribute is a
ReverseGenericRelatedObjectsDescriptor instance.

\end{fulllineitems}


\index{unit\_permissions (apps.profiles.models.Division attribute)}

\begin{fulllineitems}
\phantomsection\label{generated/apps.profiles.models:apps.profiles.models.Division.unit_permissions}\pysigline{\bfcode{unit\_permissions}}{}
This class provides the functionality that makes the related-object
managers available as attributes on a model class, for fields that have
multiple ``remote'' values and have a GenericRelation defined in their model
(rather than having another model pointed \emph{at} them). In the example
``article.publications'', the publications attribute is a
ReverseGenericRelatedObjectsDescriptor instance.

\end{fulllineitems}


\end{fulllineitems}


\index{Expertise (class in apps.profiles.models)}

\begin{fulllineitems}
\phantomsection\label{generated/apps.profiles.models:apps.profiles.models.Expertise}\pysiglinewithargsret{\strong{class }\code{apps.profiles.models.}\bfcode{Expertise}}{\emph{*args}, \emph{**kwargs}}{}
Bases: {\hyperref[generated/apps.profiles.models:apps.profiles.models.BaseModel]{\code{apps.profiles.models.BaseModel}}}

An \code{Expertise} (sometimes called an area of expertise) is a canonical
keyword that users can select to describe them in their profiles.

\index{unit\_permissions (apps.profiles.models.Expertise attribute)}

\begin{fulllineitems}
\phantomsection\label{generated/apps.profiles.models:apps.profiles.models.Expertise.unit_permissions}\pysigline{\bfcode{unit\_permissions}}{}
This class provides the functionality that makes the related-object
managers available as attributes on a model class, for fields that have
multiple ``remote'' values and have a GenericRelation defined in their model
(rather than having another model pointed \emph{at} them). In the example
``article.publications'', the publications attribute is a
ReverseGenericRelatedObjectsDescriptor instance.

\end{fulllineitems}


\end{fulllineitems}


\index{FacultyMember (class in apps.profiles.models)}

\begin{fulllineitems}
\phantomsection\label{generated/apps.profiles.models:apps.profiles.models.FacultyMember}\pysiglinewithargsret{\strong{class }\code{apps.profiles.models.}\bfcode{FacultyMember}}{\emph{*args}, \emph{**kwargs}}{}
Bases: {\hyperref[generated/apps.profiles.models:apps.profiles.models.Person]{\code{apps.profiles.models.Person}}}

A \code{FacultyMember} is a \code{Person} (believe it or not) that teaches one or more
\code{Section} objects of a \code{Course} object.  Their association with a \code{Section}
is their key distinction with other \code{Person} objects.

\end{fulllineitems}


\index{Invitation (class in apps.profiles.models)}

\begin{fulllineitems}
\phantomsection\label{generated/apps.profiles.models:apps.profiles.models.Invitation}\pysiglinewithargsret{\strong{class }\code{apps.profiles.models.}\bfcode{Invitation}}{\emph{*args}, \emph{**kwargs}}{}
Bases: \code{django.db.models.base.Model}

An \code{Invitation} object allows a \code{host} user to invite a \code{guest} user
to be part of another object within the DataMYNE system.  The guest can either 
be a registered user or someone with an email address outside of the university.

Once a guest receives his or her invitation email, they are directed to follow
a url (composed of a unique \code{slug}) that will do one of the following:
\begin{itemize}
\item {} 
if the user is a member of DataMYNE and is signed in, he or she will be taken
directly to the object in question

\item {} 
if the user is a member of DataMYNE and is not signed it, he or she will be
taken first to a login screen

\item {} 
if the user is not yet a member of DataMYNE, he or she will still be directed
to the login screen.
\begin{itemize}
\item {} 
If he or she is able to join (e.g. has a listing within LDAP), a profile will
automatically be created.

\item {} 
If he or shee is not able to join, at this point, tough luck.

\end{itemize}

\end{itemize}

This last case is an interesting one, since DataMYNE still represents a closed
community.  The best way to deal with this is most likely best handled within
the \code{profiles.backends} module.  This will be most important for alumni who
no longer can authenticate through LDAP, as well as incoming students who are
not yet established within LDAP.  In any case, the \code{Invitation} model should
remain more or less agnostic to this.

\index{content\_object (apps.profiles.models.Invitation attribute)}

\begin{fulllineitems}
\phantomsection\label{generated/apps.profiles.models:apps.profiles.models.Invitation.content_object}\pysigline{\bfcode{content\_object}}{}
Provides a generic relation to any object through content-type/object-id
fields.

\end{fulllineitems}


\end{fulllineitems}


\index{Link (class in apps.profiles.models)}

\begin{fulllineitems}
\phantomsection\label{generated/apps.profiles.models:apps.profiles.models.Link}\pysiglinewithargsret{\strong{class }\code{apps.profiles.models.}\bfcode{Link}}{\emph{*args}, \emph{**kwargs}}{}
Bases: {\hyperref[generated/apps.profiles.models:apps.profiles.models.BaseModel]{\code{apps.profiles.models.BaseModel}}}

A \code{Link} allows for adding a URL, plus description and other metadata,
to any object in the system.

It should succeed \code{WorkURL} for \code{Person} links and be used for other
DataMyne objects should links become useful with them.

\index{content\_object (apps.profiles.models.Link attribute)}

\begin{fulllineitems}
\phantomsection\label{generated/apps.profiles.models:apps.profiles.models.Link.content_object}\pysigline{\bfcode{content\_object}}{}
Provides a generic relation to any object through content-type/object-id
fields.

\end{fulllineitems}


\index{unit\_permissions (apps.profiles.models.Link attribute)}

\begin{fulllineitems}
\phantomsection\label{generated/apps.profiles.models:apps.profiles.models.Link.unit_permissions}\pysigline{\bfcode{unit\_permissions}}{}
This class provides the functionality that makes the related-object
managers available as attributes on a model class, for fields that have
multiple ``remote'' values and have a GenericRelation defined in their model
(rather than having another model pointed \emph{at} them). In the example
``article.publications'', the publications attribute is a
ReverseGenericRelatedObjectsDescriptor instance.

\end{fulllineitems}


\end{fulllineitems}


\index{Organization (class in apps.profiles.models)}

\begin{fulllineitems}
\phantomsection\label{generated/apps.profiles.models:apps.profiles.models.Organization}\pysiglinewithargsret{\strong{class }\code{apps.profiles.models.}\bfcode{Organization}}{\emph{*args}, \emph{**kwargs}}{}
Bases: {\hyperref[generated/apps.profiles.models:apps.profiles.models.BaseModel]{\code{apps.profiles.models.BaseModel}}}

An \code{Organization} is an object that represents any number of kinds of groups
that may appear within the university.  It comprises everything from officially
designated labs to ad hoc student groups.

The \code{projects} field should be refactored out and, most likely, replaced with
some kind of relationship to the \code{Work} model (either via a new ManyToMany field
or something akin to how the \code{Person} models relates to other objects via the
\code{Affiliation}

\index{affiliations (apps.profiles.models.Organization attribute)}

\begin{fulllineitems}
\phantomsection\label{generated/apps.profiles.models:apps.profiles.models.Organization.affiliations}\pysigline{\bfcode{affiliations}}{}
This class provides the functionality that makes the related-object
managers available as attributes on a model class, for fields that have
multiple ``remote'' values and have a GenericRelation defined in their model
(rather than having another model pointed \emph{at} them). In the example
``article.publications'', the publications attribute is a
ReverseGenericRelatedObjectsDescriptor instance.

\end{fulllineitems}


\index{has\_unit\_permission() (apps.profiles.models.Organization method)}

\begin{fulllineitems}
\phantomsection\label{generated/apps.profiles.models:apps.profiles.models.Organization.has_unit_permission}\pysiglinewithargsret{\bfcode{has\_unit\_permission}}{\emph{user}}{}
Organization models don't have a parent unit, therefore there are no unit 
restrictions on this.  This situation may change if the \code{Sponsorship}
object becomes more widely used.

\end{fulllineitems}


\index{meetings (apps.profiles.models.Organization attribute)}

\begin{fulllineitems}
\phantomsection\label{generated/apps.profiles.models:apps.profiles.models.Organization.meetings}\pysigline{\bfcode{meetings}}{}
This class provides the functionality that makes the related-object
managers available as attributes on a model class, for fields that have
multiple ``remote'' values and have a GenericRelation defined in their model
(rather than having another model pointed \emph{at} them). In the example
``article.publications'', the publications attribute is a
ReverseGenericRelatedObjectsDescriptor instance.

\end{fulllineitems}


\index{unit\_permissions (apps.profiles.models.Organization attribute)}

\begin{fulllineitems}
\phantomsection\label{generated/apps.profiles.models:apps.profiles.models.Organization.unit_permissions}\pysigline{\bfcode{unit\_permissions}}{}
This class provides the functionality that makes the related-object
managers available as attributes on a model class, for fields that have
multiple ``remote'' values and have a GenericRelation defined in their model
(rather than having another model pointed \emph{at} them). In the example
``article.publications'', the publications attribute is a
ReverseGenericRelatedObjectsDescriptor instance.

\end{fulllineitems}


\end{fulllineitems}


\index{OrganizationType (class in apps.profiles.models)}

\begin{fulllineitems}
\phantomsection\label{generated/apps.profiles.models:apps.profiles.models.OrganizationType}\pysiglinewithargsret{\strong{class }\code{apps.profiles.models.}\bfcode{OrganizationType}}{\emph{*args}, \emph{**kwargs}}{}
Bases: {\hyperref[generated/apps.profiles.models:apps.profiles.models.BaseModel]{\code{apps.profiles.models.BaseModel}}}

An \code{OrganizationType} defines a canonical category for \code{Organization}
objects (e.g. lab, center, institute, etc.)

\index{unit\_permissions (apps.profiles.models.OrganizationType attribute)}

\begin{fulllineitems}
\phantomsection\label{generated/apps.profiles.models:apps.profiles.models.OrganizationType.unit_permissions}\pysigline{\bfcode{unit\_permissions}}{}
This class provides the functionality that makes the related-object
managers available as attributes on a model class, for fields that have
multiple ``remote'' values and have a GenericRelation defined in their model
(rather than having another model pointed \emph{at} them). In the example
``article.publications'', the publications attribute is a
ReverseGenericRelatedObjectsDescriptor instance.

\end{fulllineitems}


\end{fulllineitems}


\index{Person (class in apps.profiles.models)}

\begin{fulllineitems}
\phantomsection\label{generated/apps.profiles.models:apps.profiles.models.Person}\pysiglinewithargsret{\strong{class }\code{apps.profiles.models.}\bfcode{Person}}{\emph{*args}, \emph{**kwargs}}{}
Bases: {\hyperref[generated/apps.profiles.models:apps.profiles.models.BaseModel]{\code{apps.profiles.models.BaseModel}}}

A \code{Person} represents any number of roles within the university, but
provides a unifying class for dealing with all of them.  The \code{Person}
has fields for first and last name, N Number, etc. as well as several
crucial class and object methods that pertain to all members of the
university.

A \code{Student}, \code{Staff}, and \code{FacultyMember} object are all of 
the \code{Person} type.

\index{activate() (apps.profiles.models.Person method)}

\begin{fulllineitems}
\phantomsection\label{generated/apps.profiles.models:apps.profiles.models.Person.activate}\pysiglinewithargsret{\bfcode{activate}}{\emph{email}}{}
An email-based activation method.  This is deprecated since the 
introduction of the LDAP authentication

\end{fulllineitems}


\index{cv\_text (apps.profiles.models.Person attribute)}

\begin{fulllineitems}
\phantomsection\label{generated/apps.profiles.models:apps.profiles.models.Person.cv_text}\pysigline{\bfcode{cv\_text}}{}
Docstring

\end{fulllineitems}


\index{deactivate() (apps.profiles.models.Person method)}

\begin{fulllineitems}
\phantomsection\label{generated/apps.profiles.models:apps.profiles.models.Person.deactivate}\pysiglinewithargsret{\bfcode{deactivate}}{}{}
Deactivates a person's user account

\end{fulllineitems}


\index{group\_perms\_set (apps.profiles.models.Person attribute)}

\begin{fulllineitems}
\phantomsection\label{generated/apps.profiles.models:apps.profiles.models.Person.group_perms_set}\pysigline{\bfcode{group\_perms\_set}}{}
This class provides the functionality that makes the related-object
managers available as attributes on a model class, for fields that have
multiple ``remote'' values and have a GenericRelation defined in their model
(rather than having another model pointed \emph{at} them). In the example
``article.publications'', the publications attribute is a
ReverseGenericRelatedObjectsDescriptor instance.

\end{fulllineitems}


\index{unit\_permissions (apps.profiles.models.Person attribute)}

\begin{fulllineitems}
\phantomsection\label{generated/apps.profiles.models:apps.profiles.models.Person.unit_permissions}\pysigline{\bfcode{unit\_permissions}}{}
This class provides the functionality that makes the related-object
managers available as attributes on a model class, for fields that have
multiple ``remote'' values and have a GenericRelation defined in their model
(rather than having another model pointed \emph{at} them). In the example
``article.publications'', the publications attribute is a
ReverseGenericRelatedObjectsDescriptor instance.

\end{fulllineitems}


\index{user\_perms\_set (apps.profiles.models.Person attribute)}

\begin{fulllineitems}
\phantomsection\label{generated/apps.profiles.models:apps.profiles.models.Person.user_perms_set}\pysigline{\bfcode{user\_perms\_set}}{}
This class provides the functionality that makes the related-object
managers available as attributes on a model class, for fields that have
multiple ``remote'' values and have a GenericRelation defined in their model
(rather than having another model pointed \emph{at} them). In the example
``article.publications'', the publications attribute is a
ReverseGenericRelatedObjectsDescriptor instance.

\end{fulllineitems}


\end{fulllineitems}


\index{Program (class in apps.profiles.models)}

\begin{fulllineitems}
\phantomsection\label{generated/apps.profiles.models:apps.profiles.models.Program}\pysiglinewithargsret{\strong{class }\code{apps.profiles.models.}\bfcode{Program}}{\emph{*args}, \emph{**kwargs}}{}
Bases: {\hyperref[generated/apps.profiles.models:apps.profiles.models.BaseModel]{\code{apps.profiles.models.BaseModel}}}

A \code{Program} is typically the lowest organizational unit in the university.
It sits below either a \code{School} or a \code{Department}.

\index{affiliations (apps.profiles.models.Program attribute)}

\begin{fulllineitems}
\phantomsection\label{generated/apps.profiles.models:apps.profiles.models.Program.affiliations}\pysigline{\bfcode{affiliations}}{}
This class provides the functionality that makes the related-object
managers available as attributes on a model class, for fields that have
multiple ``remote'' values and have a GenericRelation defined in their model
(rather than having another model pointed \emph{at} them). In the example
``article.publications'', the publications attribute is a
ReverseGenericRelatedObjectsDescriptor instance.

\end{fulllineitems}


\index{authorities (apps.profiles.models.Program attribute)}

\begin{fulllineitems}
\phantomsection\label{generated/apps.profiles.models:apps.profiles.models.Program.authorities}\pysigline{\bfcode{authorities}}{}
This class provides the functionality that makes the related-object
managers available as attributes on a model class, for fields that have
multiple ``remote'' values and have a GenericRelation defined in their model
(rather than having another model pointed \emph{at} them). In the example
``article.publications'', the publications attribute is a
ReverseGenericRelatedObjectsDescriptor instance.

\end{fulllineitems}


\index{unit\_permissions (apps.profiles.models.Program attribute)}

\begin{fulllineitems}
\phantomsection\label{generated/apps.profiles.models:apps.profiles.models.Program.unit_permissions}\pysigline{\bfcode{unit\_permissions}}{}
This class provides the functionality that makes the related-object
managers available as attributes on a model class, for fields that have
multiple ``remote'' values and have a GenericRelation defined in their model
(rather than having another model pointed \emph{at} them). In the example
``article.publications'', the publications attribute is a
ReverseGenericRelatedObjectsDescriptor instance.

\end{fulllineitems}


\end{fulllineitems}


\index{Project (class in apps.profiles.models)}

\begin{fulllineitems}
\phantomsection\label{generated/apps.profiles.models:apps.profiles.models.Project}\pysiglinewithargsret{\strong{class }\code{apps.profiles.models.}\bfcode{Project}}{\emph{*args}, \emph{**kwargs}}{}
Bases: {\hyperref[generated/apps.profiles.models:apps.profiles.models.BaseModel]{\code{apps.profiles.models.BaseModel}}}

Project is a now defunct way of expressing what we now use Organizations
and Works to accomplish.  This should be retired.

\index{unit\_permissions (apps.profiles.models.Project attribute)}

\begin{fulllineitems}
\phantomsection\label{generated/apps.profiles.models:apps.profiles.models.Project.unit_permissions}\pysigline{\bfcode{unit\_permissions}}{}
This class provides the functionality that makes the related-object
managers available as attributes on a model class, for fields that have
multiple ``remote'' values and have a GenericRelation defined in their model
(rather than having another model pointed \emph{at} them). In the example
``article.publications'', the publications attribute is a
ReverseGenericRelatedObjectsDescriptor instance.

\end{fulllineitems}


\end{fulllineitems}


\index{Requirement (class in apps.profiles.models)}

\begin{fulllineitems}
\phantomsection\label{generated/apps.profiles.models:apps.profiles.models.Requirement}\pysiglinewithargsret{\strong{class }\code{apps.profiles.models.}\bfcode{Requirement}}{\emph{*args}, \emph{**kwargs}}{}
Bases: {\hyperref[generated/apps.profiles.models:apps.profiles.models.BaseModel]{\code{apps.profiles.models.BaseModel}}}

A \code{Requirement} connects a \code{Program} to zero or more \code{Course} objects
in order to define a program's requirements.

Although this relationship exists in the model code, there are currently
no tools to manage this outside of the admin tool.  Considerable thought and
planning needs to go into how to manage this, as well as other models such
as \code{AreaOfStudy}, etc.

\index{unit\_permissions (apps.profiles.models.Requirement attribute)}

\begin{fulllineitems}
\phantomsection\label{generated/apps.profiles.models:apps.profiles.models.Requirement.unit_permissions}\pysigline{\bfcode{unit\_permissions}}{}
This class provides the functionality that makes the related-object
managers available as attributes on a model class, for fields that have
multiple ``remote'' values and have a GenericRelation defined in their model
(rather than having another model pointed \emph{at} them). In the example
``article.publications'', the publications attribute is a
ReverseGenericRelatedObjectsDescriptor instance.

\end{fulllineitems}


\end{fulllineitems}


\index{School (class in apps.profiles.models)}

\begin{fulllineitems}
\phantomsection\label{generated/apps.profiles.models:apps.profiles.models.School}\pysiglinewithargsret{\strong{class }\code{apps.profiles.models.}\bfcode{School}}{\emph{*args}, \emph{**kwargs}}{}
Bases: {\hyperref[generated/apps.profiles.models:apps.profiles.models.BaseModel]{\code{apps.profiles.models.BaseModel}}}

A \code{School} is one of the the highest ``units'' within the university.  
Below it are programs.  It sits at the same level as a \code{Department}

Historically, only Parsons the New School for Design has Schools.

\index{authorities (apps.profiles.models.School attribute)}

\begin{fulllineitems}
\phantomsection\label{generated/apps.profiles.models:apps.profiles.models.School.authorities}\pysigline{\bfcode{authorities}}{}
This class provides the functionality that makes the related-object
managers available as attributes on a model class, for fields that have
multiple ``remote'' values and have a GenericRelation defined in their model
(rather than having another model pointed \emph{at} them). In the example
``article.publications'', the publications attribute is a
ReverseGenericRelatedObjectsDescriptor instance.

\end{fulllineitems}


\index{unit\_permissions (apps.profiles.models.School attribute)}

\begin{fulllineitems}
\phantomsection\label{generated/apps.profiles.models:apps.profiles.models.School.unit_permissions}\pysigline{\bfcode{unit\_permissions}}{}
This class provides the functionality that makes the related-object
managers available as attributes on a model class, for fields that have
multiple ``remote'' values and have a GenericRelation defined in their model
(rather than having another model pointed \emph{at} them). In the example
``article.publications'', the publications attribute is a
ReverseGenericRelatedObjectsDescriptor instance.

\end{fulllineitems}


\end{fulllineitems}


\index{Section (class in apps.profiles.models)}

\begin{fulllineitems}
\phantomsection\label{generated/apps.profiles.models:apps.profiles.models.Section}\pysiglinewithargsret{\strong{class }\code{apps.profiles.models.}\bfcode{Section}}{\emph{*args}, \emph{**kwargs}}{}
Bases: {\hyperref[generated/apps.profiles.models:apps.profiles.models.BaseModel]{\code{apps.profiles.models.BaseModel}}}

A \code{Section} object is connected to a course and shows only the information
that is specific to a particular semester and set of faculty. A \code{Course} is 
an object that contains course information across time.

Refer to the \code{Course} documentation for a more complete explanation of this
distinction.

\index{get\_display\_title() (apps.profiles.models.Section method)}

\begin{fulllineitems}
\phantomsection\label{generated/apps.profiles.models:apps.profiles.models.Section.get_display_title}\pysiglinewithargsret{\bfcode{get\_display\_title}}{}{}
Not all sections need to have their own title, but there are enough cases
where the section of a course has a meaningfully different title that
this ought to be shown instead (e.g. Parsons AMT Collab studios)

\end{fulllineitems}


\index{unit\_permissions (apps.profiles.models.Section attribute)}

\begin{fulllineitems}
\phantomsection\label{generated/apps.profiles.models:apps.profiles.models.Section.unit_permissions}\pysigline{\bfcode{unit\_permissions}}{}
This class provides the functionality that makes the related-object
managers available as attributes on a model class, for fields that have
multiple ``remote'' values and have a GenericRelation defined in their model
(rather than having another model pointed \emph{at} them). In the example
``article.publications'', the publications attribute is a
ReverseGenericRelatedObjectsDescriptor instance.

\end{fulllineitems}


\end{fulllineitems}


\index{Semester (class in apps.profiles.models)}

\begin{fulllineitems}
\phantomsection\label{generated/apps.profiles.models:apps.profiles.models.Semester}\pysiglinewithargsret{\strong{class }\code{apps.profiles.models.}\bfcode{Semester}}{\emph{*args}, \emph{**kwargs}}{}
Bases: {\hyperref[generated/apps.profiles.models:apps.profiles.models.BaseModel]{\code{apps.profiles.models.BaseModel}}}

A \code{Semester} object represents the season and year in which a \code{Section}
of a \code{Course} is held.  Note that, in Banner exports, spring and summer
semesters are list as part of the previous year.  For example:
\begin{itemize}
\item {} 
201010 is the fall semester of 2010

\item {} 
201030 is the spring semester of 2011

\end{itemize}

For simplicity's sake, however, we convert the banner code to the correct
calendar year.  Therefore, fall 2010 has the term ``fa'' and the year ``2010''
while spring 2011 has the term ``sp'' and the year ``2011''.  All the conversions
for this are done in the import script and should never need to be
considered outside of that.

\index{unit\_permissions (apps.profiles.models.Semester attribute)}

\begin{fulllineitems}
\phantomsection\label{generated/apps.profiles.models:apps.profiles.models.Semester.unit_permissions}\pysigline{\bfcode{unit\_permissions}}{}
This class provides the functionality that makes the related-object
managers available as attributes on a model class, for fields that have
multiple ``remote'' values and have a GenericRelation defined in their model
(rather than having another model pointed \emph{at} them). In the example
``article.publications'', the publications attribute is a
ReverseGenericRelatedObjectsDescriptor instance.

\end{fulllineitems}


\end{fulllineitems}


\index{Sponsorship (class in apps.profiles.models)}

\begin{fulllineitems}
\phantomsection\label{generated/apps.profiles.models:apps.profiles.models.Sponsorship}\pysiglinewithargsret{\strong{class }\code{apps.profiles.models.}\bfcode{Sponsorship}}{\emph{*args}, \emph{**kwargs}}{}
Bases: {\hyperref[generated/apps.profiles.models:apps.profiles.models.BaseModel]{\code{apps.profiles.models.BaseModel}}}

A \code{Sponsorship} ties an organization to a bureaucratic unit, etc.

At the time of writing, this relationship exists only in the model code.
The necessity of an organization's sponsorship will need to be determined
as the \code{Organization} use cases and code achieve more maturity.

See the \code{Authority} class within the \code{reporting} documentation for
an analog to this between \code{Committee} objects and organizational units.

\index{content\_object (apps.profiles.models.Sponsorship attribute)}

\begin{fulllineitems}
\phantomsection\label{generated/apps.profiles.models:apps.profiles.models.Sponsorship.content_object}\pysigline{\bfcode{content\_object}}{}
Provides a generic relation to any object through content-type/object-id
fields.

\end{fulllineitems}


\index{unit\_permissions (apps.profiles.models.Sponsorship attribute)}

\begin{fulllineitems}
\phantomsection\label{generated/apps.profiles.models:apps.profiles.models.Sponsorship.unit_permissions}\pysigline{\bfcode{unit\_permissions}}{}
This class provides the functionality that makes the related-object
managers available as attributes on a model class, for fields that have
multiple ``remote'' values and have a GenericRelation defined in their model
(rather than having another model pointed \emph{at} them). In the example
``article.publications'', the publications attribute is a
ReverseGenericRelatedObjectsDescriptor instance.

\end{fulllineitems}


\end{fulllineitems}


\index{Staff (class in apps.profiles.models)}

\begin{fulllineitems}
\phantomsection\label{generated/apps.profiles.models:apps.profiles.models.Staff}\pysiglinewithargsret{\strong{class }\code{apps.profiles.models.}\bfcode{Staff}}{\emph{*args}, \emph{**kwargs}}{}
Bases: {\hyperref[generated/apps.profiles.models:apps.profiles.models.Person]{\code{apps.profiles.models.Person}}}

A \code{Staff} member is a \code{Person} that works for the university in a role
other than faculty.  They are assigned to a \code{Division} and have a 
job description.

\end{fulllineitems}


\index{Student (class in apps.profiles.models)}

\begin{fulllineitems}
\phantomsection\label{generated/apps.profiles.models:apps.profiles.models.Student}\pysiglinewithargsret{\strong{class }\code{apps.profiles.models.}\bfcode{Student}}{\emph{*args}, \emph{**kwargs}}{}
Bases: {\hyperref[generated/apps.profiles.models:apps.profiles.models.Person]{\code{apps.profiles.models.Person}}}

A \code{Student} is a \code{Person} that has a graduation year and home program.

Students need to be addressed with the utmost sensitivity and security, since
we must follow FERPA guidelines and cannot divulge any more data than is
absolutely necessary (i.e. directory information).  See:

\href{http://www2.ed.gov/policy/gen/guid/fpco/ferpa/mndirectoryinfo.html}{http://www2.ed.gov/policy/gen/guid/fpco/ferpa/mndirectoryinfo.html}

\end{fulllineitems}


\index{Subject (class in apps.profiles.models)}

\begin{fulllineitems}
\phantomsection\label{generated/apps.profiles.models:apps.profiles.models.Subject}\pysiglinewithargsret{\strong{class }\code{apps.profiles.models.}\bfcode{Subject}}{\emph{*args}, \emph{**kwargs}}{}
Bases: {\hyperref[generated/apps.profiles.models:apps.profiles.models.BaseModel]{\code{apps.profiles.models.BaseModel}}}

A \code{Subject} typically represents a group of \code{Course} objects below a
\code{Program}.  Recent changes in the way courses are constructed, though,
may require this simple relationship to be expanded somewhat.  A \code{Subject}
also contains the four-letter abbreviation that precedes a couse number
in the course directories.

\index{unit\_permissions (apps.profiles.models.Subject attribute)}

\begin{fulllineitems}
\phantomsection\label{generated/apps.profiles.models:apps.profiles.models.Subject.unit_permissions}\pysigline{\bfcode{unit\_permissions}}{}
This class provides the functionality that makes the related-object
managers available as attributes on a model class, for fields that have
multiple ``remote'' values and have a GenericRelation defined in their model
(rather than having another model pointed \emph{at} them). In the example
``article.publications'', the publications attribute is a
ReverseGenericRelatedObjectsDescriptor instance.

\end{fulllineitems}


\end{fulllineitems}


\index{UnitPermission (class in apps.profiles.models)}

\begin{fulllineitems}
\phantomsection\label{generated/apps.profiles.models:apps.profiles.models.UnitPermission}\pysiglinewithargsret{\strong{class }\code{apps.profiles.models.}\bfcode{UnitPermission}}{\emph{*args}, \emph{**kwargs}}{}
Bases: \code{django.db.models.base.Model}

A \code{UnitPermission} object attaches itself to a user and another DataMYNE
object, most suitably an organizational unit (e.g. \code{Division}, \code{Program}).

Unit permissions allow for an expanded authorization framework that goes
beyond the stock permissions built into Django which are based solely on
\code{ContentType}.  For example, a user may:
\begin{itemize}
\item {} 
have permission to change only \code{Course} objects

\item {} 
have permission to change only objects that falls under the Parsons \code{Division}

\end{itemize}

The effect of this is to create a matrix that constrains administrators to
only edit certain kind of objects and only object that fall under their
purview.

At the time of writing, this can be managed from within the admin tool by
DataMYNE administrators.  The administrator need to go to a unit's admin
page (e.g. Parsons) and, under the Unit Permissions section, add only those
users who are able to act on behalf of the unit (e.g. an operations manager
for Parsons, a program director of Communication Design, etc.)

While this addresses the problem of how to restrict editorial permissions
to both object (via Django's system) and unit, more work needs to be done
on creating an effective set of admin interfaces that could allow the 
assignment of unit permissions by people within the university, rather
than the rather rough way DataMYNE administrators need to work now
within the admin tool.  This will become especially important as the
system expands to cover the entire university.

\index{content\_object (apps.profiles.models.UnitPermission attribute)}

\begin{fulllineitems}
\phantomsection\label{generated/apps.profiles.models:apps.profiles.models.UnitPermission.content_object}\pysigline{\bfcode{content\_object}}{}
Provides a generic relation to any object through content-type/object-id
fields.

\end{fulllineitems}


\end{fulllineitems}


\index{Work (class in apps.profiles.models)}

\begin{fulllineitems}
\phantomsection\label{generated/apps.profiles.models:apps.profiles.models.Work}\pysiglinewithargsret{\strong{class }\code{apps.profiles.models.}\bfcode{Work}}{\emph{*args}, \emph{**kwargs}}{}
Bases: {\hyperref[generated/apps.profiles.models:apps.profiles.models.BaseModel]{\code{apps.profiles.models.BaseModel}}}

A \code{Work} represents any work of art or any other product/document/etc.
created within the system.  We do NOT associate these with a \code{Person}
directly.  Instead, these relationships are managed with the 
\code{Affiliation} class from the \code{reporting} application.  The allows for
many-to-many relationships, as well as providing subtle variations in the
role, date of participation, etc.

\index{affiliations (apps.profiles.models.Work attribute)}

\begin{fulllineitems}
\phantomsection\label{generated/apps.profiles.models:apps.profiles.models.Work.affiliations}\pysigline{\bfcode{affiliations}}{}
This class provides the functionality that makes the related-object
managers available as attributes on a model class, for fields that have
multiple ``remote'' values and have a GenericRelation defined in their model
(rather than having another model pointed \emph{at} them). In the example
``article.publications'', the publications attribute is a
ReverseGenericRelatedObjectsDescriptor instance.

\end{fulllineitems}


\index{unit\_permissions (apps.profiles.models.Work attribute)}

\begin{fulllineitems}
\phantomsection\label{generated/apps.profiles.models:apps.profiles.models.Work.unit_permissions}\pysigline{\bfcode{unit\_permissions}}{}
This class provides the functionality that makes the related-object
managers available as attributes on a model class, for fields that have
multiple ``remote'' values and have a GenericRelation defined in their model
(rather than having another model pointed \emph{at} them). In the example
``article.publications'', the publications attribute is a
ReverseGenericRelatedObjectsDescriptor instance.

\end{fulllineitems}


\end{fulllineitems}


\index{WorkURL (class in apps.profiles.models)}

\begin{fulllineitems}
\phantomsection\label{generated/apps.profiles.models:apps.profiles.models.WorkURL}\pysiglinewithargsret{\strong{class }\code{apps.profiles.models.}\bfcode{WorkURL}}{\emph{*args}, \emph{**kwargs}}{}
Bases: {\hyperref[generated/apps.profiles.models:apps.profiles.models.BaseModel]{\code{apps.profiles.models.BaseModel}}}

A \code{WorkURL} provides a URL associated with a persons work
(e.g. portfolio links, etc.).

This should be deprecated in favor of the more flexible
\code{Link} class.

\index{unit\_permissions (apps.profiles.models.WorkURL attribute)}

\begin{fulllineitems}
\phantomsection\label{generated/apps.profiles.models:apps.profiles.models.WorkURL.unit_permissions}\pysigline{\bfcode{unit\_permissions}}{}
This class provides the functionality that makes the related-object
managers available as attributes on a model class, for fields that have
multiple ``remote'' values and have a GenericRelation defined in their model
(rather than having another model pointed \emph{at} them). In the example
``article.publications'', the publications attribute is a
ReverseGenericRelatedObjectsDescriptor instance.

\end{fulllineitems}


\end{fulllineitems}



\chapter{Profiles: Views}
\label{generated/apps.profiles.views:module-apps.profiles.views}\label{generated/apps.profiles.views::doc}\label{generated/apps.profiles.views:profiles-views}
\index{apps.profiles.views (module)}
\index{accept\_invitation() (in module apps.profiles.views)}

\begin{fulllineitems}
\phantomsection\label{generated/apps.profiles.views:apps.profiles.views.accept_invitation}\pysiglinewithargsret{\code{apps.profiles.views.}\bfcode{accept\_invitation}}{\emph{request}, \emph{*args}, \emph{**kwargs}}{}
If the user if logged in, this view simply redirects that user to the
appropriate view and marks the invite as accepted.

This view requires that an object that can have an accepted invitation
possess an \code{accept\_invitiation} method.

\end{fulllineitems}


\index{accept\_organization\_invitation() (in module apps.profiles.views)}

\begin{fulllineitems}
\phantomsection\label{generated/apps.profiles.views:apps.profiles.views.accept_organization_invitation}\pysiglinewithargsret{\code{apps.profiles.views.}\bfcode{accept\_organization\_invitation}}{\emph{request}, \emph{*args}, \emph{**kwargs}}{}
This view marks an invitation to an organization as accepted.

Note that this should be deprecated in favor of the more general
\code{accept\_organization} view, but requires that the \code{Organization}
model have an \code{accept\_organization} model built into it.

\end{fulllineitems}


\index{activate() (in module apps.profiles.views)}

\begin{fulllineitems}
\phantomsection\label{generated/apps.profiles.views:apps.profiles.views.activate}\pysiglinewithargsret{\code{apps.profiles.views.}\bfcode{activate}}{\emph{request}, \emph{*args}, \emph{**kwargs}}{}
This is a deprecated method that preceded the move to LDAP authentication.

\end{fulllineitems}


\index{add\_syllabus() (in module apps.profiles.views)}

\begin{fulllineitems}
\phantomsection\label{generated/apps.profiles.views:apps.profiles.views.add_syllabus}\pysiglinewithargsret{\code{apps.profiles.views.}\bfcode{add\_syllabus}}{\emph{request}, \emph{*args}, \emph{**kwargs}}{}
This view allows specific users the ability to upload a syllabus
and attach it to the \code{Section} object.

\end{fulllineitems}


\index{admin() (in module apps.profiles.views)}

\begin{fulllineitems}
\phantomsection\label{generated/apps.profiles.views:apps.profiles.views.admin}\pysiglinewithargsret{\code{apps.profiles.views.}\bfcode{admin}}{\emph{request}, \emph{*args}, \emph{**kwargs}}{}
Deprecated.  A new and more complete admin section needs to be created to
handle the move from DataMYNE as a Parsons-based experiment to university-
wide infrastrucure.

\end{fulllineitems}


\index{api() (in module apps.profiles.views)}

\begin{fulllineitems}
\phantomsection\label{generated/apps.profiles.views:apps.profiles.views.api}\pysiglinewithargsret{\code{apps.profiles.views.}\bfcode{api}}{\emph{request}}{}
Deprecated

\end{fulllineitems}


\index{browse() (in module apps.profiles.views)}

\begin{fulllineitems}
\phantomsection\label{generated/apps.profiles.views:apps.profiles.views.browse}\pysiglinewithargsret{\code{apps.profiles.views.}\bfcode{browse}}{\emph{request}}{}
This is a deprecated view.  It served primarily as the \emph{browse} page, which has
since been temporarily removed from the main nav pending a redesign.

\end{fulllineitems}


\index{contact() (in module apps.profiles.views)}

\begin{fulllineitems}
\phantomsection\label{generated/apps.profiles.views:apps.profiles.views.contact}\pysiglinewithargsret{\code{apps.profiles.views.}\bfcode{contact}}{\emph{request}}{}
A simple contact-form processor.

\end{fulllineitems}


\index{contact\_student() (in module apps.profiles.views)}

\begin{fulllineitems}
\phantomsection\label{generated/apps.profiles.views:apps.profiles.views.contact_student}\pysiglinewithargsret{\code{apps.profiles.views.}\bfcode{contact\_student}}{\emph{request}, \emph{person\_id}}{}
This view allows users to send students email without revealing the
student's email address publicly.

There most likely should be a setting on the \code{Student} profile page
which allows the student to disable this.  At time of writing, this
does not exist in either model or view code and should be considered.

\end{fulllineitems}


\index{decline\_invitation() (in module apps.profiles.views)}

\begin{fulllineitems}
\phantomsection\label{generated/apps.profiles.views:apps.profiles.views.decline_invitation}\pysiglinewithargsret{\code{apps.profiles.views.}\bfcode{decline\_invitation}}{\emph{request}, \emph{*args}, \emph{**kwargs}}{}
If the user if logged in, this view simply redirects that user back to his or
her previous view and deletes the invitation.  The \code{Invitation} is one of the
very few objects that get deleted in the DataMYNE system.  In the future, it
may prove useful to only ``logically'' delete these, but still retain the connection
in order to mine more data on social dynamics.

\end{fulllineitems}


\index{delete\_work() (in module apps.profiles.views)}

\begin{fulllineitems}
\phantomsection\label{generated/apps.profiles.views:apps.profiles.views.delete_work}\pysiglinewithargsret{\code{apps.profiles.views.}\bfcode{delete\_work}}{\emph{request}, \emph{work\_id}, \emph{person\_id}}{}
This view deletes a \code{Work} object.  It is one of the few models
that we allow to be publicly deleted.  Since this is user-contributed content,
however, it only seems fair that we respect their wishes about the display
and use of their own work.

\end{fulllineitems}


\index{download\_syllabus() (in module apps.profiles.views)}

\begin{fulllineitems}
\phantomsection\label{generated/apps.profiles.views:apps.profiles.views.download_syllabus}\pysiglinewithargsret{\code{apps.profiles.views.}\bfcode{download\_syllabus}}{\emph{request}, \emph{section\_id}}{}
This view generates a document based on the uploaded syllabus.
It tries to create a name for the file that matches the university's
guidelines for naming syllabus documents rather than defaulting to
the user's file naming scheme.

\end{fulllineitems}


\index{edit\_course\_profile() (in module apps.profiles.views)}

\begin{fulllineitems}
\phantomsection\label{generated/apps.profiles.views:apps.profiles.views.edit_course_profile}\pysiglinewithargsret{\code{apps.profiles.views.}\bfcode{edit\_course\_profile}}{\emph{request}, \emph{*args}, \emph{**kwargs}}{}
This view allows for the editing of courses on the public site.
Like the \code{Person} profile views, it has both admin and non-admin
versions, so be careful with that distinction.  It also uses the 
unit permissions and Django permissions code.  While simpler here than
in other parts of the site (cf. \code{edit\_organization}), the ability to 
edit is probably best determined with a object method written into the 
model and not in the view code, as this makes the permissions system
overly brittle.

\end{fulllineitems}


\index{edit\_organization() (in module apps.profiles.views)}

\begin{fulllineitems}
\phantomsection\label{generated/apps.profiles.views:apps.profiles.views.edit_organization}\pysiglinewithargsret{\code{apps.profiles.views.}\bfcode{edit\_organization}}{\emph{request}, \emph{*args}, \emph{**kwargs}}{}
This view edits an \code{Organization} object.  See the model documentation
for a more complete description of what an organization represents.

Along with the \code{Committee} model, organizations represent one of the most
complicated objects in terms of security.  This model should be refactored first
as part of a general clean up to remove edittable from the view code and port
it into the model code itself.

Also note the use of the \code{current} and \code{past} managers for the member
and leader affiliations.  This is one of the benefits of using the 
\code{Affiliation} object over a \code{ManyToManyField}: we can retain historical
information even after the connection is no longer active.  For example, 
we can know all of the previous leaders of an organization while still allowing 
the current ones to be the only recipients of security clearance, public display,
etc.

In addition, the \code{Affiliation} managers have the \code{begin} and \code{retire} methods
that allow connections to see easily set to current or past without having to 
rewrite complicated code.  ``Retirement'' is the preferred way for disposing of a current
affiliation.  It has the same effect as a deletion, while still retaining the 
connections for historical and data-mining purposes.

\end{fulllineitems}


\index{edit\_person\_profile() (in module apps.profiles.views)}

\begin{fulllineitems}
\phantomsection\label{generated/apps.profiles.views:apps.profiles.views.edit_person_profile}\pysiglinewithargsret{\code{apps.profiles.views.}\bfcode{edit\_person\_profile}}{\emph{request}, \emph{person\_id}}{}
This view serves to unify the \code{FacultyMember}, \code{Student}, and \code{Staff}
profile edit views under a single edit view.  Within the \code{urls} module, 
any person's role-specific profile editing page can be reached from here.  
This allows a common ``gateway'' to profiles editors without needing to decide, 
within the templates, which role a \code{Person} occupies.

This should be the destination for all profile edit requests going forward.

\end{fulllineitems}


\index{edit\_profile() (in module apps.profiles.views)}

\begin{fulllineitems}
\phantomsection\label{generated/apps.profiles.views:apps.profiles.views.edit_profile}\pysiglinewithargsret{\code{apps.profiles.views.}\bfcode{edit\_profile}}{\emph{request}, \emph{*args}, \emph{**kwargs}}{}
This is the profile edit view for \code{FacultyMember} objects.  Its function is pretty
straightforward, but there are some tricky areas:
\begin{itemize}
\item {} 
much of this code is similar to other \code{Person} objects.  This could be
refactored into the \code{edit\_person\_profile} view to reduce repetition.

\item {} 
edittable defines whether a user can edit this object.  This really needs
to be refactored out of the view code and into the model code, either
at the level of \code{FacultyMember} or, at least in part, at the level of
\code{Person} itself.  The reason for this is that the combination of 
Django permissions and unit permissions is becoming increasingly brittle.
Unifying this with a well-written object method would improve security,
reusability, and performance.

\item {} 
There are really two edit views: one for regular users and one for admins.
Be sensitive to the distinction, since admins have access to many fields
that we do not want users to be able to update (e.g. field provided by
Banner, etc.)

\end{itemize}

\end{fulllineitems}


\index{edit\_section\_profile() (in module apps.profiles.views)}

\begin{fulllineitems}
\phantomsection\label{generated/apps.profiles.views:apps.profiles.views.edit_section_profile}\pysiglinewithargsret{\code{apps.profiles.views.}\bfcode{edit\_section\_profile}}{\emph{request}, \emph{*args}, \emph{**kwargs}}{}
This view allows for the editing of sections on the public site.
Like the \code{Person} profile views, it has both admin and non-admin
versions, so be careful with that distinction.  It also uses the 
unit permissions and Django permissions code.  While simpler here than
in other parts of the site (cf. \code{edit\_organization}), the ability to 
edit is probably best determined with a object method written into the 
model and not in the view code, as this makes the permissions system
overly brittle.

One important consideration for the editting permissions is that we
allow assigned faculty to edit this page, as well as the usual 
administrators.

\end{fulllineitems}


\index{edit\_staff\_profile() (in module apps.profiles.views)}

\begin{fulllineitems}
\phantomsection\label{generated/apps.profiles.views:apps.profiles.views.edit_staff_profile}\pysiglinewithargsret{\code{apps.profiles.views.}\bfcode{edit\_staff\_profile}}{\emph{request}, \emph{*args}, \emph{**kwargs}}{}
This is the profile edit view for \code{Staff} objects.  Its function is pretty
straightforward, but there are some tricky areas:
\begin{itemize}
\item {} 
much of this code is similar to other \code{Person} objects.  This could be
refactored into the \code{edit\_person\_profile} view to reduce repetition.

\item {} 
edittable defines whether a user can edit this object.  This really needs
to be refactored out of the view code and into the model code, either
at the level of \code{Staff} or, at least in part, at the level of
\code{Person} itself.  The reason for this is that the combination of 
Django permissions and unit permissions is becoming increasingly brittle.
Unifying this with a well-written object method would improve security,
reusability, and performance.

\item {} 
There are really two edit views: one for regular users and one for admins.
Be sensitive to the distinction, since admins have access to many fields
that we do not want users to be able to update (e.g. field provided by
Banner, etc.)

\end{itemize}

\end{fulllineitems}


\index{edit\_student\_profile() (in module apps.profiles.views)}

\begin{fulllineitems}
\phantomsection\label{generated/apps.profiles.views:apps.profiles.views.edit_student_profile}\pysiglinewithargsret{\code{apps.profiles.views.}\bfcode{edit\_student\_profile}}{\emph{request}, \emph{*args}, \emph{**kwargs}}{}
This is the profile edit view for \code{Student} objects.  Its function is pretty
straightforward, but there are some tricky areas:
\begin{itemize}
\item {} 
much of this code is similar to other \code{Person} objects.  This could be
refactored into the \code{edit\_person\_profile} view to reduce repetition.

\item {} 
edittable defines whether a user can edit this object.  This really needs
to be refactored out of the view code and into the model code, either
at the level of \code{Student} or, at least in part, at the level of
\code{Person} itself.  The reason for this is that the combination of 
Django permissions and unit permissions is becoming increasingly brittle.
Unifying this with a well-written object method would improve security,
reusability, and performance.

\item {} 
There are really two edit views: one for regular users and one for admins.
Be sensitive to the distinction, since admins have access to many fields
that we do not want users to be able to update (e.g. field provided by
Banner, etc.)

\end{itemize}

\end{fulllineitems}


\index{edit\_work() (in module apps.profiles.views)}

\begin{fulllineitems}
\phantomsection\label{generated/apps.profiles.views:apps.profiles.views.edit_work}\pysiglinewithargsret{\code{apps.profiles.views.}\bfcode{edit\_work}}{\emph{request}, \emph{*args}, \emph{**kwargs}}{}
This view edits a \code{Work} object and its affiliated creators.

As with other model views, this should be refactored to move the
security code into the model itself rather than re-writing the 
``edittable'' flag within the view code.

\end{fulllineitems}


\index{filter() (in module apps.profiles.views)}

\begin{fulllineitems}
\phantomsection\label{generated/apps.profiles.views:apps.profiles.views.filter}\pysiglinewithargsret{\code{apps.profiles.views.}\bfcode{filter}}{\emph{request}}{}
This view has been deprecated since the introduction of new \code{Student}
and \code{Staff} models.  A more comprehensive browse/filter view is in
the planning stages at the time of writing.

\end{fulllineitems}


\index{home() (in module apps.profiles.views)}

\begin{fulllineitems}
\phantomsection\label{generated/apps.profiles.views:apps.profiles.views.home}\pysiglinewithargsret{\code{apps.profiles.views.}\bfcode{home}}{\emph{request}}{}
This is the home page of DataMYNE.  It calls the \code{\_random\_images}
function to generate a set of images that contain a mix of user
profile images and work images.

\end{fulllineitems}


\index{list\_profiles() (in module apps.profiles.views)}

\begin{fulllineitems}
\phantomsection\label{generated/apps.profiles.views:apps.profiles.views.list_profiles}\pysiglinewithargsret{\code{apps.profiles.views.}\bfcode{list\_profiles}}{\emph{request}, \emph{tag='`}}{}
This is a deprecated view.  It served primarily as destination for tag links, but has
since been temporarily removed in favor of a direct link to the search results for a tag.

\end{fulllineitems}


\index{stats\_report() (in module apps.profiles.views)}

\begin{fulllineitems}
\phantomsection\label{generated/apps.profiles.views:apps.profiles.views.stats_report}\pysiglinewithargsret{\code{apps.profiles.views.}\bfcode{stats\_report}}{\emph{request}, \emph{*args}, \emph{**kwargs}}{}
Deprecated.  A new admin (with stats) needs to be created.

\end{fulllineitems}


\index{view\_course() (in module apps.profiles.views)}

\begin{fulllineitems}
\phantomsection\label{generated/apps.profiles.views:apps.profiles.views.view_course}\pysiglinewithargsret{\code{apps.profiles.views.}\bfcode{view\_course}}{\emph{request}, \emph{course\_id}}{}
This view displays the course.  It's surprisingly simple, given how much work has
been done with courses over time.  In fact, a lot of the work of displaying the
page appears on the template itself, for good or ill.

Like other simple model views, this view could benefit from being refactored so
that the security logic exists in the Course model itself and not in the course
view.

\end{fulllineitems}


\index{view\_invitation() (in module apps.profiles.views)}

\begin{fulllineitems}
\phantomsection\label{generated/apps.profiles.views:apps.profiles.views.view_invitation}\pysiglinewithargsret{\code{apps.profiles.views.}\bfcode{view\_invitation}}{\emph{request}, \emph{*args}, \emph{**kwargs}}{}
If the user if logged in, this view simply redirects that user to the
appropriate view and marks the invite as received.

\end{fulllineitems}


\index{view\_organization() (in module apps.profiles.views)}

\begin{fulllineitems}
\phantomsection\label{generated/apps.profiles.views:apps.profiles.views.view_organization}\pysiglinewithargsret{\code{apps.profiles.views.}\bfcode{view\_organization}}{\emph{request}, \emph{organization\_id}}{}
This view displays an \code{Organization} object.  See the model documentation
for a more complete description of what an organization represents.

Along with the \code{Committee} model, organizations represent one of the most
complicated objects in terms of security.  This model should be refactored first
as part of a general clean up to remove edittable from the view code and port
it into the model code itself.

It would also be a good idea to add an ``invite'' permission that could allow
members with fewer security privileges to still invite new members.

Also note the use of the \code{current} and \code{past} managers for the member
and leader affiliations.  This is one of the benefits of using the 
\code{Affiliation} object over a \code{ManyToManyField}: we can retain historical
information even after the connection is no longer active.  For example, 
we can know all of the previous leaders of an organization while still allowing 
the current ones to be the only recipients of security clearance, public display,
etc.

\end{fulllineitems}


\index{view\_person\_profile() (in module apps.profiles.views)}

\begin{fulllineitems}
\phantomsection\label{generated/apps.profiles.views:apps.profiles.views.view_person_profile}\pysiglinewithargsret{\code{apps.profiles.views.}\bfcode{view\_person\_profile}}{\emph{request}, \emph{person\_id}}{}
This view serves to unify the \code{FacultyMember}, \code{Student}, and \code{Staff}
profile views under a single view.  Within the \code{urls} module, any person's
role-specific profile can be reached from here.  This allows a common
``gateway'' to profiles without needing to decide, within the templates,
which role a \code{Person} occupies.

This should be the destination for all profile requests going forward.

\end{fulllineitems}


\index{view\_profile() (in module apps.profiles.views)}

\begin{fulllineitems}
\phantomsection\label{generated/apps.profiles.views:apps.profiles.views.view_profile}\pysiglinewithargsret{\code{apps.profiles.views.}\bfcode{view\_profile}}{\emph{request}, \emph{person\_id}}{}
This is the profile view for \code{FacultyMembers}.  Its function is pretty
straightforward, but there are some tricky areas:
\begin{itemize}
\item {} 
much of this code is similar to other \code{Person} objects.  This could be
refactored into the \code{view\_person\_profile} view to reduce repetition.

\item {} 
edittable defines whether the Edit button appears.  This really needs
to be refactored out of the view code and into the model code, either
at the level of \code{FacultyMember} or, at least in part, at the level of
\code{Person} itself.  The reason for this is that the combination of 
Django permissions and unit permissions is becoming increasingly brittle.
Unifying this with a well-written object method would improve security,
reusability, and performance.

\item {} 
mlt refers to the \code{more\_like\_this} method available in \code{haystack} 
(and, in turn, \code{Solr}).  It produces what appears to be a fairly interesting
list of related documents (i.e. other DataMYNE objects with sufficient
document similarity).  However, this could be improved with a faceted search.
For example, perhaps only other people and/or works are shown, but not courses
or committees.  Further user testing is necessary to determine this.

\item {} 
OpenCalais code exists here which updates the OpenCalais objects in
the background.  The OC code is still experimental within DataMYNE and
should probably either be removed or made subject to a debug setting.

\end{itemize}

\end{fulllineitems}


\index{view\_program() (in module apps.profiles.views)}

\begin{fulllineitems}
\phantomsection\label{generated/apps.profiles.views:apps.profiles.views.view_program}\pysiglinewithargsret{\code{apps.profiles.views.}\bfcode{view\_program}}{\emph{request}, \emph{program\_id}}{}
This view is more or less a stub of pag for displaying the \code{Program}
organizational unit.  Ultimately, this should be a place where admins and other
authorized users can go to work with program-related functions, like managing
\code{Committee} and \code{Authority} relationships, etc.  Other org units should 
probably get similar pages.

\end{fulllineitems}


\index{view\_project\_profile() (in module apps.profiles.views)}

\begin{fulllineitems}
\phantomsection\label{generated/apps.profiles.views:apps.profiles.views.view_project_profile}\pysiglinewithargsret{\code{apps.profiles.views.}\bfcode{view\_project\_profile}}{\emph{request}, \emph{project\_id}}{}
This view, and its associated \code{Project} model, is deprecated and should
be refactored out.

\end{fulllineitems}


\index{view\_section() (in module apps.profiles.views)}

\begin{fulllineitems}
\phantomsection\label{generated/apps.profiles.views:apps.profiles.views.view_section}\pysiglinewithargsret{\code{apps.profiles.views.}\bfcode{view\_section}}{\emph{request}, \emph{section\_id}}{}
This view displays a \code{Section} of a \code{Course} (see the model
documentation for a complete description of the distinction between
the two.)

This view is notable as being one of the places in which the object-based
permissions are used.  Object-based permissions are applied at the level of
a specific object -- in this case, whether the user is allowed to read the
syllabus of a specific instructor.

As with other view, the edittable flag here would best be refactored into the
model code itself rather than remain in the view.

\end{fulllineitems}


\index{view\_staff\_profile() (in module apps.profiles.views)}

\begin{fulllineitems}
\phantomsection\label{generated/apps.profiles.views:apps.profiles.views.view_staff_profile}\pysiglinewithargsret{\code{apps.profiles.views.}\bfcode{view\_staff\_profile}}{\emph{request}, \emph{person\_id}}{}
This is the profile view for \code{Staff} objects.  Its function is pretty
straightforward, but there are some tricky areas:
\begin{itemize}
\item {} 
much of this code is similar to other \code{Person} objects.  This could be
refactored into the \code{view\_person\_profile} view to reduce repetition.

\item {} 
edittable defines whether the Edit button appears.  This really needs
to be refactored out of the view code and into the model code, either
at the level of \code{Staff} or, at least in part, at the level of
\code{Person} itself.  The reason for this is that the combination of 
Django permissions and unit permissions is becoming increasingly brittle.
Unifying this with a well-written object method would improve security,
reusability, and performance.

\item {} 
mlt refers to the \code{more\_like\_this} method available in \code{haystack} 
(and, in turn, \code{Solr}).  It produces what appears to be a fairly interesting
list of related documents (i.e. other DataMYNE objects with sufficient
document similarity).  However, this could be improved with a faceted search.
For example, perhaps only other people and/or works are shown, but not courses
or committees.  Further user testing is necessary to determine this.

\item {} 
OpenCalais code exists here which updates the OpenCalais objects in
the background.  The OC code is still experimental within DataMYNE and
should probably either be removed or made subject to a debug setting.

\end{itemize}

\end{fulllineitems}


\index{view\_student\_profile() (in module apps.profiles.views)}

\begin{fulllineitems}
\phantomsection\label{generated/apps.profiles.views:apps.profiles.views.view_student_profile}\pysiglinewithargsret{\code{apps.profiles.views.}\bfcode{view\_student\_profile}}{\emph{request}, \emph{person\_id}}{}
This is the profile view for \code{Student} objects.  Its function is pretty
straightforward, but there are some tricky areas:
\begin{itemize}
\item {} 
much of this code is similar to other \code{Person} objects.  This could be
refactored into the \code{view\_person\_profile} view to reduce repetition.

\item {} 
edittable defines whether the Edit button appears.  This really needs
to be refactored out of the view code and into the model code, either
at the level of \code{Student} or, at least in part, at the level of
\code{Person} itself.  The reason for this is that the combination of 
Django permissions and unit permissions is becoming increasingly brittle.
Unifying this with a well-written object method would improve security,
reusability, and performance.

\item {} 
mlt refers to the \code{more\_like\_this} method available in \code{haystack} 
(and, in turn, \code{Solr}).  It produces what appears to be a fairly interesting
list of related documents (i.e. other DataMYNE objects with sufficient
document similarity).  However, this could be improved with a faceted search.
For example, perhaps only other people and/or works are shown, but not courses
or committees.  Further user testing is necessary to determine this.

\item {} 
OpenCalais code exists here which updates the OpenCalais objects in
the background.  The OC code is still experimental within DataMYNE and
should probably either be removed or made subject to a debug setting.

\end{itemize}

\end{fulllineitems}


\index{view\_work() (in module apps.profiles.views)}

\begin{fulllineitems}
\phantomsection\label{generated/apps.profiles.views:apps.profiles.views.view_work}\pysiglinewithargsret{\code{apps.profiles.views.}\bfcode{view\_work}}{\emph{request}, \emph{work\_id}}{}
This view displays a \code{Work} object and its affiliated creators.

\end{fulllineitems}


\index{wordpress() (in module apps.profiles.views)}

\begin{fulllineitems}
\phantomsection\label{generated/apps.profiles.views:apps.profiles.views.wordpress}\pysiglinewithargsret{\code{apps.profiles.views.}\bfcode{wordpress}}{\emph{request}, \emph{faculty\_id}}{}
\end{fulllineitems}



\chapter{Profiles: Fields}
\label{generated/apps.profiles.fields:profiles-fields}\label{generated/apps.profiles.fields::doc}\label{generated/apps.profiles.fields:module-apps.profiles.fields}
\index{apps.profiles.fields (module)}
Created on Apr 27, 2011

@author: Mike\_Edwards

\index{DataMyneSplitDateTimeField (class in apps.profiles.fields)}

\begin{fulllineitems}
\phantomsection\label{generated/apps.profiles.fields:apps.profiles.fields.DataMyneSplitDateTimeField}\pysiglinewithargsret{\strong{class }\code{apps.profiles.fields.}\bfcode{DataMyneSplitDateTimeField}}{\emph{*args}, \emph{**kwargs}}{}
Bases: \code{django.forms.fields.MultiValueField}

based on: \href{http://copiesofcopies.org/webl/2010/04/26/a-better-datetime-widget-for-django/}{http://copiesofcopies.org/webl/2010/04/26/a-better-datetime-widget-for-django/}

This field allows for split date/time form entries with ajax-powered widgets

\index{compress() (apps.profiles.fields.DataMyneSplitDateTimeField method)}

\begin{fulllineitems}
\phantomsection\label{generated/apps.profiles.fields:apps.profiles.fields.DataMyneSplitDateTimeField.compress}\pysiglinewithargsret{\bfcode{compress}}{\emph{data\_list}}{}
Takes the values from the MultiWidget and passes them as a
list to this function. This function needs to compress the
list into a single object to save.

\end{fulllineitems}


\index{widget (apps.profiles.fields.DataMyneSplitDateTimeField attribute)}

\begin{fulllineitems}
\phantomsection\label{generated/apps.profiles.fields:apps.profiles.fields.DataMyneSplitDateTimeField.widget}\pysigline{\bfcode{widget}}{}
alias of \code{DataMyneSplitDateTimeWidget}

\end{fulllineitems}


\end{fulllineitems}



\chapter{Profiles: Forms}
\label{generated/apps.profiles.forms:profiles-forms}\label{generated/apps.profiles.forms::doc}\label{generated/apps.profiles.forms:module-apps.profiles.forms}
\index{apps.profiles.forms (module)}
\index{AdminCourseForm (class in apps.profiles.forms)}

\begin{fulllineitems}
\phantomsection\label{generated/apps.profiles.forms:apps.profiles.forms.AdminCourseForm}\pysiglinewithargsret{\strong{class }\code{apps.profiles.forms.}\bfcode{AdminCourseForm}}{\emph{data=None}, \emph{files=None}, \emph{auto\_id='id\_\%s'}, \emph{prefix=None}, \emph{initial=None}, \emph{error\_class=\textless{}class `django.forms.util.ErrorList'\textgreater{}}, \emph{label\_suffix=':'}, \emph{empty\_permitted=False}, \emph{instance=None}}{}
Bases: \code{django.forms.models.ModelForm}

\index{AdminCourseForm.Meta (class in apps.profiles.forms)}

\begin{fulllineitems}
\phantomsection\label{generated/apps.profiles.forms:apps.profiles.forms.AdminCourseForm.Meta}\pysigline{\strong{class }\bfcode{Meta}}{}~
\index{model (apps.profiles.forms.AdminCourseForm.Meta attribute)}

\begin{fulllineitems}
\phantomsection\label{generated/apps.profiles.forms:apps.profiles.forms.AdminCourseForm.Meta.model}\pysigline{\bfcode{model}}{}
alias of \code{Course}

\end{fulllineitems}


\end{fulllineitems}


\index{media (apps.profiles.forms.AdminCourseForm attribute)}

\begin{fulllineitems}
\phantomsection\label{generated/apps.profiles.forms:apps.profiles.forms.AdminCourseForm.media}\pysigline{\code{AdminCourseForm.}\bfcode{media}}{}
\end{fulllineitems}


\end{fulllineitems}


\index{AdminFacultyForm (class in apps.profiles.forms)}

\begin{fulllineitems}
\phantomsection\label{generated/apps.profiles.forms:apps.profiles.forms.AdminFacultyForm}\pysiglinewithargsret{\strong{class }\code{apps.profiles.forms.}\bfcode{AdminFacultyForm}}{\emph{data=None}, \emph{files=None}, \emph{auto\_id='id\_\%s'}, \emph{prefix=None}, \emph{initial=None}, \emph{error\_class=\textless{}class `django.forms.util.ErrorList'\textgreater{}}, \emph{label\_suffix=':'}, \emph{empty\_permitted=False}, \emph{instance=None}}{}
Bases: \code{django.forms.models.ModelForm}

\index{AdminFacultyForm.Meta (class in apps.profiles.forms)}

\begin{fulllineitems}
\phantomsection\label{generated/apps.profiles.forms:apps.profiles.forms.AdminFacultyForm.Meta}\pysigline{\strong{class }\bfcode{Meta}}{}~
\index{model (apps.profiles.forms.AdminFacultyForm.Meta attribute)}

\begin{fulllineitems}
\phantomsection\label{generated/apps.profiles.forms:apps.profiles.forms.AdminFacultyForm.Meta.model}\pysigline{\bfcode{model}}{}
alias of \code{FacultyMember}

\end{fulllineitems}


\end{fulllineitems}


\index{clean\_bio() (apps.profiles.forms.AdminFacultyForm method)}

\begin{fulllineitems}
\phantomsection\label{generated/apps.profiles.forms:apps.profiles.forms.AdminFacultyForm.clean_bio}\pysiglinewithargsret{\code{AdminFacultyForm.}\bfcode{clean\_bio}}{}{}
\end{fulllineitems}


\index{clean\_expertise() (apps.profiles.forms.AdminFacultyForm method)}

\begin{fulllineitems}
\phantomsection\label{generated/apps.profiles.forms:apps.profiles.forms.AdminFacultyForm.clean_expertise}\pysiglinewithargsret{\code{AdminFacultyForm.}\bfcode{clean\_expertise}}{}{}
\end{fulllineitems}


\index{media (apps.profiles.forms.AdminFacultyForm attribute)}

\begin{fulllineitems}
\phantomsection\label{generated/apps.profiles.forms:apps.profiles.forms.AdminFacultyForm.media}\pysigline{\code{AdminFacultyForm.}\bfcode{media}}{}
\end{fulllineitems}


\end{fulllineitems}


\index{AdminSectionForm (class in apps.profiles.forms)}

\begin{fulllineitems}
\phantomsection\label{generated/apps.profiles.forms:apps.profiles.forms.AdminSectionForm}\pysiglinewithargsret{\strong{class }\code{apps.profiles.forms.}\bfcode{AdminSectionForm}}{\emph{data=None}, \emph{files=None}, \emph{auto\_id='id\_\%s'}, \emph{prefix=None}, \emph{initial=None}, \emph{error\_class=\textless{}class `django.forms.util.ErrorList'\textgreater{}}, \emph{label\_suffix=':'}, \emph{empty\_permitted=False}, \emph{instance=None}}{}
Bases: \code{django.forms.models.ModelForm}

\index{AdminSectionForm.Meta (class in apps.profiles.forms)}

\begin{fulllineitems}
\phantomsection\label{generated/apps.profiles.forms:apps.profiles.forms.AdminSectionForm.Meta}\pysigline{\strong{class }\bfcode{Meta}}{}~
\index{model (apps.profiles.forms.AdminSectionForm.Meta attribute)}

\begin{fulllineitems}
\phantomsection\label{generated/apps.profiles.forms:apps.profiles.forms.AdminSectionForm.Meta.model}\pysigline{\bfcode{model}}{}
alias of \code{Section}

\end{fulllineitems}


\end{fulllineitems}


\index{media (apps.profiles.forms.AdminSectionForm attribute)}

\begin{fulllineitems}
\phantomsection\label{generated/apps.profiles.forms:apps.profiles.forms.AdminSectionForm.media}\pysigline{\code{AdminSectionForm.}\bfcode{media}}{}
\end{fulllineitems}


\end{fulllineitems}


\index{AdminStaffForm (class in apps.profiles.forms)}

\begin{fulllineitems}
\phantomsection\label{generated/apps.profiles.forms:apps.profiles.forms.AdminStaffForm}\pysiglinewithargsret{\strong{class }\code{apps.profiles.forms.}\bfcode{AdminStaffForm}}{\emph{data=None}, \emph{files=None}, \emph{auto\_id='id\_\%s'}, \emph{prefix=None}, \emph{initial=None}, \emph{error\_class=\textless{}class `django.forms.util.ErrorList'\textgreater{}}, \emph{label\_suffix=':'}, \emph{empty\_permitted=False}, \emph{instance=None}}{}
Bases: \code{django.forms.models.ModelForm}

\index{AdminStaffForm.Meta (class in apps.profiles.forms)}

\begin{fulllineitems}
\phantomsection\label{generated/apps.profiles.forms:apps.profiles.forms.AdminStaffForm.Meta}\pysigline{\strong{class }\bfcode{Meta}}{}~
\index{model (apps.profiles.forms.AdminStaffForm.Meta attribute)}

\begin{fulllineitems}
\phantomsection\label{generated/apps.profiles.forms:apps.profiles.forms.AdminStaffForm.Meta.model}\pysigline{\bfcode{model}}{}
alias of \code{Staff}

\end{fulllineitems}


\end{fulllineitems}


\index{clean\_bio() (apps.profiles.forms.AdminStaffForm method)}

\begin{fulllineitems}
\phantomsection\label{generated/apps.profiles.forms:apps.profiles.forms.AdminStaffForm.clean_bio}\pysiglinewithargsret{\code{AdminStaffForm.}\bfcode{clean\_bio}}{}{}
\end{fulllineitems}


\index{clean\_expertise() (apps.profiles.forms.AdminStaffForm method)}

\begin{fulllineitems}
\phantomsection\label{generated/apps.profiles.forms:apps.profiles.forms.AdminStaffForm.clean_expertise}\pysiglinewithargsret{\code{AdminStaffForm.}\bfcode{clean\_expertise}}{}{}
\end{fulllineitems}


\index{media (apps.profiles.forms.AdminStaffForm attribute)}

\begin{fulllineitems}
\phantomsection\label{generated/apps.profiles.forms:apps.profiles.forms.AdminStaffForm.media}\pysigline{\code{AdminStaffForm.}\bfcode{media}}{}
\end{fulllineitems}


\end{fulllineitems}


\index{AdminStudentForm (class in apps.profiles.forms)}

\begin{fulllineitems}
\phantomsection\label{generated/apps.profiles.forms:apps.profiles.forms.AdminStudentForm}\pysiglinewithargsret{\strong{class }\code{apps.profiles.forms.}\bfcode{AdminStudentForm}}{\emph{data=None}, \emph{files=None}, \emph{auto\_id='id\_\%s'}, \emph{prefix=None}, \emph{initial=None}, \emph{error\_class=\textless{}class `django.forms.util.ErrorList'\textgreater{}}, \emph{label\_suffix=':'}, \emph{empty\_permitted=False}, \emph{instance=None}}{}
Bases: \code{django.forms.models.ModelForm}

\index{AdminStudentForm.Meta (class in apps.profiles.forms)}

\begin{fulllineitems}
\phantomsection\label{generated/apps.profiles.forms:apps.profiles.forms.AdminStudentForm.Meta}\pysigline{\strong{class }\bfcode{Meta}}{}~
\index{model (apps.profiles.forms.AdminStudentForm.Meta attribute)}

\begin{fulllineitems}
\phantomsection\label{generated/apps.profiles.forms:apps.profiles.forms.AdminStudentForm.Meta.model}\pysigline{\bfcode{model}}{}
alias of \code{Student}

\end{fulllineitems}


\end{fulllineitems}


\index{clean\_bio() (apps.profiles.forms.AdminStudentForm method)}

\begin{fulllineitems}
\phantomsection\label{generated/apps.profiles.forms:apps.profiles.forms.AdminStudentForm.clean_bio}\pysiglinewithargsret{\code{AdminStudentForm.}\bfcode{clean\_bio}}{}{}
\end{fulllineitems}


\index{clean\_expertise() (apps.profiles.forms.AdminStudentForm method)}

\begin{fulllineitems}
\phantomsection\label{generated/apps.profiles.forms:apps.profiles.forms.AdminStudentForm.clean_expertise}\pysiglinewithargsret{\code{AdminStudentForm.}\bfcode{clean\_expertise}}{}{}
\end{fulllineitems}


\index{media (apps.profiles.forms.AdminStudentForm attribute)}

\begin{fulllineitems}
\phantomsection\label{generated/apps.profiles.forms:apps.profiles.forms.AdminStudentForm.media}\pysigline{\code{AdminStudentForm.}\bfcode{media}}{}
\end{fulllineitems}


\end{fulllineitems}


\index{ContactForm (class in apps.profiles.forms)}

\begin{fulllineitems}
\phantomsection\label{generated/apps.profiles.forms:apps.profiles.forms.ContactForm}\pysiglinewithargsret{\strong{class }\code{apps.profiles.forms.}\bfcode{ContactForm}}{\emph{data=None}, \emph{files=None}, \emph{auto\_id='id\_\%s'}, \emph{prefix=None}, \emph{initial=None}, \emph{error\_class=\textless{}class `django.forms.util.ErrorList'\textgreater{}}, \emph{label\_suffix=':'}, \emph{empty\_permitted=False}}{}
Bases: \code{django.forms.forms.Form}

\index{media (apps.profiles.forms.ContactForm attribute)}

\begin{fulllineitems}
\phantomsection\label{generated/apps.profiles.forms:apps.profiles.forms.ContactForm.media}\pysigline{\bfcode{media}}{}
\end{fulllineitems}


\end{fulllineitems}


\index{ContactStudentForm (class in apps.profiles.forms)}

\begin{fulllineitems}
\phantomsection\label{generated/apps.profiles.forms:apps.profiles.forms.ContactStudentForm}\pysiglinewithargsret{\strong{class }\code{apps.profiles.forms.}\bfcode{ContactStudentForm}}{\emph{data=None}, \emph{files=None}, \emph{auto\_id='id\_\%s'}, \emph{prefix=None}, \emph{initial=None}, \emph{error\_class=\textless{}class `django.forms.util.ErrorList'\textgreater{}}, \emph{label\_suffix=':'}, \emph{empty\_permitted=False}}{}
Bases: \code{django.forms.forms.Form}

\index{media (apps.profiles.forms.ContactStudentForm attribute)}

\begin{fulllineitems}
\phantomsection\label{generated/apps.profiles.forms:apps.profiles.forms.ContactStudentForm.media}\pysigline{\bfcode{media}}{}
\end{fulllineitems}


\end{fulllineitems}


\index{CourseForm (class in apps.profiles.forms)}

\begin{fulllineitems}
\phantomsection\label{generated/apps.profiles.forms:apps.profiles.forms.CourseForm}\pysiglinewithargsret{\strong{class }\code{apps.profiles.forms.}\bfcode{CourseForm}}{\emph{data=None}, \emph{files=None}, \emph{auto\_id='id\_\%s'}, \emph{prefix=None}, \emph{initial=None}, \emph{error\_class=\textless{}class `django.forms.util.ErrorList'\textgreater{}}, \emph{label\_suffix=':'}, \emph{empty\_permitted=False}, \emph{instance=None}}{}
Bases: {\hyperref[generated/apps.profiles.forms:apps.profiles.forms.AdminCourseForm]{\code{apps.profiles.forms.AdminCourseForm}}}

\index{CourseForm.Meta (class in apps.profiles.forms)}

\begin{fulllineitems}
\phantomsection\label{generated/apps.profiles.forms:apps.profiles.forms.CourseForm.Meta}\pysigline{\strong{class }\bfcode{Meta}}{}
Bases: \code{apps.profiles.forms.Meta}

\end{fulllineitems}


\index{media (apps.profiles.forms.CourseForm attribute)}

\begin{fulllineitems}
\phantomsection\label{generated/apps.profiles.forms:apps.profiles.forms.CourseForm.media}\pysigline{\code{CourseForm.}\bfcode{media}}{}
\end{fulllineitems}


\end{fulllineitems}


\index{FacultyForm (class in apps.profiles.forms)}

\begin{fulllineitems}
\phantomsection\label{generated/apps.profiles.forms:apps.profiles.forms.FacultyForm}\pysiglinewithargsret{\strong{class }\code{apps.profiles.forms.}\bfcode{FacultyForm}}{\emph{data=None}, \emph{files=None}, \emph{auto\_id='id\_\%s'}, \emph{prefix=None}, \emph{initial=None}, \emph{error\_class=\textless{}class `django.forms.util.ErrorList'\textgreater{}}, \emph{label\_suffix=':'}, \emph{empty\_permitted=False}, \emph{instance=None}}{}
Bases: {\hyperref[generated/apps.profiles.forms:apps.profiles.forms.AdminFacultyForm]{\code{apps.profiles.forms.AdminFacultyForm}}}

\index{FacultyForm.Meta (class in apps.profiles.forms)}

\begin{fulllineitems}
\phantomsection\label{generated/apps.profiles.forms:apps.profiles.forms.FacultyForm.Meta}\pysigline{\strong{class }\bfcode{Meta}}{}
Bases: \code{apps.profiles.forms.Meta}

\end{fulllineitems}


\index{media (apps.profiles.forms.FacultyForm attribute)}

\begin{fulllineitems}
\phantomsection\label{generated/apps.profiles.forms:apps.profiles.forms.FacultyForm.media}\pysigline{\code{FacultyForm.}\bfcode{media}}{}
\end{fulllineitems}


\end{fulllineitems}


\index{FilterForm (class in apps.profiles.forms)}

\begin{fulllineitems}
\phantomsection\label{generated/apps.profiles.forms:apps.profiles.forms.FilterForm}\pysiglinewithargsret{\strong{class }\code{apps.profiles.forms.}\bfcode{FilterForm}}{\emph{data=None}, \emph{files=None}, \emph{auto\_id='id\_\%s'}, \emph{prefix=None}, \emph{initial=None}, \emph{error\_class=\textless{}class `django.forms.util.ErrorList'\textgreater{}}, \emph{label\_suffix=':'}, \emph{empty\_permitted=False}}{}
Bases: \code{django.forms.forms.Form}

\index{media (apps.profiles.forms.FilterForm attribute)}

\begin{fulllineitems}
\phantomsection\label{generated/apps.profiles.forms:apps.profiles.forms.FilterForm.media}\pysigline{\bfcode{media}}{}
\end{fulllineitems}


\end{fulllineitems}


\index{InvitationForm (class in apps.profiles.forms)}

\begin{fulllineitems}
\phantomsection\label{generated/apps.profiles.forms:apps.profiles.forms.InvitationForm}\pysiglinewithargsret{\strong{class }\code{apps.profiles.forms.}\bfcode{InvitationForm}}{\emph{data=None}, \emph{files=None}, \emph{auto\_id='id\_\%s'}, \emph{prefix=None}, \emph{initial=None}, \emph{error\_class=\textless{}class `django.forms.util.ErrorList'\textgreater{}}, \emph{label\_suffix=':'}, \emph{empty\_permitted=False}}{}
Bases: \code{django.forms.forms.Form}

\index{media (apps.profiles.forms.InvitationForm attribute)}

\begin{fulllineitems}
\phantomsection\label{generated/apps.profiles.forms:apps.profiles.forms.InvitationForm.media}\pysigline{\bfcode{media}}{}
\end{fulllineitems}


\end{fulllineitems}


\index{OrganizationForm (class in apps.profiles.forms)}

\begin{fulllineitems}
\phantomsection\label{generated/apps.profiles.forms:apps.profiles.forms.OrganizationForm}\pysiglinewithargsret{\strong{class }\code{apps.profiles.forms.}\bfcode{OrganizationForm}}{\emph{data=None}, \emph{files=None}, \emph{auto\_id='id\_\%s'}, \emph{prefix=None}, \emph{initial=None}, \emph{error\_class=\textless{}class `django.forms.util.ErrorList'\textgreater{}}, \emph{label\_suffix=':'}, \emph{empty\_permitted=False}, \emph{instance=None}}{}
Bases: \code{django.forms.models.ModelForm}

\index{OrganizationForm.Meta (class in apps.profiles.forms)}

\begin{fulllineitems}
\phantomsection\label{generated/apps.profiles.forms:apps.profiles.forms.OrganizationForm.Meta}\pysigline{\strong{class }\bfcode{Meta}}{}~
\index{model (apps.profiles.forms.OrganizationForm.Meta attribute)}

\begin{fulllineitems}
\phantomsection\label{generated/apps.profiles.forms:apps.profiles.forms.OrganizationForm.Meta.model}\pysigline{\bfcode{model}}{}
alias of \code{Organization}

\end{fulllineitems}


\end{fulllineitems}


\index{media (apps.profiles.forms.OrganizationForm attribute)}

\begin{fulllineitems}
\phantomsection\label{generated/apps.profiles.forms:apps.profiles.forms.OrganizationForm.media}\pysigline{\code{OrganizationForm.}\bfcode{media}}{}
\end{fulllineitems}


\end{fulllineitems}


\index{PersonActivateForm (class in apps.profiles.forms)}

\begin{fulllineitems}
\phantomsection\label{generated/apps.profiles.forms:apps.profiles.forms.PersonActivateForm}\pysiglinewithargsret{\strong{class }\code{apps.profiles.forms.}\bfcode{PersonActivateForm}}{\emph{data=None}, \emph{files=None}, \emph{auto\_id='id\_\%s'}, \emph{prefix=None}, \emph{initial=None}, \emph{error\_class=\textless{}class `django.forms.util.ErrorList'\textgreater{}}, \emph{label\_suffix=':'}, \emph{empty\_permitted=False}}{}
Bases: \code{django.forms.forms.Form}

\index{media (apps.profiles.forms.PersonActivateForm attribute)}

\begin{fulllineitems}
\phantomsection\label{generated/apps.profiles.forms:apps.profiles.forms.PersonActivateForm.media}\pysigline{\bfcode{media}}{}
\end{fulllineitems}


\end{fulllineitems}


\index{SearchForm (class in apps.profiles.forms)}

\begin{fulllineitems}
\phantomsection\label{generated/apps.profiles.forms:apps.profiles.forms.SearchForm}\pysiglinewithargsret{\strong{class }\code{apps.profiles.forms.}\bfcode{SearchForm}}{\emph{data=None}, \emph{files=None}, \emph{auto\_id='id\_\%s'}, \emph{prefix=None}, \emph{initial=None}, \emph{error\_class=\textless{}class `django.forms.util.ErrorList'\textgreater{}}, \emph{label\_suffix=':'}, \emph{empty\_permitted=False}}{}
Bases: \code{django.forms.forms.Form}

\index{media (apps.profiles.forms.SearchForm attribute)}

\begin{fulllineitems}
\phantomsection\label{generated/apps.profiles.forms:apps.profiles.forms.SearchForm.media}\pysigline{\bfcode{media}}{}
\end{fulllineitems}


\end{fulllineitems}


\index{SectionForm (class in apps.profiles.forms)}

\begin{fulllineitems}
\phantomsection\label{generated/apps.profiles.forms:apps.profiles.forms.SectionForm}\pysiglinewithargsret{\strong{class }\code{apps.profiles.forms.}\bfcode{SectionForm}}{\emph{data=None}, \emph{files=None}, \emph{auto\_id='id\_\%s'}, \emph{prefix=None}, \emph{initial=None}, \emph{error\_class=\textless{}class `django.forms.util.ErrorList'\textgreater{}}, \emph{label\_suffix=':'}, \emph{empty\_permitted=False}, \emph{instance=None}}{}
Bases: {\hyperref[generated/apps.profiles.forms:apps.profiles.forms.AdminSectionForm]{\code{apps.profiles.forms.AdminSectionForm}}}

\index{SectionForm.Meta (class in apps.profiles.forms)}

\begin{fulllineitems}
\phantomsection\label{generated/apps.profiles.forms:apps.profiles.forms.SectionForm.Meta}\pysiglinewithargsret{\strong{class }\bfcode{Meta}}{\emph{data=None}, \emph{files=None}, \emph{auto\_id='id\_\%s'}, \emph{prefix=None}, \emph{initial=None}, \emph{error\_class=\textless{}class `django.forms.util.ErrorList'\textgreater{}}, \emph{label\_suffix=':'}, \emph{empty\_permitted=False}, \emph{instance=None}}{}
Bases: {\hyperref[generated/apps.profiles.forms:apps.profiles.forms.AdminSectionForm]{\code{apps.profiles.forms.AdminSectionForm}}}

\index{media (apps.profiles.forms.SectionForm.Meta attribute)}

\begin{fulllineitems}
\phantomsection\label{generated/apps.profiles.forms:apps.profiles.forms.SectionForm.Meta.media}\pysigline{\bfcode{media}}{}
\end{fulllineitems}


\end{fulllineitems}


\index{media (apps.profiles.forms.SectionForm attribute)}

\begin{fulllineitems}
\phantomsection\label{generated/apps.profiles.forms:apps.profiles.forms.SectionForm.media}\pysigline{\code{SectionForm.}\bfcode{media}}{}
\end{fulllineitems}


\end{fulllineitems}


\index{StaffForm (class in apps.profiles.forms)}

\begin{fulllineitems}
\phantomsection\label{generated/apps.profiles.forms:apps.profiles.forms.StaffForm}\pysiglinewithargsret{\strong{class }\code{apps.profiles.forms.}\bfcode{StaffForm}}{\emph{data=None}, \emph{files=None}, \emph{auto\_id='id\_\%s'}, \emph{prefix=None}, \emph{initial=None}, \emph{error\_class=\textless{}class `django.forms.util.ErrorList'\textgreater{}}, \emph{label\_suffix=':'}, \emph{empty\_permitted=False}, \emph{instance=None}}{}
Bases: {\hyperref[generated/apps.profiles.forms:apps.profiles.forms.AdminStaffForm]{\code{apps.profiles.forms.AdminStaffForm}}}

\index{StaffForm.Meta (class in apps.profiles.forms)}

\begin{fulllineitems}
\phantomsection\label{generated/apps.profiles.forms:apps.profiles.forms.StaffForm.Meta}\pysigline{\strong{class }\bfcode{Meta}}{}
Bases: \code{apps.profiles.forms.Meta}

\end{fulllineitems}


\index{media (apps.profiles.forms.StaffForm attribute)}

\begin{fulllineitems}
\phantomsection\label{generated/apps.profiles.forms:apps.profiles.forms.StaffForm.media}\pysigline{\code{StaffForm.}\bfcode{media}}{}
\end{fulllineitems}


\end{fulllineitems}


\index{StudentForm (class in apps.profiles.forms)}

\begin{fulllineitems}
\phantomsection\label{generated/apps.profiles.forms:apps.profiles.forms.StudentForm}\pysiglinewithargsret{\strong{class }\code{apps.profiles.forms.}\bfcode{StudentForm}}{\emph{data=None}, \emph{files=None}, \emph{auto\_id='id\_\%s'}, \emph{prefix=None}, \emph{initial=None}, \emph{error\_class=\textless{}class `django.forms.util.ErrorList'\textgreater{}}, \emph{label\_suffix=':'}, \emph{empty\_permitted=False}, \emph{instance=None}}{}
Bases: {\hyperref[generated/apps.profiles.forms:apps.profiles.forms.AdminStudentForm]{\code{apps.profiles.forms.AdminStudentForm}}}

\index{StudentForm.Meta (class in apps.profiles.forms)}

\begin{fulllineitems}
\phantomsection\label{generated/apps.profiles.forms:apps.profiles.forms.StudentForm.Meta}\pysigline{\strong{class }\bfcode{Meta}}{}
Bases: \code{apps.profiles.forms.Meta}

\end{fulllineitems}


\index{media (apps.profiles.forms.StudentForm attribute)}

\begin{fulllineitems}
\phantomsection\label{generated/apps.profiles.forms:apps.profiles.forms.StudentForm.media}\pysigline{\code{StudentForm.}\bfcode{media}}{}
\end{fulllineitems}


\end{fulllineitems}


\index{SyllabusForm (class in apps.profiles.forms)}

\begin{fulllineitems}
\phantomsection\label{generated/apps.profiles.forms:apps.profiles.forms.SyllabusForm}\pysiglinewithargsret{\strong{class }\code{apps.profiles.forms.}\bfcode{SyllabusForm}}{\emph{data=None}, \emph{files=None}, \emph{auto\_id='id\_\%s'}, \emph{prefix=None}, \emph{initial=None}, \emph{error\_class=\textless{}class `django.forms.util.ErrorList'\textgreater{}}, \emph{label\_suffix=':'}, \emph{empty\_permitted=False}, \emph{instance=None}}{}
Bases: \code{django.forms.models.ModelForm}

\index{SyllabusForm.Meta (class in apps.profiles.forms)}

\begin{fulllineitems}
\phantomsection\label{generated/apps.profiles.forms:apps.profiles.forms.SyllabusForm.Meta}\pysigline{\strong{class }\bfcode{Meta}}{}~
\index{model (apps.profiles.forms.SyllabusForm.Meta attribute)}

\begin{fulllineitems}
\phantomsection\label{generated/apps.profiles.forms:apps.profiles.forms.SyllabusForm.Meta.model}\pysigline{\bfcode{model}}{}
alias of \code{Section}

\end{fulllineitems}


\end{fulllineitems}


\index{media (apps.profiles.forms.SyllabusForm attribute)}

\begin{fulllineitems}
\phantomsection\label{generated/apps.profiles.forms:apps.profiles.forms.SyllabusForm.media}\pysigline{\code{SyllabusForm.}\bfcode{media}}{}
\end{fulllineitems}


\end{fulllineitems}


\index{WorkForm (class in apps.profiles.forms)}

\begin{fulllineitems}
\phantomsection\label{generated/apps.profiles.forms:apps.profiles.forms.WorkForm}\pysiglinewithargsret{\strong{class }\code{apps.profiles.forms.}\bfcode{WorkForm}}{\emph{data=None}, \emph{files=None}, \emph{auto\_id='id\_\%s'}, \emph{prefix=None}, \emph{initial=None}, \emph{error\_class=\textless{}class `django.forms.util.ErrorList'\textgreater{}}, \emph{label\_suffix=':'}, \emph{empty\_permitted=False}, \emph{instance=None}}{}
Bases: \code{django.forms.models.ModelForm}

\index{WorkForm.Meta (class in apps.profiles.forms)}

\begin{fulllineitems}
\phantomsection\label{generated/apps.profiles.forms:apps.profiles.forms.WorkForm.Meta}\pysigline{\strong{class }\bfcode{Meta}}{}~
\index{model (apps.profiles.forms.WorkForm.Meta attribute)}

\begin{fulllineitems}
\phantomsection\label{generated/apps.profiles.forms:apps.profiles.forms.WorkForm.Meta.model}\pysigline{\bfcode{model}}{}
alias of \code{Work}

\end{fulllineitems}


\end{fulllineitems}


\index{media (apps.profiles.forms.WorkForm attribute)}

\begin{fulllineitems}
\phantomsection\label{generated/apps.profiles.forms:apps.profiles.forms.WorkForm.media}\pysigline{\code{WorkForm.}\bfcode{media}}{}
\end{fulllineitems}


\end{fulllineitems}


\index{WorkURLForm (class in apps.profiles.forms)}

\begin{fulllineitems}
\phantomsection\label{generated/apps.profiles.forms:apps.profiles.forms.WorkURLForm}\pysiglinewithargsret{\strong{class }\code{apps.profiles.forms.}\bfcode{WorkURLForm}}{\emph{data=None}, \emph{files=None}, \emph{auto\_id='id\_\%s'}, \emph{prefix=None}, \emph{initial=None}, \emph{error\_class=\textless{}class `django.forms.util.ErrorList'\textgreater{}}, \emph{label\_suffix=':'}, \emph{empty\_permitted=False}, \emph{instance=None}}{}
Bases: \code{django.forms.models.ModelForm}

\index{WorkURLForm.Meta (class in apps.profiles.forms)}

\begin{fulllineitems}
\phantomsection\label{generated/apps.profiles.forms:apps.profiles.forms.WorkURLForm.Meta}\pysigline{\strong{class }\bfcode{Meta}}{}~
\index{model (apps.profiles.forms.WorkURLForm.Meta attribute)}

\begin{fulllineitems}
\phantomsection\label{generated/apps.profiles.forms:apps.profiles.forms.WorkURLForm.Meta.model}\pysigline{\bfcode{model}}{}
alias of \code{WorkURL}

\end{fulllineitems}


\end{fulllineitems}


\index{media (apps.profiles.forms.WorkURLForm attribute)}

\begin{fulllineitems}
\phantomsection\label{generated/apps.profiles.forms:apps.profiles.forms.WorkURLForm.media}\pysigline{\code{WorkURLForm.}\bfcode{media}}{}
\end{fulllineitems}


\end{fulllineitems}



\chapter{Profiles: Handlers}
\label{generated/apps.profiles.handlers:profiles-handlers}\label{generated/apps.profiles.handlers::doc}\label{generated/apps.profiles.handlers:module-apps.profiles.handlers}
\index{apps.profiles.handlers (module)}
Created on Aug 18, 2010

@author: edwards

\index{CourseHandler (class in apps.profiles.handlers)}

\begin{fulllineitems}
\phantomsection\label{generated/apps.profiles.handlers:apps.profiles.handlers.CourseHandler}\pysigline{\strong{class }\code{apps.profiles.handlers.}\bfcode{CourseHandler}}{}
Bases: \code{piston.handler.BaseHandler}

This handler returns courses.

\index{model (apps.profiles.handlers.CourseHandler attribute)}

\begin{fulllineitems}
\phantomsection\label{generated/apps.profiles.handlers:apps.profiles.handlers.CourseHandler.model}\pysigline{\bfcode{model}}{}
alias of \code{Course}

\end{fulllineitems}


\index{queryset() (apps.profiles.handlers.CourseHandler method)}

\begin{fulllineitems}
\phantomsection\label{generated/apps.profiles.handlers:apps.profiles.handlers.CourseHandler.queryset}\pysiglinewithargsret{\bfcode{queryset}}{\emph{request}}{}
\end{fulllineitems}


\end{fulllineitems}


\index{ExpertiseHandler (class in apps.profiles.handlers)}

\begin{fulllineitems}
\phantomsection\label{generated/apps.profiles.handlers:apps.profiles.handlers.ExpertiseHandler}\pysigline{\strong{class }\code{apps.profiles.handlers.}\bfcode{ExpertiseHandler}}{}
Bases: \code{piston.handler.BaseHandler}

This handler returns areas of expertise.

\index{model (apps.profiles.handlers.ExpertiseHandler attribute)}

\begin{fulllineitems}
\phantomsection\label{generated/apps.profiles.handlers:apps.profiles.handlers.ExpertiseHandler.model}\pysigline{\bfcode{model}}{}
alias of \code{Expertise}

\end{fulllineitems}


\index{queryset() (apps.profiles.handlers.ExpertiseHandler method)}

\begin{fulllineitems}
\phantomsection\label{generated/apps.profiles.handlers:apps.profiles.handlers.ExpertiseHandler.queryset}\pysiglinewithargsret{\bfcode{queryset}}{\emph{request}}{}
\end{fulllineitems}


\end{fulllineitems}


\index{FacultyHandler (class in apps.profiles.handlers)}

\begin{fulllineitems}
\phantomsection\label{generated/apps.profiles.handlers:apps.profiles.handlers.FacultyHandler}\pysigline{\strong{class }\code{apps.profiles.handlers.}\bfcode{FacultyHandler}}{}
Bases: \code{piston.handler.BaseHandler}

This handler returns faculty members.

\index{model (apps.profiles.handlers.FacultyHandler attribute)}

\begin{fulllineitems}
\phantomsection\label{generated/apps.profiles.handlers:apps.profiles.handlers.FacultyHandler.model}\pysigline{\bfcode{model}}{}
alias of \code{FacultyMember}

\end{fulllineitems}


\index{queryset() (apps.profiles.handlers.FacultyHandler method)}

\begin{fulllineitems}
\phantomsection\label{generated/apps.profiles.handlers:apps.profiles.handlers.FacultyHandler.queryset}\pysiglinewithargsret{\bfcode{queryset}}{\emph{request}}{}
\end{fulllineitems}


\end{fulllineitems}


\index{FacultyResultHandler (class in apps.profiles.handlers)}

\begin{fulllineitems}
\phantomsection\label{generated/apps.profiles.handlers:apps.profiles.handlers.FacultyResultHandler}\pysigline{\strong{class }\code{apps.profiles.handlers.}\bfcode{FacultyResultHandler}}{}
Bases: {\hyperref[generated/apps.profiles.handlers:apps.profiles.handlers.ResultHandler]{\code{apps.profiles.handlers.ResultHandler}}}

This \code{ResultHandler} allows for searches to be faceted to just return the faculty
members within a search result set.

\index{read() (apps.profiles.handlers.FacultyResultHandler method)}

\begin{fulllineitems}
\phantomsection\label{generated/apps.profiles.handlers:apps.profiles.handlers.FacultyResultHandler.read}\pysiglinewithargsret{\bfcode{read}}{\emph{request}, \emph{model=None}, \emph{*args}, \emph{**kwargs}}{}
\end{fulllineitems}


\end{fulllineitems}


\index{OrganizationHandler (class in apps.profiles.handlers)}

\begin{fulllineitems}
\phantomsection\label{generated/apps.profiles.handlers:apps.profiles.handlers.OrganizationHandler}\pysigline{\strong{class }\code{apps.profiles.handlers.}\bfcode{OrganizationHandler}}{}
Bases: \code{piston.handler.BaseHandler}

This handler returns organizations.

\index{model (apps.profiles.handlers.OrganizationHandler attribute)}

\begin{fulllineitems}
\phantomsection\label{generated/apps.profiles.handlers:apps.profiles.handlers.OrganizationHandler.model}\pysigline{\bfcode{model}}{}
alias of \code{Organization}

\end{fulllineitems}


\index{queryset() (apps.profiles.handlers.OrganizationHandler method)}

\begin{fulllineitems}
\phantomsection\label{generated/apps.profiles.handlers:apps.profiles.handlers.OrganizationHandler.queryset}\pysiglinewithargsret{\bfcode{queryset}}{\emph{request}}{}
\end{fulllineitems}


\end{fulllineitems}


\index{PersonHandler (class in apps.profiles.handlers)}

\begin{fulllineitems}
\phantomsection\label{generated/apps.profiles.handlers:apps.profiles.handlers.PersonHandler}\pysigline{\strong{class }\code{apps.profiles.handlers.}\bfcode{PersonHandler}}{}
Bases: \code{piston.handler.BaseHandler}

This handler is a generic handler for all people.  It's most useful in
queries like with affiliations where people of multiple roles could be 
returned (e.g. FacultyMember, Student, Staff)

\index{model (apps.profiles.handlers.PersonHandler attribute)}

\begin{fulllineitems}
\phantomsection\label{generated/apps.profiles.handlers:apps.profiles.handlers.PersonHandler.model}\pysigline{\bfcode{model}}{}
alias of \code{Person}

\end{fulllineitems}


\index{queryset() (apps.profiles.handlers.PersonHandler method)}

\begin{fulllineitems}
\phantomsection\label{generated/apps.profiles.handlers:apps.profiles.handlers.PersonHandler.queryset}\pysiglinewithargsret{\bfcode{queryset}}{\emph{request}}{}
\end{fulllineitems}


\end{fulllineitems}


\index{ProjectHandler (class in apps.profiles.handlers)}

\begin{fulllineitems}
\phantomsection\label{generated/apps.profiles.handlers:apps.profiles.handlers.ProjectHandler}\pysigline{\strong{class }\code{apps.profiles.handlers.}\bfcode{ProjectHandler}}{}
Bases: \code{piston.handler.BaseHandler}

\index{model (apps.profiles.handlers.ProjectHandler attribute)}

\begin{fulllineitems}
\phantomsection\label{generated/apps.profiles.handlers:apps.profiles.handlers.ProjectHandler.model}\pysigline{\bfcode{model}}{}
alias of \code{Project}

\end{fulllineitems}


\index{queryset() (apps.profiles.handlers.ProjectHandler method)}

\begin{fulllineitems}
\phantomsection\label{generated/apps.profiles.handlers:apps.profiles.handlers.ProjectHandler.queryset}\pysiglinewithargsret{\bfcode{queryset}}{\emph{request}}{}
\end{fulllineitems}


\end{fulllineitems}


\index{RecentFacultyHandler (class in apps.profiles.handlers)}

\begin{fulllineitems}
\phantomsection\label{generated/apps.profiles.handlers:apps.profiles.handlers.RecentFacultyHandler}\pysigline{\strong{class }\code{apps.profiles.handlers.}\bfcode{RecentFacultyHandler}}{}
Bases: {\hyperref[generated/apps.profiles.handlers:apps.profiles.handlers.FacultyHandler]{\code{apps.profiles.handlers.FacultyHandler}}}

This handler returns the 50 most recently updated faculty members.

\index{queryset() (apps.profiles.handlers.RecentFacultyHandler method)}

\begin{fulllineitems}
\phantomsection\label{generated/apps.profiles.handlers:apps.profiles.handlers.RecentFacultyHandler.queryset}\pysiglinewithargsret{\bfcode{queryset}}{\emph{request}}{}
\end{fulllineitems}


\end{fulllineitems}


\index{RecentStudentHandler (class in apps.profiles.handlers)}

\begin{fulllineitems}
\phantomsection\label{generated/apps.profiles.handlers:apps.profiles.handlers.RecentStudentHandler}\pysigline{\strong{class }\code{apps.profiles.handlers.}\bfcode{RecentStudentHandler}}{}
Bases: {\hyperref[generated/apps.profiles.handlers:apps.profiles.handlers.StudentHandler]{\code{apps.profiles.handlers.StudentHandler}}}

This handler returns the 50 most recently updated students.

\index{queryset() (apps.profiles.handlers.RecentStudentHandler method)}

\begin{fulllineitems}
\phantomsection\label{generated/apps.profiles.handlers:apps.profiles.handlers.RecentStudentHandler.queryset}\pysiglinewithargsret{\bfcode{queryset}}{\emph{request}}{}
\end{fulllineitems}


\end{fulllineitems}


\index{RecentWorkHandler (class in apps.profiles.handlers)}

\begin{fulllineitems}
\phantomsection\label{generated/apps.profiles.handlers:apps.profiles.handlers.RecentWorkHandler}\pysigline{\strong{class }\code{apps.profiles.handlers.}\bfcode{RecentWorkHandler}}{}
Bases: {\hyperref[generated/apps.profiles.handlers:apps.profiles.handlers.WorkHandler]{\code{apps.profiles.handlers.WorkHandler}}}

This handler returns the 50 most recently updated works.

\index{queryset() (apps.profiles.handlers.RecentWorkHandler method)}

\begin{fulllineitems}
\phantomsection\label{generated/apps.profiles.handlers:apps.profiles.handlers.RecentWorkHandler.queryset}\pysiglinewithargsret{\bfcode{queryset}}{\emph{request}}{}
\end{fulllineitems}


\end{fulllineitems}


\index{ResultHandler (class in apps.profiles.handlers)}

\begin{fulllineitems}
\phantomsection\label{generated/apps.profiles.handlers:apps.profiles.handlers.ResultHandler}\pysigline{\strong{class }\code{apps.profiles.handlers.}\bfcode{ResultHandler}}{}
Bases: \code{piston.handler.BaseHandler}

This handler is unique in that is uses a \code{SearchQuerySet} from \code{haystack}
rather than the typical Django \code{QuerySet}.  This requires a bit of customization
with the read output, but allows us to treat \code{haystack} search results the
same way as any other set coming out of the \code{piston} API application.

\index{convert\_search\_queryset() (apps.profiles.handlers.ResultHandler method)}

\begin{fulllineitems}
\phantomsection\label{generated/apps.profiles.handlers:apps.profiles.handlers.ResultHandler.convert_search_queryset}\pysiglinewithargsret{\bfcode{convert\_search\_queryset}}{\emph{queryset}, \emph{mlt=False}}{}
\end{fulllineitems}


\index{model (apps.profiles.handlers.ResultHandler attribute)}

\begin{fulllineitems}
\phantomsection\label{generated/apps.profiles.handlers:apps.profiles.handlers.ResultHandler.model}\pysigline{\bfcode{model}}{}
alias of \code{SearchResult}

\end{fulllineitems}


\index{read() (apps.profiles.handlers.ResultHandler method)}

\begin{fulllineitems}
\phantomsection\label{generated/apps.profiles.handlers:apps.profiles.handlers.ResultHandler.read}\pysiglinewithargsret{\bfcode{read}}{\emph{request}, \emph{model=None}, \emph{*args}, \emph{**kwargs}}{}
\end{fulllineitems}


\end{fulllineitems}


\index{StudentHandler (class in apps.profiles.handlers)}

\begin{fulllineitems}
\phantomsection\label{generated/apps.profiles.handlers:apps.profiles.handlers.StudentHandler}\pysigline{\strong{class }\code{apps.profiles.handlers.}\bfcode{StudentHandler}}{}
Bases: \code{piston.handler.BaseHandler}

This handler returns students.

\index{model (apps.profiles.handlers.StudentHandler attribute)}

\begin{fulllineitems}
\phantomsection\label{generated/apps.profiles.handlers:apps.profiles.handlers.StudentHandler.model}\pysigline{\bfcode{model}}{}
alias of \code{Student}

\end{fulllineitems}


\index{queryset() (apps.profiles.handlers.StudentHandler method)}

\begin{fulllineitems}
\phantomsection\label{generated/apps.profiles.handlers:apps.profiles.handlers.StudentHandler.queryset}\pysiglinewithargsret{\bfcode{queryset}}{\emph{request}}{}
\end{fulllineitems}


\end{fulllineitems}


\index{StudentResultHandler (class in apps.profiles.handlers)}

\begin{fulllineitems}
\phantomsection\label{generated/apps.profiles.handlers:apps.profiles.handlers.StudentResultHandler}\pysigline{\strong{class }\code{apps.profiles.handlers.}\bfcode{StudentResultHandler}}{}
Bases: {\hyperref[generated/apps.profiles.handlers:apps.profiles.handlers.ResultHandler]{\code{apps.profiles.handlers.ResultHandler}}}

This \code{ResultHandler} allows for searches to be faceted to just return the students
within a search result set.

\index{read() (apps.profiles.handlers.StudentResultHandler method)}

\begin{fulllineitems}
\phantomsection\label{generated/apps.profiles.handlers:apps.profiles.handlers.StudentResultHandler.read}\pysiglinewithargsret{\bfcode{read}}{\emph{request}, \emph{model=None}, \emph{*args}, \emph{**kwargs}}{}
\end{fulllineitems}


\end{fulllineitems}


\index{TagHandler (class in apps.profiles.handlers)}

\begin{fulllineitems}
\phantomsection\label{generated/apps.profiles.handlers:apps.profiles.handlers.TagHandler}\pysigline{\strong{class }\code{apps.profiles.handlers.}\bfcode{TagHandler}}{}
Bases: \code{piston.handler.BaseHandler}

This handler returns tags.

\index{model (apps.profiles.handlers.TagHandler attribute)}

\begin{fulllineitems}
\phantomsection\label{generated/apps.profiles.handlers:apps.profiles.handlers.TagHandler.model}\pysigline{\bfcode{model}}{}
alias of \code{Tag}

\end{fulllineitems}


\end{fulllineitems}


\index{TaggedPersonHandler (class in apps.profiles.handlers)}

\begin{fulllineitems}
\phantomsection\label{generated/apps.profiles.handlers:apps.profiles.handlers.TaggedPersonHandler}\pysigline{\strong{class }\code{apps.profiles.handlers.}\bfcode{TaggedPersonHandler}}{}
Bases: \code{piston.handler.BaseHandler}

This handler returns tagged items, but restricts the results to \code{Student} and \code{FacultyMember} objects.

\index{model (apps.profiles.handlers.TaggedPersonHandler attribute)}

\begin{fulllineitems}
\phantomsection\label{generated/apps.profiles.handlers:apps.profiles.handlers.TaggedPersonHandler.model}\pysigline{\bfcode{model}}{}
alias of \code{TaggedItem}

\end{fulllineitems}


\index{queryset() (apps.profiles.handlers.TaggedPersonHandler method)}

\begin{fulllineitems}
\phantomsection\label{generated/apps.profiles.handlers:apps.profiles.handlers.TaggedPersonHandler.queryset}\pysiglinewithargsret{\bfcode{queryset}}{\emph{request}}{}
\end{fulllineitems}


\end{fulllineitems}


\index{TaggedWorkHandler (class in apps.profiles.handlers)}

\begin{fulllineitems}
\phantomsection\label{generated/apps.profiles.handlers:apps.profiles.handlers.TaggedWorkHandler}\pysigline{\strong{class }\code{apps.profiles.handlers.}\bfcode{TaggedWorkHandler}}{}
Bases: \code{piston.handler.BaseHandler}

This handler returns tagged items, but restricts the results to \code{Work} objects.

\index{model (apps.profiles.handlers.TaggedWorkHandler attribute)}

\begin{fulllineitems}
\phantomsection\label{generated/apps.profiles.handlers:apps.profiles.handlers.TaggedWorkHandler.model}\pysigline{\bfcode{model}}{}
alias of \code{TaggedItem}

\end{fulllineitems}


\index{queryset() (apps.profiles.handlers.TaggedWorkHandler method)}

\begin{fulllineitems}
\phantomsection\label{generated/apps.profiles.handlers:apps.profiles.handlers.TaggedWorkHandler.queryset}\pysiglinewithargsret{\bfcode{queryset}}{\emph{request}}{}
\end{fulllineitems}


\end{fulllineitems}


\index{WorkHandler (class in apps.profiles.handlers)}

\begin{fulllineitems}
\phantomsection\label{generated/apps.profiles.handlers:apps.profiles.handlers.WorkHandler}\pysigline{\strong{class }\code{apps.profiles.handlers.}\bfcode{WorkHandler}}{}
Bases: \code{piston.handler.BaseHandler}

This handler returns works.

\index{model (apps.profiles.handlers.WorkHandler attribute)}

\begin{fulllineitems}
\phantomsection\label{generated/apps.profiles.handlers:apps.profiles.handlers.WorkHandler.model}\pysigline{\bfcode{model}}{}
alias of \code{Work}

\end{fulllineitems}


\end{fulllineitems}


\index{WorkResultHandler (class in apps.profiles.handlers)}

\begin{fulllineitems}
\phantomsection\label{generated/apps.profiles.handlers:apps.profiles.handlers.WorkResultHandler}\pysigline{\strong{class }\code{apps.profiles.handlers.}\bfcode{WorkResultHandler}}{}
Bases: {\hyperref[generated/apps.profiles.handlers:apps.profiles.handlers.ResultHandler]{\code{apps.profiles.handlers.ResultHandler}}}

This \code{ResultHandler} allows for searches to be faceted to just return the works
within a search result set.

\index{read() (apps.profiles.handlers.WorkResultHandler method)}

\begin{fulllineitems}
\phantomsection\label{generated/apps.profiles.handlers:apps.profiles.handlers.WorkResultHandler.read}\pysiglinewithargsret{\bfcode{read}}{\emph{request}, \emph{model=None}, \emph{*args}, \emph{**kwargs}}{}
\end{fulllineitems}


\end{fulllineitems}



\chapter{Profiles: Lookups}
\label{generated/apps.profiles.lookups:profiles-lookups}\label{generated/apps.profiles.lookups:module-apps.profiles.lookups}\label{generated/apps.profiles.lookups::doc}
\index{apps.profiles.lookups (module)}
\index{CourseLookup (class in apps.profiles.lookups)}

\begin{fulllineitems}
\phantomsection\label{generated/apps.profiles.lookups:apps.profiles.lookups.CourseLookup}\pysigline{\strong{class }\code{apps.profiles.lookups.}\bfcode{CourseLookup}}{}
Bases: \code{object}

This lookup pulls in \code{Course} objects to complete ajax-powered form fields.

\index{format\_item() (apps.profiles.lookups.CourseLookup method)}

\begin{fulllineitems}
\phantomsection\label{generated/apps.profiles.lookups:apps.profiles.lookups.CourseLookup.format_item}\pysiglinewithargsret{\bfcode{format\_item}}{\emph{course}}{}
\end{fulllineitems}


\index{format\_result() (apps.profiles.lookups.CourseLookup method)}

\begin{fulllineitems}
\phantomsection\label{generated/apps.profiles.lookups:apps.profiles.lookups.CourseLookup.format_result}\pysiglinewithargsret{\bfcode{format\_result}}{\emph{course}}{}
\end{fulllineitems}


\index{get\_objects() (apps.profiles.lookups.CourseLookup method)}

\begin{fulllineitems}
\phantomsection\label{generated/apps.profiles.lookups:apps.profiles.lookups.CourseLookup.get_objects}\pysiglinewithargsret{\bfcode{get\_objects}}{\emph{ids}}{}
\end{fulllineitems}


\index{get\_query() (apps.profiles.lookups.CourseLookup method)}

\begin{fulllineitems}
\phantomsection\label{generated/apps.profiles.lookups:apps.profiles.lookups.CourseLookup.get_query}\pysiglinewithargsret{\bfcode{get\_query}}{\emph{q}, \emph{request}}{}
\end{fulllineitems}


\end{fulllineitems}


\index{DepartmentLookup (class in apps.profiles.lookups)}

\begin{fulllineitems}
\phantomsection\label{generated/apps.profiles.lookups:apps.profiles.lookups.DepartmentLookup}\pysigline{\strong{class }\code{apps.profiles.lookups.}\bfcode{DepartmentLookup}}{}
Bases: \code{object}

This lookup pulls in \code{Department} objects to complete ajax-powered form fields.

\index{format\_item() (apps.profiles.lookups.DepartmentLookup method)}

\begin{fulllineitems}
\phantomsection\label{generated/apps.profiles.lookups:apps.profiles.lookups.DepartmentLookup.format_item}\pysiglinewithargsret{\bfcode{format\_item}}{\emph{object}}{}
\end{fulllineitems}


\index{format\_result() (apps.profiles.lookups.DepartmentLookup method)}

\begin{fulllineitems}
\phantomsection\label{generated/apps.profiles.lookups:apps.profiles.lookups.DepartmentLookup.format_result}\pysiglinewithargsret{\bfcode{format\_result}}{\emph{object}}{}
\end{fulllineitems}


\index{get\_objects() (apps.profiles.lookups.DepartmentLookup method)}

\begin{fulllineitems}
\phantomsection\label{generated/apps.profiles.lookups:apps.profiles.lookups.DepartmentLookup.get_objects}\pysiglinewithargsret{\bfcode{get\_objects}}{\emph{ids}}{}
\end{fulllineitems}


\index{get\_query() (apps.profiles.lookups.DepartmentLookup method)}

\begin{fulllineitems}
\phantomsection\label{generated/apps.profiles.lookups:apps.profiles.lookups.DepartmentLookup.get_query}\pysiglinewithargsret{\bfcode{get\_query}}{\emph{q}, \emph{request}}{}
\end{fulllineitems}


\end{fulllineitems}


\index{DivisionLookup (class in apps.profiles.lookups)}

\begin{fulllineitems}
\phantomsection\label{generated/apps.profiles.lookups:apps.profiles.lookups.DivisionLookup}\pysigline{\strong{class }\code{apps.profiles.lookups.}\bfcode{DivisionLookup}}{}
Bases: \code{object}

This lookup pulls in \code{Division} objects to complete ajax-powered form fields.

\index{format\_item() (apps.profiles.lookups.DivisionLookup method)}

\begin{fulllineitems}
\phantomsection\label{generated/apps.profiles.lookups:apps.profiles.lookups.DivisionLookup.format_item}\pysiglinewithargsret{\bfcode{format\_item}}{\emph{object}}{}
\end{fulllineitems}


\index{format\_result() (apps.profiles.lookups.DivisionLookup method)}

\begin{fulllineitems}
\phantomsection\label{generated/apps.profiles.lookups:apps.profiles.lookups.DivisionLookup.format_result}\pysiglinewithargsret{\bfcode{format\_result}}{\emph{object}}{}
\end{fulllineitems}


\index{get\_objects() (apps.profiles.lookups.DivisionLookup method)}

\begin{fulllineitems}
\phantomsection\label{generated/apps.profiles.lookups:apps.profiles.lookups.DivisionLookup.get_objects}\pysiglinewithargsret{\bfcode{get\_objects}}{\emph{ids}}{}
\end{fulllineitems}


\index{get\_query() (apps.profiles.lookups.DivisionLookup method)}

\begin{fulllineitems}
\phantomsection\label{generated/apps.profiles.lookups:apps.profiles.lookups.DivisionLookup.get_query}\pysiglinewithargsret{\bfcode{get\_query}}{\emph{q}, \emph{request}}{}
\end{fulllineitems}


\end{fulllineitems}


\index{PersonLookup (class in apps.profiles.lookups)}

\begin{fulllineitems}
\phantomsection\label{generated/apps.profiles.lookups:apps.profiles.lookups.PersonLookup}\pysigline{\strong{class }\code{apps.profiles.lookups.}\bfcode{PersonLookup}}{}
Bases: \code{object}

This lookup pulls in \code{Person} objects to complete ajax-powered form fields.

\index{format\_item() (apps.profiles.lookups.PersonLookup method)}

\begin{fulllineitems}
\phantomsection\label{generated/apps.profiles.lookups:apps.profiles.lookups.PersonLookup.format_item}\pysiglinewithargsret{\bfcode{format\_item}}{\emph{person}}{}
\end{fulllineitems}


\index{format\_result() (apps.profiles.lookups.PersonLookup method)}

\begin{fulllineitems}
\phantomsection\label{generated/apps.profiles.lookups:apps.profiles.lookups.PersonLookup.format_result}\pysiglinewithargsret{\bfcode{format\_result}}{\emph{person}}{}
\end{fulllineitems}


\index{get\_objects() (apps.profiles.lookups.PersonLookup method)}

\begin{fulllineitems}
\phantomsection\label{generated/apps.profiles.lookups:apps.profiles.lookups.PersonLookup.get_objects}\pysiglinewithargsret{\bfcode{get\_objects}}{\emph{ids}}{}
\end{fulllineitems}


\index{get\_query() (apps.profiles.lookups.PersonLookup method)}

\begin{fulllineitems}
\phantomsection\label{generated/apps.profiles.lookups:apps.profiles.lookups.PersonLookup.get_query}\pysiglinewithargsret{\bfcode{get\_query}}{\emph{q}, \emph{request}}{}
\end{fulllineitems}


\end{fulllineitems}


\index{ProgramLookup (class in apps.profiles.lookups)}

\begin{fulllineitems}
\phantomsection\label{generated/apps.profiles.lookups:apps.profiles.lookups.ProgramLookup}\pysigline{\strong{class }\code{apps.profiles.lookups.}\bfcode{ProgramLookup}}{}
Bases: \code{object}

This lookup pulls in \code{Program} objects to complete ajax-powered form fields.

\index{format\_item() (apps.profiles.lookups.ProgramLookup method)}

\begin{fulllineitems}
\phantomsection\label{generated/apps.profiles.lookups:apps.profiles.lookups.ProgramLookup.format_item}\pysiglinewithargsret{\bfcode{format\_item}}{\emph{object}}{}
\end{fulllineitems}


\index{format\_result() (apps.profiles.lookups.ProgramLookup method)}

\begin{fulllineitems}
\phantomsection\label{generated/apps.profiles.lookups:apps.profiles.lookups.ProgramLookup.format_result}\pysiglinewithargsret{\bfcode{format\_result}}{\emph{object}}{}
\end{fulllineitems}


\index{get\_objects() (apps.profiles.lookups.ProgramLookup method)}

\begin{fulllineitems}
\phantomsection\label{generated/apps.profiles.lookups:apps.profiles.lookups.ProgramLookup.get_objects}\pysiglinewithargsret{\bfcode{get\_objects}}{\emph{ids}}{}
\end{fulllineitems}


\index{get\_query() (apps.profiles.lookups.ProgramLookup method)}

\begin{fulllineitems}
\phantomsection\label{generated/apps.profiles.lookups:apps.profiles.lookups.ProgramLookup.get_query}\pysiglinewithargsret{\bfcode{get\_query}}{\emph{q}, \emph{request}}{}
\end{fulllineitems}


\end{fulllineitems}


\index{SchoolLookup (class in apps.profiles.lookups)}

\begin{fulllineitems}
\phantomsection\label{generated/apps.profiles.lookups:apps.profiles.lookups.SchoolLookup}\pysigline{\strong{class }\code{apps.profiles.lookups.}\bfcode{SchoolLookup}}{}
Bases: \code{object}

This lookup pulls in \code{School} objects to complete ajax-powered form fields.

\index{format\_item() (apps.profiles.lookups.SchoolLookup method)}

\begin{fulllineitems}
\phantomsection\label{generated/apps.profiles.lookups:apps.profiles.lookups.SchoolLookup.format_item}\pysiglinewithargsret{\bfcode{format\_item}}{\emph{object}}{}
\end{fulllineitems}


\index{format\_result() (apps.profiles.lookups.SchoolLookup method)}

\begin{fulllineitems}
\phantomsection\label{generated/apps.profiles.lookups:apps.profiles.lookups.SchoolLookup.format_result}\pysiglinewithargsret{\bfcode{format\_result}}{\emph{object}}{}
\end{fulllineitems}


\index{get\_objects() (apps.profiles.lookups.SchoolLookup method)}

\begin{fulllineitems}
\phantomsection\label{generated/apps.profiles.lookups:apps.profiles.lookups.SchoolLookup.get_objects}\pysiglinewithargsret{\bfcode{get\_objects}}{\emph{ids}}{}
\end{fulllineitems}


\index{get\_query() (apps.profiles.lookups.SchoolLookup method)}

\begin{fulllineitems}
\phantomsection\label{generated/apps.profiles.lookups:apps.profiles.lookups.SchoolLookup.get_query}\pysiglinewithargsret{\bfcode{get\_query}}{\emph{q}, \emph{request}}{}
\end{fulllineitems}


\end{fulllineitems}


\index{WorkLookup (class in apps.profiles.lookups)}

\begin{fulllineitems}
\phantomsection\label{generated/apps.profiles.lookups:apps.profiles.lookups.WorkLookup}\pysigline{\strong{class }\code{apps.profiles.lookups.}\bfcode{WorkLookup}}{}
Bases: \code{object}

This lookup pulls in \code{Work} objects to complete ajax-powered form fields.

\index{format\_item() (apps.profiles.lookups.WorkLookup method)}

\begin{fulllineitems}
\phantomsection\label{generated/apps.profiles.lookups:apps.profiles.lookups.WorkLookup.format_item}\pysiglinewithargsret{\bfcode{format\_item}}{\emph{object}}{}
\end{fulllineitems}


\index{format\_result() (apps.profiles.lookups.WorkLookup method)}

\begin{fulllineitems}
\phantomsection\label{generated/apps.profiles.lookups:apps.profiles.lookups.WorkLookup.format_result}\pysiglinewithargsret{\bfcode{format\_result}}{\emph{object}}{}
\end{fulllineitems}


\index{get\_objects() (apps.profiles.lookups.WorkLookup method)}

\begin{fulllineitems}
\phantomsection\label{generated/apps.profiles.lookups:apps.profiles.lookups.WorkLookup.get_objects}\pysiglinewithargsret{\bfcode{get\_objects}}{\emph{ids}}{}
\end{fulllineitems}


\index{get\_query() (apps.profiles.lookups.WorkLookup method)}

\begin{fulllineitems}
\phantomsection\label{generated/apps.profiles.lookups:apps.profiles.lookups.WorkLookup.get_query}\pysiglinewithargsret{\bfcode{get\_query}}{\emph{q}, \emph{request}}{}
\end{fulllineitems}


\end{fulllineitems}



\chapter{Profiles: Backends}
\label{generated/apps.profiles.backends:module-apps.profiles.backends}\label{generated/apps.profiles.backends::doc}\label{generated/apps.profiles.backends:profiles-backends}
\index{apps.profiles.backends (module)}
Created on Mar 2, 2011

@author: edwards

\index{EmailModelBackend (class in apps.profiles.backends)}

\begin{fulllineitems}
\phantomsection\label{generated/apps.profiles.backends:apps.profiles.backends.EmailModelBackend}\pysigline{\strong{class }\code{apps.profiles.backends.}\bfcode{EmailModelBackend}}{}
Bases: \code{django.contrib.auth.backends.ModelBackend}

Authenticates against django.contrib.auth.models.User.  This is an older backend whose use
preceded the SchoolLDAPBackend authentication.

\index{authenticate() (apps.profiles.backends.EmailModelBackend method)}

\begin{fulllineitems}
\phantomsection\label{generated/apps.profiles.backends:apps.profiles.backends.EmailModelBackend.authenticate}\pysiglinewithargsret{\bfcode{authenticate}}{\emph{username=None}, \emph{password=None}}{}
\end{fulllineitems}


\end{fulllineitems}


\index{SchoolLDAPBackend (class in apps.profiles.backends)}

\begin{fulllineitems}
\phantomsection\label{generated/apps.profiles.backends:apps.profiles.backends.SchoolLDAPBackend}\pysigline{\strong{class }\code{apps.profiles.backends.}\bfcode{SchoolLDAPBackend}}{}
Bases: \code{django\_auth\_ldap.backend.LDAPBackend}

This is the LDAP authentication for New School faculty, students and staff.  At present,
it does its best to figure out whether a new person is one of the three roles contained in
the system.  In the future, we will need to find ways to create hybrid profiles for users
who occupy multiple roles.

To look into the LDAP itself, you can use the following commands (replace the example text with actual values)

\begin{Verbatim}[commandchars=\\\{\}]
\PYG{g+gp}{\textgreater{}\textgreater{}\textgreater{} }\PYG{k+kn}{import} \PYG{n+nn}{ldap}
\PYG{g+gp}{\textgreater{}\textgreater{}\textgreater{} }\PYG{n}{conn} \PYG{o}{=} \PYG{n}{ldap}\PYG{o}{.}\PYG{n}{initialize}\PYG{p}{(}\PYG{l+s}{"}\PYG{l+s}{ldaps://your.ldap.server.edu}\PYG{l+s}{"}\PYG{p}{)}
\PYG{g+gp}{\textgreater{}\textgreater{}\textgreater{} }\PYG{n}{conn}\PYG{o}{.}\PYG{n}{bind\PYGZus{}s}\PYG{p}{(}\PYG{l+s}{"}\PYG{l+s}{cn=commonLoginName,o=organization}\PYG{l+s}{"}\PYG{p}{,}\PYG{l+s}{"}\PYG{l+s}{password}\PYG{l+s}{"}\PYG{p}{,}\PYG{n}{ldap}\PYG{o}{.}\PYG{n}{AUTH\PYGZus{}SIMPLE}\PYG{p}{)}
\PYG{g+gp}{\textgreater{}\textgreater{}\textgreater{} }\PYG{n}{conn}\PYG{o}{.}\PYG{n}{search\PYGZus{}s}\PYG{p}{(}\PYG{l+s}{"}\PYG{l+s}{o=organization}\PYG{l+s}{"}\PYG{p}{,}\PYG{n}{ldap}\PYG{o}{.}\PYG{n}{SCOPE\PYGZus{}SUBTREE}\PYG{p}{,} \PYG{l+s}{"}\PYG{l+s}{(\&(objectclass=user)(sn=Lastname)(givenName=Firstname))}\PYG{l+s}{"}\PYG{p}{)}
\end{Verbatim}

\index{get\_or\_create\_user() (apps.profiles.backends.SchoolLDAPBackend method)}

\begin{fulllineitems}
\phantomsection\label{generated/apps.profiles.backends:apps.profiles.backends.SchoolLDAPBackend.get_or_create_user}\pysiglinewithargsret{\bfcode{get\_or\_create\_user}}{\emph{username}, \emph{ldap\_user}}{}
\end{fulllineitems}


\end{fulllineitems}



\chapter{Reporting: Models}
\label{generated/apps.reporting.models::doc}\label{generated/apps.reporting.models:reporting-models}\label{generated/apps.reporting.models:module-apps.reporting.models}
\index{apps.reporting.models (module)}
\index{Affiliation (class in apps.reporting.models)}

\begin{fulllineitems}
\phantomsection\label{generated/apps.reporting.models:apps.reporting.models.Affiliation}\pysiglinewithargsret{\strong{class }\code{apps.reporting.models.}\bfcode{Affiliation}}{\emph{*args}, \emph{**kwargs}}{}
Bases: \code{datamining.apps.profiles.models.BaseModel}

The \code{Affiliation} objects are both useful and pervasive in the current DataMYNE
system.  In almost all cases, they have superseded the use of the \code{ManyToManyField}
for the \emph{{}`{}`Person{}`{}` to other objects} relationships.  They have the following
advantages:
\begin{itemize}
\item {} 
They are generic.  This means we do not need to redefine the relationship field on 
every new object we make.  Instead, we can assume that \textbf{any} new model will be 
able to form a many-to-many relationship to a person via an \code{Affiliation}.

\item {} 
They have a \code{Role}.  We can therefore create several different kinds of
relationships between a \code{Person} and an object.  For example, a \code{Committee}
can have both a chairperson and a member.

\item {} 
They have a start and end date.  This allows us to maintain old relationships,
and embargo new relationships, without have to delete links.  This is useful
historically and for data-mining purposes.  For example, we can create a history
of all of a \code{Committee}`s chairs as far back as we like.
\begin{itemize}
\item {} 
The use of the \code{begin} and \code{retire} methods is encouraged in maintaining
current and past affiliations.  An \code{embargo} method would probably also be
useful for maintaining future affiliations (e.g. an incoming committee chair.)

\end{itemize}

\end{itemize}

\index{Affiliation.DoesNotExist}

\begin{fulllineitems}
\phantomsection\label{generated/apps.reporting.models:apps.reporting.models.Affiliation.DoesNotExist}\pysigline{\strong{exception }\bfcode{DoesNotExist}}{}
Bases: \code{django.core.exceptions.ObjectDoesNotExist}

\end{fulllineitems}


\index{Affiliation.MultipleObjectsReturned}

\begin{fulllineitems}
\phantomsection\label{generated/apps.reporting.models:apps.reporting.models.Affiliation.MultipleObjectsReturned}\pysigline{\strong{exception }\code{Affiliation.}\bfcode{MultipleObjectsReturned}}{}
Bases: \code{django.core.exceptions.MultipleObjectsReturned}

\end{fulllineitems}


\index{begin() (apps.reporting.models.Affiliation method)}

\begin{fulllineitems}
\phantomsection\label{generated/apps.reporting.models:apps.reporting.models.Affiliation.begin}\pysiglinewithargsret{\code{Affiliation.}\bfcode{begin}}{}{}
\end{fulllineitems}


\index{content\_object (apps.reporting.models.Affiliation attribute)}

\begin{fulllineitems}
\phantomsection\label{generated/apps.reporting.models:apps.reporting.models.Affiliation.content_object}\pysigline{\code{Affiliation.}\bfcode{content\_object}}{}
Provides a generic relation to any object through content-type/object-id
fields.

\end{fulllineitems}


\index{content\_type (apps.reporting.models.Affiliation attribute)}

\begin{fulllineitems}
\phantomsection\label{generated/apps.reporting.models:apps.reporting.models.Affiliation.content_type}\pysigline{\code{Affiliation.}\bfcode{content\_type}}{}
\end{fulllineitems}


\index{created\_by (apps.reporting.models.Affiliation attribute)}

\begin{fulllineitems}
\phantomsection\label{generated/apps.reporting.models:apps.reporting.models.Affiliation.created_by}\pysigline{\code{Affiliation.}\bfcode{created\_by}}{}
\end{fulllineitems}


\index{get\_next\_by\_created\_at() (apps.reporting.models.Affiliation method)}

\begin{fulllineitems}
\phantomsection\label{generated/apps.reporting.models:apps.reporting.models.Affiliation.get_next_by_created_at}\pysiglinewithargsret{\code{Affiliation.}\bfcode{get\_next\_by\_created\_at}}{\emph{*moreargs}, \emph{**morekwargs}}{}
\end{fulllineitems}


\index{get\_next\_by\_updated\_at() (apps.reporting.models.Affiliation method)}

\begin{fulllineitems}
\phantomsection\label{generated/apps.reporting.models:apps.reporting.models.Affiliation.get_next_by_updated_at}\pysiglinewithargsret{\code{Affiliation.}\bfcode{get\_next\_by\_updated\_at}}{\emph{*moreargs}, \emph{**morekwargs}}{}
\end{fulllineitems}


\index{get\_previous\_by\_created\_at() (apps.reporting.models.Affiliation method)}

\begin{fulllineitems}
\phantomsection\label{generated/apps.reporting.models:apps.reporting.models.Affiliation.get_previous_by_created_at}\pysiglinewithargsret{\code{Affiliation.}\bfcode{get\_previous\_by\_created\_at}}{\emph{*moreargs}, \emph{**morekwargs}}{}
\end{fulllineitems}


\index{get\_previous\_by\_updated\_at() (apps.reporting.models.Affiliation method)}

\begin{fulllineitems}
\phantomsection\label{generated/apps.reporting.models:apps.reporting.models.Affiliation.get_previous_by_updated_at}\pysiglinewithargsret{\code{Affiliation.}\bfcode{get\_previous\_by\_updated\_at}}{\emph{*moreargs}, \emph{**morekwargs}}{}
\end{fulllineitems}


\index{person (apps.reporting.models.Affiliation attribute)}

\begin{fulllineitems}
\phantomsection\label{generated/apps.reporting.models:apps.reporting.models.Affiliation.person}\pysigline{\code{Affiliation.}\bfcode{person}}{}
\end{fulllineitems}


\index{retire() (apps.reporting.models.Affiliation method)}

\begin{fulllineitems}
\phantomsection\label{generated/apps.reporting.models:apps.reporting.models.Affiliation.retire}\pysiglinewithargsret{\code{Affiliation.}\bfcode{retire}}{}{}
\end{fulllineitems}


\index{role (apps.reporting.models.Affiliation attribute)}

\begin{fulllineitems}
\phantomsection\label{generated/apps.reporting.models:apps.reporting.models.Affiliation.role}\pysigline{\code{Affiliation.}\bfcode{role}}{}
\end{fulllineitems}


\index{unit\_permissions (apps.reporting.models.Affiliation attribute)}

\begin{fulllineitems}
\phantomsection\label{generated/apps.reporting.models:apps.reporting.models.Affiliation.unit_permissions}\pysigline{\code{Affiliation.}\bfcode{unit\_permissions}}{}
This class provides the functionality that makes the related-object
managers available as attributes on a model class, for fields that have
multiple ``remote'' values and have a GenericRelation defined in their model
(rather than having another model pointed \emph{at} them). In the example
``article.publications'', the publications attribute is a
ReverseGenericRelatedObjectsDescriptor instance.

\end{fulllineitems}


\end{fulllineitems}


\index{AffiliationCurrentManager (class in apps.reporting.models)}

\begin{fulllineitems}
\phantomsection\label{generated/apps.reporting.models:apps.reporting.models.AffiliationCurrentManager}\pysigline{\strong{class }\code{apps.reporting.models.}\bfcode{AffiliationCurrentManager}}{}
Bases: \code{django.db.models.manager.Manager}

This manager show affiliations that exist between two dates OR without
any dates OR where there is a previous start but not an end OR where
there is no start date but a future end date.

This affiliation manager, like all the others, allows for affiliations
to be begun or retired en masse.

\index{begin\_all() (apps.reporting.models.AffiliationCurrentManager method)}

\begin{fulllineitems}
\phantomsection\label{generated/apps.reporting.models:apps.reporting.models.AffiliationCurrentManager.begin_all}\pysiglinewithargsret{\bfcode{begin\_all}}{\emph{role}, \emph{content\_type}, \emph{object\_id}}{}
\end{fulllineitems}


\index{get\_query\_set() (apps.reporting.models.AffiliationCurrentManager method)}

\begin{fulllineitems}
\phantomsection\label{generated/apps.reporting.models:apps.reporting.models.AffiliationCurrentManager.get_query_set}\pysiglinewithargsret{\bfcode{get\_query\_set}}{}{}
\end{fulllineitems}


\index{retire\_all() (apps.reporting.models.AffiliationCurrentManager method)}

\begin{fulllineitems}
\phantomsection\label{generated/apps.reporting.models:apps.reporting.models.AffiliationCurrentManager.retire_all}\pysiglinewithargsret{\bfcode{retire\_all}}{\emph{role}, \emph{content\_type}, \emph{object\_id}}{}
\end{fulllineitems}


\end{fulllineitems}


\index{AffiliationFutureManager (class in apps.reporting.models)}

\begin{fulllineitems}
\phantomsection\label{generated/apps.reporting.models:apps.reporting.models.AffiliationFutureManager}\pysigline{\strong{class }\code{apps.reporting.models.}\bfcode{AffiliationFutureManager}}{}
Bases: {\hyperref[generated/apps.reporting.models:apps.reporting.models.AffiliationCurrentManager]{\code{apps.reporting.models.AffiliationCurrentManager}}}

This manager only lists affiliations whose start date is in the future.

\index{get\_query\_set() (apps.reporting.models.AffiliationFutureManager method)}

\begin{fulllineitems}
\phantomsection\label{generated/apps.reporting.models:apps.reporting.models.AffiliationFutureManager.get_query_set}\pysiglinewithargsret{\bfcode{get\_query\_set}}{}{}
\end{fulllineitems}


\end{fulllineitems}


\index{AffiliationPastManager (class in apps.reporting.models)}

\begin{fulllineitems}
\phantomsection\label{generated/apps.reporting.models:apps.reporting.models.AffiliationPastManager}\pysigline{\strong{class }\code{apps.reporting.models.}\bfcode{AffiliationPastManager}}{}
Bases: {\hyperref[generated/apps.reporting.models:apps.reporting.models.AffiliationCurrentManager]{\code{apps.reporting.models.AffiliationCurrentManager}}}

This manager only lists affiliations whose end date is in the past.

\index{get\_query\_set() (apps.reporting.models.AffiliationPastManager method)}

\begin{fulllineitems}
\phantomsection\label{generated/apps.reporting.models:apps.reporting.models.AffiliationPastManager.get_query_set}\pysiglinewithargsret{\bfcode{get\_query\_set}}{}{}
\end{fulllineitems}


\end{fulllineitems}


\index{Authority (class in apps.reporting.models)}

\begin{fulllineitems}
\phantomsection\label{generated/apps.reporting.models:apps.reporting.models.Authority}\pysiglinewithargsret{\strong{class }\code{apps.reporting.models.}\bfcode{Authority}}{\emph{*args}, \emph{**kwargs}}{}
Bases: \code{datamining.apps.profiles.models.BaseModel}

An \code{Authority} defines the control of a committee by an organizational
unit within the university (e.g. a \code{Division}).

\index{Authority.DoesNotExist}

\begin{fulllineitems}
\phantomsection\label{generated/apps.reporting.models:apps.reporting.models.Authority.DoesNotExist}\pysigline{\strong{exception }\bfcode{DoesNotExist}}{}
Bases: \code{django.core.exceptions.ObjectDoesNotExist}

\end{fulllineitems}


\index{Authority.MultipleObjectsReturned}

\begin{fulllineitems}
\phantomsection\label{generated/apps.reporting.models:apps.reporting.models.Authority.MultipleObjectsReturned}\pysigline{\strong{exception }\code{Authority.}\bfcode{MultipleObjectsReturned}}{}
Bases: \code{django.core.exceptions.MultipleObjectsReturned}

\end{fulllineitems}


\index{committee (apps.reporting.models.Authority attribute)}

\begin{fulllineitems}
\phantomsection\label{generated/apps.reporting.models:apps.reporting.models.Authority.committee}\pysigline{\code{Authority.}\bfcode{committee}}{}
\end{fulllineitems}


\index{content\_object (apps.reporting.models.Authority attribute)}

\begin{fulllineitems}
\phantomsection\label{generated/apps.reporting.models:apps.reporting.models.Authority.content_object}\pysigline{\code{Authority.}\bfcode{content\_object}}{}
Provides a generic relation to any object through content-type/object-id
fields.

\end{fulllineitems}


\index{content\_type (apps.reporting.models.Authority attribute)}

\begin{fulllineitems}
\phantomsection\label{generated/apps.reporting.models:apps.reporting.models.Authority.content_type}\pysigline{\code{Authority.}\bfcode{content\_type}}{}
\end{fulllineitems}


\index{created\_by (apps.reporting.models.Authority attribute)}

\begin{fulllineitems}
\phantomsection\label{generated/apps.reporting.models:apps.reporting.models.Authority.created_by}\pysigline{\code{Authority.}\bfcode{created\_by}}{}
\end{fulllineitems}


\index{get\_next\_by\_created\_at() (apps.reporting.models.Authority method)}

\begin{fulllineitems}
\phantomsection\label{generated/apps.reporting.models:apps.reporting.models.Authority.get_next_by_created_at}\pysiglinewithargsret{\code{Authority.}\bfcode{get\_next\_by\_created\_at}}{\emph{*moreargs}, \emph{**morekwargs}}{}
\end{fulllineitems}


\index{get\_next\_by\_updated\_at() (apps.reporting.models.Authority method)}

\begin{fulllineitems}
\phantomsection\label{generated/apps.reporting.models:apps.reporting.models.Authority.get_next_by_updated_at}\pysiglinewithargsret{\code{Authority.}\bfcode{get\_next\_by\_updated\_at}}{\emph{*moreargs}, \emph{**morekwargs}}{}
\end{fulllineitems}


\index{get\_previous\_by\_created\_at() (apps.reporting.models.Authority method)}

\begin{fulllineitems}
\phantomsection\label{generated/apps.reporting.models:apps.reporting.models.Authority.get_previous_by_created_at}\pysiglinewithargsret{\code{Authority.}\bfcode{get\_previous\_by\_created\_at}}{\emph{*moreargs}, \emph{**morekwargs}}{}
\end{fulllineitems}


\index{get\_previous\_by\_updated\_at() (apps.reporting.models.Authority method)}

\begin{fulllineitems}
\phantomsection\label{generated/apps.reporting.models:apps.reporting.models.Authority.get_previous_by_updated_at}\pysiglinewithargsret{\code{Authority.}\bfcode{get\_previous\_by\_updated\_at}}{\emph{*moreargs}, \emph{**morekwargs}}{}
\end{fulllineitems}


\index{unit\_permissions (apps.reporting.models.Authority attribute)}

\begin{fulllineitems}
\phantomsection\label{generated/apps.reporting.models:apps.reporting.models.Authority.unit_permissions}\pysigline{\code{Authority.}\bfcode{unit\_permissions}}{}
This class provides the functionality that makes the related-object
managers available as attributes on a model class, for fields that have
multiple ``remote'' values and have a GenericRelation defined in their model
(rather than having another model pointed \emph{at} them). In the example
``article.publications'', the publications attribute is a
ReverseGenericRelatedObjectsDescriptor instance.

\end{fulllineitems}


\end{fulllineitems}


\index{Committee (class in apps.reporting.models)}

\begin{fulllineitems}
\phantomsection\label{generated/apps.reporting.models:apps.reporting.models.Committee}\pysiglinewithargsret{\strong{class }\code{apps.reporting.models.}\bfcode{Committee}}{\emph{*args}, \emph{**kwargs}}{}
Bases: \code{datamining.apps.profiles.models.BaseModel}

A \code{Committee} is an official designated group of people who have a mandate and who
presumably meet regularly.  A committee may have a parent committee to which it reports.
It is more structured than an \code{Organization} and is attached to organization units
such as \code{Division} or \code{Program} via the \code{Authority} objects.

\index{Committee.DoesNotExist}

\begin{fulllineitems}
\phantomsection\label{generated/apps.reporting.models:apps.reporting.models.Committee.DoesNotExist}\pysigline{\strong{exception }\bfcode{DoesNotExist}}{}
Bases: \code{django.core.exceptions.ObjectDoesNotExist}

\end{fulllineitems}


\index{Committee.MultipleObjectsReturned}

\begin{fulllineitems}
\phantomsection\label{generated/apps.reporting.models:apps.reporting.models.Committee.MultipleObjectsReturned}\pysigline{\strong{exception }\code{Committee.}\bfcode{MultipleObjectsReturned}}{}
Bases: \code{django.core.exceptions.MultipleObjectsReturned}

\end{fulllineitems}


\index{accept\_invitation() (apps.reporting.models.Committee method)}

\begin{fulllineitems}
\phantomsection\label{generated/apps.reporting.models:apps.reporting.models.Committee.accept_invitation}\pysiglinewithargsret{\code{Committee.}\bfcode{accept\_invitation}}{\emph{invitation}}{}
\end{fulllineitems}


\index{affiliations (apps.reporting.models.Committee attribute)}

\begin{fulllineitems}
\phantomsection\label{generated/apps.reporting.models:apps.reporting.models.Committee.affiliations}\pysigline{\code{Committee.}\bfcode{affiliations}}{}
This class provides the functionality that makes the related-object
managers available as attributes on a model class, for fields that have
multiple ``remote'' values and have a GenericRelation defined in their model
(rather than having another model pointed \emph{at} them). In the example
``article.publications'', the publications attribute is a
ReverseGenericRelatedObjectsDescriptor instance.

\end{fulllineitems}


\index{authorities (apps.reporting.models.Committee attribute)}

\begin{fulllineitems}
\phantomsection\label{generated/apps.reporting.models:apps.reporting.models.Committee.authorities}\pysigline{\code{Committee.}\bfcode{authorities}}{}
\end{fulllineitems}


\index{created\_by (apps.reporting.models.Committee attribute)}

\begin{fulllineitems}
\phantomsection\label{generated/apps.reporting.models:apps.reporting.models.Committee.created_by}\pysigline{\code{Committee.}\bfcode{created\_by}}{}
\end{fulllineitems}


\index{get\_absolute\_url() (apps.reporting.models.Committee method)}

\begin{fulllineitems}
\phantomsection\label{generated/apps.reporting.models:apps.reporting.models.Committee.get_absolute_url}\pysiglinewithargsret{\code{Committee.}\bfcode{get\_absolute\_url}}{\emph{*moreargs}, \emph{**morekwargs}}{}
\end{fulllineitems}


\index{get\_next\_by\_created\_at() (apps.reporting.models.Committee method)}

\begin{fulllineitems}
\phantomsection\label{generated/apps.reporting.models:apps.reporting.models.Committee.get_next_by_created_at}\pysiglinewithargsret{\code{Committee.}\bfcode{get\_next\_by\_created\_at}}{\emph{*moreargs}, \emph{**morekwargs}}{}
\end{fulllineitems}


\index{get\_next\_by\_updated\_at() (apps.reporting.models.Committee method)}

\begin{fulllineitems}
\phantomsection\label{generated/apps.reporting.models:apps.reporting.models.Committee.get_next_by_updated_at}\pysiglinewithargsret{\code{Committee.}\bfcode{get\_next\_by\_updated\_at}}{\emph{*moreargs}, \emph{**morekwargs}}{}
\end{fulllineitems}


\index{get\_previous\_by\_created\_at() (apps.reporting.models.Committee method)}

\begin{fulllineitems}
\phantomsection\label{generated/apps.reporting.models:apps.reporting.models.Committee.get_previous_by_created_at}\pysiglinewithargsret{\code{Committee.}\bfcode{get\_previous\_by\_created\_at}}{\emph{*moreargs}, \emph{**morekwargs}}{}
\end{fulllineitems}


\index{get\_previous\_by\_updated\_at() (apps.reporting.models.Committee method)}

\begin{fulllineitems}
\phantomsection\label{generated/apps.reporting.models:apps.reporting.models.Committee.get_previous_by_updated_at}\pysiglinewithargsret{\code{Committee.}\bfcode{get\_previous\_by\_updated\_at}}{\emph{*moreargs}, \emph{**morekwargs}}{}
\end{fulllineitems}


\index{get\_unit() (apps.reporting.models.Committee method)}

\begin{fulllineitems}
\phantomsection\label{generated/apps.reporting.models:apps.reporting.models.Committee.get_unit}\pysiglinewithargsret{\code{Committee.}\bfcode{get\_unit}}{}{}
\end{fulllineitems}


\index{meetings (apps.reporting.models.Committee attribute)}

\begin{fulllineitems}
\phantomsection\label{generated/apps.reporting.models:apps.reporting.models.Committee.meetings}\pysigline{\code{Committee.}\bfcode{meetings}}{}
This class provides the functionality that makes the related-object
managers available as attributes on a model class, for fields that have
multiple ``remote'' values and have a GenericRelation defined in their model
(rather than having another model pointed \emph{at} them). In the example
``article.publications'', the publications attribute is a
ReverseGenericRelatedObjectsDescriptor instance.

\end{fulllineitems}


\index{parent (apps.reporting.models.Committee attribute)}

\begin{fulllineitems}
\phantomsection\label{generated/apps.reporting.models:apps.reporting.models.Committee.parent}\pysigline{\code{Committee.}\bfcode{parent}}{}
\end{fulllineitems}


\index{subcommittees (apps.reporting.models.Committee attribute)}

\begin{fulllineitems}
\phantomsection\label{generated/apps.reporting.models:apps.reporting.models.Committee.subcommittees}\pysigline{\code{Committee.}\bfcode{subcommittees}}{}
\end{fulllineitems}


\index{unit\_permissions (apps.reporting.models.Committee attribute)}

\begin{fulllineitems}
\phantomsection\label{generated/apps.reporting.models:apps.reporting.models.Committee.unit_permissions}\pysigline{\code{Committee.}\bfcode{unit\_permissions}}{}
This class provides the functionality that makes the related-object
managers available as attributes on a model class, for fields that have
multiple ``remote'' values and have a GenericRelation defined in their model
(rather than having another model pointed \emph{at} them). In the example
``article.publications'', the publications attribute is a
ReverseGenericRelatedObjectsDescriptor instance.

\end{fulllineitems}


\end{fulllineitems}


\index{Meeting (class in apps.reporting.models)}

\begin{fulllineitems}
\phantomsection\label{generated/apps.reporting.models:apps.reporting.models.Meeting}\pysiglinewithargsret{\strong{class }\code{apps.reporting.models.}\bfcode{Meeting}}{\emph{*args}, \emph{**kwargs}}{}
Bases: \code{datamining.apps.profiles.models.BaseModel}

A \code{Meeting} generically connects itself to any other object.  This allows other
models like \code{Committee} and \code{Organization} to use the same meeting code.

\index{Meeting.DoesNotExist}

\begin{fulllineitems}
\phantomsection\label{generated/apps.reporting.models:apps.reporting.models.Meeting.DoesNotExist}\pysigline{\strong{exception }\bfcode{DoesNotExist}}{}
Bases: \code{django.core.exceptions.ObjectDoesNotExist}

\end{fulllineitems}


\index{Meeting.MultipleObjectsReturned}

\begin{fulllineitems}
\phantomsection\label{generated/apps.reporting.models:apps.reporting.models.Meeting.MultipleObjectsReturned}\pysigline{\strong{exception }\code{Meeting.}\bfcode{MultipleObjectsReturned}}{}
Bases: \code{django.core.exceptions.MultipleObjectsReturned}

\end{fulllineitems}


\index{accept\_invitation() (apps.reporting.models.Meeting method)}

\begin{fulllineitems}
\phantomsection\label{generated/apps.reporting.models:apps.reporting.models.Meeting.accept_invitation}\pysiglinewithargsret{\code{Meeting.}\bfcode{accept\_invitation}}{\emph{invitation}}{}
\end{fulllineitems}


\index{content\_object (apps.reporting.models.Meeting attribute)}

\begin{fulllineitems}
\phantomsection\label{generated/apps.reporting.models:apps.reporting.models.Meeting.content_object}\pysigline{\code{Meeting.}\bfcode{content\_object}}{}
Provides a generic relation to any object through content-type/object-id
fields.

\end{fulllineitems}


\index{content\_type (apps.reporting.models.Meeting attribute)}

\begin{fulllineitems}
\phantomsection\label{generated/apps.reporting.models:apps.reporting.models.Meeting.content_type}\pysigline{\code{Meeting.}\bfcode{content\_type}}{}
\end{fulllineitems}


\index{created\_by (apps.reporting.models.Meeting attribute)}

\begin{fulllineitems}
\phantomsection\label{generated/apps.reporting.models:apps.reporting.models.Meeting.created_by}\pysigline{\code{Meeting.}\bfcode{created\_by}}{}
\end{fulllineitems}


\index{get\_absolute\_url() (apps.reporting.models.Meeting method)}

\begin{fulllineitems}
\phantomsection\label{generated/apps.reporting.models:apps.reporting.models.Meeting.get_absolute_url}\pysiglinewithargsret{\code{Meeting.}\bfcode{get\_absolute\_url}}{\emph{*moreargs}, \emph{**morekwargs}}{}
\end{fulllineitems}


\index{get\_next\_by\_created\_at() (apps.reporting.models.Meeting method)}

\begin{fulllineitems}
\phantomsection\label{generated/apps.reporting.models:apps.reporting.models.Meeting.get_next_by_created_at}\pysiglinewithargsret{\code{Meeting.}\bfcode{get\_next\_by\_created\_at}}{\emph{*moreargs}, \emph{**morekwargs}}{}
\end{fulllineitems}


\index{get\_next\_by\_end\_time() (apps.reporting.models.Meeting method)}

\begin{fulllineitems}
\phantomsection\label{generated/apps.reporting.models:apps.reporting.models.Meeting.get_next_by_end_time}\pysiglinewithargsret{\code{Meeting.}\bfcode{get\_next\_by\_end\_time}}{\emph{*moreargs}, \emph{**morekwargs}}{}
\end{fulllineitems}


\index{get\_next\_by\_start\_time() (apps.reporting.models.Meeting method)}

\begin{fulllineitems}
\phantomsection\label{generated/apps.reporting.models:apps.reporting.models.Meeting.get_next_by_start_time}\pysiglinewithargsret{\code{Meeting.}\bfcode{get\_next\_by\_start\_time}}{\emph{*moreargs}, \emph{**morekwargs}}{}
\end{fulllineitems}


\index{get\_next\_by\_updated\_at() (apps.reporting.models.Meeting method)}

\begin{fulllineitems}
\phantomsection\label{generated/apps.reporting.models:apps.reporting.models.Meeting.get_next_by_updated_at}\pysiglinewithargsret{\code{Meeting.}\bfcode{get\_next\_by\_updated\_at}}{\emph{*moreargs}, \emph{**morekwargs}}{}
\end{fulllineitems}


\index{get\_previous\_by\_created\_at() (apps.reporting.models.Meeting method)}

\begin{fulllineitems}
\phantomsection\label{generated/apps.reporting.models:apps.reporting.models.Meeting.get_previous_by_created_at}\pysiglinewithargsret{\code{Meeting.}\bfcode{get\_previous\_by\_created\_at}}{\emph{*moreargs}, \emph{**morekwargs}}{}
\end{fulllineitems}


\index{get\_previous\_by\_end\_time() (apps.reporting.models.Meeting method)}

\begin{fulllineitems}
\phantomsection\label{generated/apps.reporting.models:apps.reporting.models.Meeting.get_previous_by_end_time}\pysiglinewithargsret{\code{Meeting.}\bfcode{get\_previous\_by\_end\_time}}{\emph{*moreargs}, \emph{**morekwargs}}{}
\end{fulllineitems}


\index{get\_previous\_by\_start\_time() (apps.reporting.models.Meeting method)}

\begin{fulllineitems}
\phantomsection\label{generated/apps.reporting.models:apps.reporting.models.Meeting.get_previous_by_start_time}\pysiglinewithargsret{\code{Meeting.}\bfcode{get\_previous\_by\_start\_time}}{\emph{*moreargs}, \emph{**morekwargs}}{}
\end{fulllineitems}


\index{get\_previous\_by\_updated\_at() (apps.reporting.models.Meeting method)}

\begin{fulllineitems}
\phantomsection\label{generated/apps.reporting.models:apps.reporting.models.Meeting.get_previous_by_updated_at}\pysiglinewithargsret{\code{Meeting.}\bfcode{get\_previous\_by\_updated\_at}}{\emph{*moreargs}, \emph{**morekwargs}}{}
\end{fulllineitems}


\index{unit\_permissions (apps.reporting.models.Meeting attribute)}

\begin{fulllineitems}
\phantomsection\label{generated/apps.reporting.models:apps.reporting.models.Meeting.unit_permissions}\pysigline{\code{Meeting.}\bfcode{unit\_permissions}}{}
This class provides the functionality that makes the related-object
managers available as attributes on a model class, for fields that have
multiple ``remote'' values and have a GenericRelation defined in their model
(rather than having another model pointed \emph{at} them). In the example
``article.publications'', the publications attribute is a
ReverseGenericRelatedObjectsDescriptor instance.

\end{fulllineitems}


\end{fulllineitems}


\index{MeetingManager (class in apps.reporting.models)}

\begin{fulllineitems}
\phantomsection\label{generated/apps.reporting.models:apps.reporting.models.MeetingManager}\pysigline{\strong{class }\code{apps.reporting.models.}\bfcode{MeetingManager}}{}
Bases: \code{django.db.models.manager.Manager}

\index{daily\_occurrences() (apps.reporting.models.MeetingManager method)}

\begin{fulllineitems}
\phantomsection\label{generated/apps.reporting.models:apps.reporting.models.MeetingManager.daily_occurrences}\pysiglinewithargsret{\bfcode{daily\_occurrences}}{\emph{dt=None}, \emph{content\_type=None}, \emph{object\_id=None}}{}
Returns a queryset of for instances that have any overlap with a 
particular day.
\begin{itemize}
\item {} 
\code{dt} may be either a datetime.datetime, datetime.date object, or
\code{None}. If \code{None}, default to the current day.

\item {} 
\code{event} can be an \code{Event} instance for further filtering.

\end{itemize}

\end{fulllineitems}


\index{monthly\_occurrences() (apps.reporting.models.MeetingManager method)}

\begin{fulllineitems}
\phantomsection\label{generated/apps.reporting.models:apps.reporting.models.MeetingManager.monthly_occurrences}\pysiglinewithargsret{\bfcode{monthly\_occurrences}}{\emph{dt=None}, \emph{content\_type=None}, \emph{object\_id=None}}{}
Returns a queryset of for instances that have any overlap with a 
particular day.
\begin{itemize}
\item {} 
\code{dt} may be either a datetime.datetime, datetime.date object, or
\code{None}. If \code{None}, default to the current day.

\item {} 
\code{event} can be an \code{Event} instance for further filtering.

\end{itemize}

\end{fulllineitems}


\index{range\_occurences() (apps.reporting.models.MeetingManager method)}

\begin{fulllineitems}
\phantomsection\label{generated/apps.reporting.models:apps.reporting.models.MeetingManager.range_occurences}\pysiglinewithargsret{\bfcode{range\_occurences}}{\emph{start=None}, \emph{end=None}, \emph{content\_type=None}, \emph{object\_id=None}}{}
\end{fulllineitems}


\index{weekly\_occurrences() (apps.reporting.models.MeetingManager method)}

\begin{fulllineitems}
\phantomsection\label{generated/apps.reporting.models:apps.reporting.models.MeetingManager.weekly_occurrences}\pysiglinewithargsret{\bfcode{weekly\_occurrences}}{\emph{dt=None}, \emph{content\_type=None}, \emph{object\_id=None}}{}
Returns a queryset of for instances that have any overlap with a 
particular day.
\begin{itemize}
\item {} 
\code{dt} may be either a datetime.datetime, datetime.date object, or
\code{None}. If \code{None}, default to the current day.

\item {} 
\code{event} can be an \code{Event} instance for further filtering.

\end{itemize}

\end{fulllineitems}


\end{fulllineitems}


\index{Role (class in apps.reporting.models)}

\begin{fulllineitems}
\phantomsection\label{generated/apps.reporting.models:apps.reporting.models.Role}\pysiglinewithargsret{\strong{class }\code{apps.reporting.models.}\bfcode{Role}}{\emph{*args}, \emph{**kwargs}}{}
Bases: \code{datamining.apps.profiles.models.BaseModel}

A \code{Role} is a refinement of \code{Affiliation} between a \code{Person}
and another object.  A roles has:
\begin{description}
\item[{title}] \leavevmode
A role's title is simply that.  Good examples include ``creator,'' as in
``Jane Doe is the creator of Artwork X'' and ``chairperson,'' as in
``John Does is the chairperson of Committee ABC''

\item[{content\_type}] \leavevmode
A role must also have a content type.  This allows for there to be a 
distinct ``member'' role, for example, in affiliations to both a
\code{Committee} and an \code{Organization}.

\end{description}

\index{Role.DoesNotExist}

\begin{fulllineitems}
\phantomsection\label{generated/apps.reporting.models:apps.reporting.models.Role.DoesNotExist}\pysigline{\strong{exception }\bfcode{DoesNotExist}}{}
Bases: \code{django.core.exceptions.ObjectDoesNotExist}

\end{fulllineitems}


\index{Role.MultipleObjectsReturned}

\begin{fulllineitems}
\phantomsection\label{generated/apps.reporting.models:apps.reporting.models.Role.MultipleObjectsReturned}\pysigline{\strong{exception }\code{Role.}\bfcode{MultipleObjectsReturned}}{}
Bases: \code{django.core.exceptions.MultipleObjectsReturned}

\end{fulllineitems}


\index{affiliations (apps.reporting.models.Role attribute)}

\begin{fulllineitems}
\phantomsection\label{generated/apps.reporting.models:apps.reporting.models.Role.affiliations}\pysigline{\code{Role.}\bfcode{affiliations}}{}
\end{fulllineitems}


\index{content\_type (apps.reporting.models.Role attribute)}

\begin{fulllineitems}
\phantomsection\label{generated/apps.reporting.models:apps.reporting.models.Role.content_type}\pysigline{\code{Role.}\bfcode{content\_type}}{}
\end{fulllineitems}


\index{created\_by (apps.reporting.models.Role attribute)}

\begin{fulllineitems}
\phantomsection\label{generated/apps.reporting.models:apps.reporting.models.Role.created_by}\pysigline{\code{Role.}\bfcode{created\_by}}{}
\end{fulllineitems}


\index{get\_next\_by\_created\_at() (apps.reporting.models.Role method)}

\begin{fulllineitems}
\phantomsection\label{generated/apps.reporting.models:apps.reporting.models.Role.get_next_by_created_at}\pysiglinewithargsret{\code{Role.}\bfcode{get\_next\_by\_created\_at}}{\emph{*moreargs}, \emph{**morekwargs}}{}
\end{fulllineitems}


\index{get\_next\_by\_updated\_at() (apps.reporting.models.Role method)}

\begin{fulllineitems}
\phantomsection\label{generated/apps.reporting.models:apps.reporting.models.Role.get_next_by_updated_at}\pysiglinewithargsret{\code{Role.}\bfcode{get\_next\_by\_updated\_at}}{\emph{*moreargs}, \emph{**morekwargs}}{}
\end{fulllineitems}


\index{get\_previous\_by\_created\_at() (apps.reporting.models.Role method)}

\begin{fulllineitems}
\phantomsection\label{generated/apps.reporting.models:apps.reporting.models.Role.get_previous_by_created_at}\pysiglinewithargsret{\code{Role.}\bfcode{get\_previous\_by\_created\_at}}{\emph{*moreargs}, \emph{**morekwargs}}{}
\end{fulllineitems}


\index{get\_previous\_by\_updated\_at() (apps.reporting.models.Role method)}

\begin{fulllineitems}
\phantomsection\label{generated/apps.reporting.models:apps.reporting.models.Role.get_previous_by_updated_at}\pysiglinewithargsret{\code{Role.}\bfcode{get\_previous\_by\_updated\_at}}{\emph{*moreargs}, \emph{**morekwargs}}{}
\end{fulllineitems}


\index{unit\_permissions (apps.reporting.models.Role attribute)}

\begin{fulllineitems}
\phantomsection\label{generated/apps.reporting.models:apps.reporting.models.Role.unit_permissions}\pysigline{\code{Role.}\bfcode{unit\_permissions}}{}
This class provides the functionality that makes the related-object
managers available as attributes on a model class, for fields that have
multiple ``remote'' values and have a GenericRelation defined in their model
(rather than having another model pointed \emph{at} them). In the example
``article.publications'', the publications attribute is a
ReverseGenericRelatedObjectsDescriptor instance.

\end{fulllineitems}


\end{fulllineitems}


\index{delete\_area\_of\_study\_affiliations() (in module apps.reporting.models)}

\begin{fulllineitems}
\phantomsection\label{generated/apps.reporting.models:apps.reporting.models.delete_area_of_study_affiliations}\pysiglinewithargsret{\code{apps.reporting.models.}\bfcode{delete\_area\_of\_study\_affiliations}}{\emph{sender}, \emph{*args}, \emph{**kwargs}}{}
\end{fulllineitems}


\index{delete\_committee\_affiliations() (in module apps.reporting.models)}

\begin{fulllineitems}
\phantomsection\label{generated/apps.reporting.models:apps.reporting.models.delete_committee_affiliations}\pysiglinewithargsret{\code{apps.reporting.models.}\bfcode{delete\_committee\_affiliations}}{\emph{sender}, \emph{*args}, \emph{**kwargs}}{}
\end{fulllineitems}


\index{delete\_committee\_meetings() (in module apps.reporting.models)}

\begin{fulllineitems}
\phantomsection\label{generated/apps.reporting.models:apps.reporting.models.delete_committee_meetings}\pysiglinewithargsret{\code{apps.reporting.models.}\bfcode{delete\_committee\_meetings}}{\emph{sender}, \emph{*args}, \emph{**kwargs}}{}
\end{fulllineitems}


\index{delete\_department\_authorities() (in module apps.reporting.models)}

\begin{fulllineitems}
\phantomsection\label{generated/apps.reporting.models:apps.reporting.models.delete_department_authorities}\pysiglinewithargsret{\code{apps.reporting.models.}\bfcode{delete\_department\_authorities}}{\emph{sender}, \emph{*args}, \emph{**kwargs}}{}
\end{fulllineitems}


\index{delete\_division\_authorities() (in module apps.reporting.models)}

\begin{fulllineitems}
\phantomsection\label{generated/apps.reporting.models:apps.reporting.models.delete_division_authorities}\pysiglinewithargsret{\code{apps.reporting.models.}\bfcode{delete\_division\_authorities}}{\emph{sender}, \emph{*args}, \emph{**kwargs}}{}
\end{fulllineitems}


\index{delete\_organization\_affiliations() (in module apps.reporting.models)}

\begin{fulllineitems}
\phantomsection\label{generated/apps.reporting.models:apps.reporting.models.delete_organization_affiliations}\pysiglinewithargsret{\code{apps.reporting.models.}\bfcode{delete\_organization\_affiliations}}{\emph{sender}, \emph{*args}, \emph{**kwargs}}{}
\end{fulllineitems}


\index{delete\_organization\_meetings() (in module apps.reporting.models)}

\begin{fulllineitems}
\phantomsection\label{generated/apps.reporting.models:apps.reporting.models.delete_organization_meetings}\pysiglinewithargsret{\code{apps.reporting.models.}\bfcode{delete\_organization\_meetings}}{\emph{sender}, \emph{*args}, \emph{**kwargs}}{}
\end{fulllineitems}


\index{delete\_program\_affiliations() (in module apps.reporting.models)}

\begin{fulllineitems}
\phantomsection\label{generated/apps.reporting.models:apps.reporting.models.delete_program_affiliations}\pysiglinewithargsret{\code{apps.reporting.models.}\bfcode{delete\_program\_affiliations}}{\emph{sender}, \emph{*args}, \emph{**kwargs}}{}
\end{fulllineitems}


\index{delete\_program\_authorities() (in module apps.reporting.models)}

\begin{fulllineitems}
\phantomsection\label{generated/apps.reporting.models:apps.reporting.models.delete_program_authorities}\pysiglinewithargsret{\code{apps.reporting.models.}\bfcode{delete\_program\_authorities}}{\emph{sender}, \emph{*args}, \emph{**kwargs}}{}
\end{fulllineitems}


\index{delete\_school\_authorities() (in module apps.reporting.models)}

\begin{fulllineitems}
\phantomsection\label{generated/apps.reporting.models:apps.reporting.models.delete_school_authorities}\pysiglinewithargsret{\code{apps.reporting.models.}\bfcode{delete\_school\_authorities}}{\emph{sender}, \emph{*args}, \emph{**kwargs}}{}
\end{fulllineitems}


\index{index\_committee() (in module apps.reporting.models)}

\begin{fulllineitems}
\phantomsection\label{generated/apps.reporting.models:apps.reporting.models.index_committee}\pysiglinewithargsret{\code{apps.reporting.models.}\bfcode{index\_committee}}{\emph{sender}, \emph{*args}, \emph{**kwargs}}{}
\end{fulllineitems}


\index{index\_meeting() (in module apps.reporting.models)}

\begin{fulllineitems}
\phantomsection\label{generated/apps.reporting.models:apps.reporting.models.index_meeting}\pysiglinewithargsret{\code{apps.reporting.models.}\bfcode{index\_meeting}}{\emph{sender}, \emph{*args}, \emph{**kwargs}}{}
\end{fulllineitems}



\chapter{Reporting: Views}
\label{generated/apps.reporting.views:reporting-views}\label{generated/apps.reporting.views::doc}\label{generated/apps.reporting.views:module-apps.reporting.views}
\index{apps.reporting.views (module)}
\index{edit\_committee() (in module apps.reporting.views)}

\begin{fulllineitems}
\phantomsection\label{generated/apps.reporting.views:apps.reporting.views.edit_committee}\pysiglinewithargsret{\code{apps.reporting.views.}\bfcode{edit\_committee}}{\emph{request}, \emph{*args}, \emph{**kwargs}}{}
This view edits a \code{Committee} object.  See the model documentation
for a more complete description of what an organization represents.

Along with the \code{Organization} model, committees represent one of the most
complicated objects in terms of security.  This model should be refactored first
as part of a general clean up to remove edittable from the view code and port
it into the model code itself.

Also note the use of the \code{current} and \code{past} managers for the member
and chairperson affiliations.  This is one of the benefits of using the 
\code{Affiliation} object over a \code{ManyToManyField}: we can retain historical
information even after the connection is no longer active.  For example, 
we can know all of the previous chairs of a committee while still allowing 
the current ones to be the only recipients of security clearance, public display,
etc.

In addition, the \code{Affiliation} managers have the \code{begin} and \code{retire} methods
that allow connections to see easily set to current or past without having to 
rewrite complicated code.  ``Retirement'' is the preferred way for disposing of a current
affiliation.  It has the same effect as a deletion, while still retaining the 
connections for historical and data-mining purposes.

\end{fulllineitems}


\index{edit\_meeting() (in module apps.reporting.views)}

\begin{fulllineitems}
\phantomsection\label{generated/apps.reporting.views:apps.reporting.views.edit_meeting}\pysiglinewithargsret{\code{apps.reporting.views.}\bfcode{edit\_meeting}}{\emph{request}, \emph{*args}, \emph{**kwargs}}{}
This view edits a meeting.

Note that meetings, currently, can be attached to any object, specifically the
\code{Organization} and \code{Committee} objects.  This creates a rather complex security
situation, in that the \code{Meeting} mode is not only capable of having permissions 
assigned to it directly but, logically, is also subject to the admin permissions for
the \code{Organization} and \code{Committee}.  As with other models, this should be 
refactored, to as great an extent as possible, into the models themselves.

\end{fulllineitems}


\index{list\_committees\_by\_school() (in module apps.reporting.views)}

\begin{fulllineitems}
\phantomsection\label{generated/apps.reporting.views:apps.reporting.views.list_committees_by_school}\pysiglinewithargsret{\code{apps.reporting.views.}\bfcode{list\_committees\_by\_school}}{\emph{request}}{}
This view lists all of the committees of all the schools.

\end{fulllineitems}


\index{view\_committee() (in module apps.reporting.views)}

\begin{fulllineitems}
\phantomsection\label{generated/apps.reporting.views:apps.reporting.views.view_committee}\pysiglinewithargsret{\code{apps.reporting.views.}\bfcode{view\_committee}}{\emph{request}, \emph{committee\_id}}{}
This view displays an \code{Committee} object.  See the model documentation
for a more complete description of what an organization represents.

Along with the \code{Organization} model, committees represent one of the most
complicated objects in terms of security.  This model should be refactored first
as part of a general clean up to remove edittable from the view code and port
it into the model code itself.

It would be good to add a ``admin'' role to the committees, in
addition to the current ``chairperson'' and ``member'' roles.

Also note the use of the \code{current} and \code{past} managers for the member
and chairperson affiliations.  This is one of the benefits of using the 
\code{Affiliation} object over a \code{ManyToManyField}: we can retain historical
information even after the connection is no longer active.  For example, 
we can know all of the previous chairs of a committee while still allowing 
the current ones to be the only recipients of security clearance, public display,
etc.

\end{fulllineitems}


\index{view\_meeting() (in module apps.reporting.views)}

\begin{fulllineitems}
\phantomsection\label{generated/apps.reporting.views:apps.reporting.views.view_meeting}\pysiglinewithargsret{\code{apps.reporting.views.}\bfcode{view\_meeting}}{\emph{request}, \emph{meeting\_id}}{}
This view displays a meeting.

Note that meetings, currently, can be attached to any object, specifically the
\code{Organization} and \code{Committee} objects.  This creates a rather complex security
situation, in that the \code{Meeting} mode is not only capable of having permissions 
assigned to it directly but, logically, is also subject to the admin permissions for
the \code{Organization} and \code{Committee}.  As with other models, this should be 
refactored, to as great an extent as possible, into the models themselves.

\end{fulllineitems}



\chapter{Reporting: Forms}
\label{generated/apps.reporting.forms:module-apps.reporting.forms}\label{generated/apps.reporting.forms::doc}\label{generated/apps.reporting.forms:reporting-forms}
\index{apps.reporting.forms (module)}
Created on Apr 7, 2011

@author: Mike\_Edwards

\index{CommitteeAffiliationForm (class in apps.reporting.forms)}

\begin{fulllineitems}
\phantomsection\label{generated/apps.reporting.forms:apps.reporting.forms.CommitteeAffiliationForm}\pysiglinewithargsret{\strong{class }\code{apps.reporting.forms.}\bfcode{CommitteeAffiliationForm}}{\emph{data=None}, \emph{files=None}, \emph{auto\_id='id\_\%s'}, \emph{prefix=None}, \emph{initial=None}, \emph{error\_class=\textless{}class `django.forms.util.ErrorList'\textgreater{}}, \emph{label\_suffix=':'}, \emph{empty\_permitted=False}, \emph{instance=None}}{}
Bases: \code{django.forms.models.ModelForm}

\index{CommitteeAffiliationForm.Meta (class in apps.reporting.forms)}

\begin{fulllineitems}
\phantomsection\label{generated/apps.reporting.forms:apps.reporting.forms.CommitteeAffiliationForm.Meta}\pysigline{\strong{class }\bfcode{Meta}}{}~
\index{model (apps.reporting.forms.CommitteeAffiliationForm.Meta attribute)}

\begin{fulllineitems}
\phantomsection\label{generated/apps.reporting.forms:apps.reporting.forms.CommitteeAffiliationForm.Meta.model}\pysigline{\bfcode{model}}{}
alias of \code{Affiliation}

\end{fulllineitems}


\end{fulllineitems}


\index{media (apps.reporting.forms.CommitteeAffiliationForm attribute)}

\begin{fulllineitems}
\phantomsection\label{generated/apps.reporting.forms:apps.reporting.forms.CommitteeAffiliationForm.media}\pysigline{\code{CommitteeAffiliationForm.}\bfcode{media}}{}
\end{fulllineitems}


\end{fulllineitems}


\index{CommitteeForm (class in apps.reporting.forms)}

\begin{fulllineitems}
\phantomsection\label{generated/apps.reporting.forms:apps.reporting.forms.CommitteeForm}\pysiglinewithargsret{\strong{class }\code{apps.reporting.forms.}\bfcode{CommitteeForm}}{\emph{data=None}, \emph{files=None}, \emph{auto\_id='id\_\%s'}, \emph{prefix=None}, \emph{initial=None}, \emph{error\_class=\textless{}class `django.forms.util.ErrorList'\textgreater{}}, \emph{label\_suffix=':'}, \emph{empty\_permitted=False}, \emph{instance=None}}{}
Bases: \code{django.forms.models.ModelForm}

\index{CommitteeForm.Meta (class in apps.reporting.forms)}

\begin{fulllineitems}
\phantomsection\label{generated/apps.reporting.forms:apps.reporting.forms.CommitteeForm.Meta}\pysigline{\strong{class }\bfcode{Meta}}{}~
\index{model (apps.reporting.forms.CommitteeForm.Meta attribute)}

\begin{fulllineitems}
\phantomsection\label{generated/apps.reporting.forms:apps.reporting.forms.CommitteeForm.Meta.model}\pysigline{\bfcode{model}}{}
alias of \code{Committee}

\end{fulllineitems}


\end{fulllineitems}


\index{media (apps.reporting.forms.CommitteeForm attribute)}

\begin{fulllineitems}
\phantomsection\label{generated/apps.reporting.forms:apps.reporting.forms.CommitteeForm.media}\pysigline{\code{CommitteeForm.}\bfcode{media}}{}
\end{fulllineitems}


\end{fulllineitems}


\index{MeetingForm (class in apps.reporting.forms)}

\begin{fulllineitems}
\phantomsection\label{generated/apps.reporting.forms:apps.reporting.forms.MeetingForm}\pysiglinewithargsret{\strong{class }\code{apps.reporting.forms.}\bfcode{MeetingForm}}{\emph{data=None}, \emph{files=None}, \emph{auto\_id='id\_\%s'}, \emph{prefix=None}, \emph{initial=None}, \emph{error\_class=\textless{}class `django.forms.util.ErrorList'\textgreater{}}, \emph{label\_suffix=':'}, \emph{empty\_permitted=False}, \emph{instance=None}}{}
Bases: \code{django.forms.models.ModelForm}

\index{MeetingForm.Meta (class in apps.reporting.forms)}

\begin{fulllineitems}
\phantomsection\label{generated/apps.reporting.forms:apps.reporting.forms.MeetingForm.Meta}\pysigline{\strong{class }\bfcode{Meta}}{}~
\index{model (apps.reporting.forms.MeetingForm.Meta attribute)}

\begin{fulllineitems}
\phantomsection\label{generated/apps.reporting.forms:apps.reporting.forms.MeetingForm.Meta.model}\pysigline{\bfcode{model}}{}
alias of \code{Meeting}

\end{fulllineitems}


\end{fulllineitems}


\index{media (apps.reporting.forms.MeetingForm attribute)}

\begin{fulllineitems}
\phantomsection\label{generated/apps.reporting.forms:apps.reporting.forms.MeetingForm.media}\pysigline{\code{MeetingForm.}\bfcode{media}}{}
\end{fulllineitems}


\end{fulllineitems}



\chapter{Reporting: Handlers}
\label{generated/apps.reporting.handlers:module-apps.reporting.handlers}\label{generated/apps.reporting.handlers::doc}\label{generated/apps.reporting.handlers:reporting-handlers}
\index{apps.reporting.handlers (module)}
Created on Aug 18, 2010

@author: edwards

\index{CommitteeHandler (class in apps.reporting.handlers)}

\begin{fulllineitems}
\phantomsection\label{generated/apps.reporting.handlers:apps.reporting.handlers.CommitteeHandler}\pysigline{\strong{class }\code{apps.reporting.handlers.}\bfcode{CommitteeHandler}}{}
Bases: \code{piston.handler.BaseHandler}

This handler returns committees.

\index{model (apps.reporting.handlers.CommitteeHandler attribute)}

\begin{fulllineitems}
\phantomsection\label{generated/apps.reporting.handlers:apps.reporting.handlers.CommitteeHandler.model}\pysigline{\bfcode{model}}{}
alias of \code{Committee}

\end{fulllineitems}


\index{queryset() (apps.reporting.handlers.CommitteeHandler method)}

\begin{fulllineitems}
\phantomsection\label{generated/apps.reporting.handlers:apps.reporting.handlers.CommitteeHandler.queryset}\pysiglinewithargsret{\bfcode{queryset}}{\emph{request}}{}
\end{fulllineitems}


\end{fulllineitems}


\index{StaffHandler (class in apps.reporting.handlers)}

\begin{fulllineitems}
\phantomsection\label{generated/apps.reporting.handlers:apps.reporting.handlers.StaffHandler}\pysigline{\strong{class }\code{apps.reporting.handlers.}\bfcode{StaffHandler}}{}
Bases: \code{piston.handler.BaseHandler}

This handler returns staff members.

\index{model (apps.reporting.handlers.StaffHandler attribute)}

\begin{fulllineitems}
\phantomsection\label{generated/apps.reporting.handlers:apps.reporting.handlers.StaffHandler.model}\pysigline{\bfcode{model}}{}
alias of \code{Staff}

\end{fulllineitems}


\index{queryset() (apps.reporting.handlers.StaffHandler method)}

\begin{fulllineitems}
\phantomsection\label{generated/apps.reporting.handlers:apps.reporting.handlers.StaffHandler.queryset}\pysiglinewithargsret{\bfcode{queryset}}{\emph{request}}{}
\end{fulllineitems}


\end{fulllineitems}



\chapter{Mobile: Views}
\label{generated/apps.mobile.views:module-apps.mobile.views}\label{generated/apps.mobile.views::doc}\label{generated/apps.mobile.views:mobile-views}
\index{apps.mobile.views (module)}
\index{home() (in module apps.mobile.views)}

\begin{fulllineitems}
\phantomsection\label{generated/apps.mobile.views:apps.mobile.views.home}\pysiglinewithargsret{\code{apps.mobile.views.}\bfcode{home}}{\emph{request}}{}
\end{fulllineitems}



\chapter{Indices and tables}
\label{index:indices-and-tables}\begin{itemize}
\item {} 
\emph{genindex}

\item {} 
\emph{modindex}

\item {} 
\emph{search}

\end{itemize}


\renewcommand{\indexname}{Python Module Index}
\begin{theindex}
\def\bigletter#1{{\Large\sffamily#1}\nopagebreak\vspace{1mm}}
\bigletter{a}
\item {\texttt{apps.mobile.views}}, \pageref{generated/apps.mobile.views:module-apps.mobile.views}
\item {\texttt{apps.profiles.backends}}, \pageref{generated/apps.profiles.backends:module-apps.profiles.backends}
\item {\texttt{apps.profiles.fields}}, \pageref{generated/apps.profiles.fields:module-apps.profiles.fields}
\item {\texttt{apps.profiles.forms}}, \pageref{generated/apps.profiles.forms:module-apps.profiles.forms}
\item {\texttt{apps.profiles.handlers}}, \pageref{generated/apps.profiles.handlers:module-apps.profiles.handlers}
\item {\texttt{apps.profiles.lookups}}, \pageref{generated/apps.profiles.lookups:module-apps.profiles.lookups}
\item {\texttt{apps.profiles.models}}, \pageref{generated/apps.profiles.models:module-apps.profiles.models}
\item {\texttt{apps.profiles.views}}, \pageref{generated/apps.profiles.views:module-apps.profiles.views}
\item {\texttt{apps.reporting.forms}}, \pageref{generated/apps.reporting.forms:module-apps.reporting.forms}
\item {\texttt{apps.reporting.handlers}}, \pageref{generated/apps.reporting.handlers:module-apps.reporting.handlers}
\item {\texttt{apps.reporting.models}}, \pageref{generated/apps.reporting.models:module-apps.reporting.models}
\item {\texttt{apps.reporting.views}}, \pageref{generated/apps.reporting.views:module-apps.reporting.views}
\end{theindex}

\renewcommand{\indexname}{Index}
\printindex
\end{document}
